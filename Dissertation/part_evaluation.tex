\chapter{Результаты экспериментов}
\label{chapter_evaluation}

Было реализовано две основные модификации анализа с раздельным рассмотрением потоков с эффектами окружения
\begin{enumerate}
\item $\mathbb{C_1}(\mathbb{L}, \mathbb{CS}, \mathbb{T}, \mathbb{S}, \mathbb{P})$ -- включает анализ потоков, ограниченный количеством создаваемых потоков;
\item $\mathbb{C_2}(\mathbb{L}, \mathbb{CS}, \mathbb{S}, \mathbb{P})$ -- поддерживает неограниченное создание потоков;
\end{enumerate}
с объединением эффектов окружения анализа предикатов $\mathbb{P}$ при равных состояниях другив видов анализа $\mathbb{T}$, $\mathbb{S}$
(то есть, равных точках программы, стеках вызова, потоках и множествах захваченных блокировок),\\
где
\begin{itemize}
\item Анализ точек программы $\mathbb{L}$~(раздел \ref{sect_location_analysis})
\item Анализ стеков вызова $\mathbb{CS}$~(not described in the paper, remove it?)
\item Анализ потоков $\mathbb{T}$~(раздел \ref{sect_thread_analysis}),
\item Анализ примитивов синхронизации $\mathbb{S}$%~(раздел \ref{sect_lock_analysis}), 
с опцией 
\begin{enumerate}
\item {\em EmptyLockset=true/false}
\begin{itemize}
\item true -- отключить эффекты окружения, используя $empty$ эффект окружения (всегда является совместным)
\item false -- использовать множества захваченных блокировок, как эффект окружения (проверять пересечение захваченных блокировок при вычислении совместности).
\end{itemize}
\end{enumerate}

\item Анализ предикатов $\mathbb{P}$~(раздел \ref{sect_predicate_analysis}).
с опциями:
\begin{itemize}
\item {\em Merge=Join}. Объединение первой (условие) и второй (формула перехода) части эффекта окружения.
\item {\em Merge=Eq}. Объединение эффектов окружения при одинаковом условии (то есть, при равных предикатных абстракциях).
\item {\em Merge=Sep}. Не объединять эффекты никогда.
\end{itemize}
\end{itemize}

TODO: add $\mathbb{C_2}$.
Now only $\mathbb{C_1}$ is presented.

Подходы для сравнения
\begin{enumerate}
\item Варианты анализа с эффектами окружения
\begin{enumerate}
\item {\textbf EmptyLockset} -- отключение условие на проверку множества захваченных блокировок {\em EmptyLockset=true} и {\em Merge=Join} опции в анализе предикатов.
\item Включенная опция {\em EmptyLockset=false} с вариантами объединения для анализа предикатов
\begin{enumerate}
\item {\textbf MergeJoin}. Объединять обе части эффекта окружения({\em Merge=Join}). 
\item {\textbf MergeEq}. Объединять эффекты только для равных условий ({\em Merge=Eq}).
\item {\textbf MergeSep}. Не объединять эффекты ({\em Merge=Sep}). 
\end{enumerate}
\end{enumerate}
\item {\textbf Threading}. ThreadingCPA
described in~\cite{MEMICS16-Multi-Threaded} uses Classic CPA (see section~\ref{sect_classic_cpa}).
\item Other tools YogarCBMC, SMACK, CSeq, ESBMC
\end{enumerate}

%----------------------------------------------------
\subsection{Evaluation on Linux Device Drivers}
We have 32 benchmarks prepared from 16 known bugs found during GSoC 2017 http://linuxtesting.org/28-08-2017
(to be added to the SV-COMP subcategory SoftwareSystems Concurrency). For each bug we have {\em false} (with reachable error) and {\em true} (fixed) benchmark version. TODO: How they are prepared?

\begin{center}
  \begin{table}[h]\footnotesize
    \caption{Evaluation on Linux Device Drivers}
    \begin{tabular}{ | l | c | c | c | c | c | c |}
      \hline
       Approach       & MergeJoin   & MergeEq & MergeSep & EmptyLockset & Threading & Other tool \\ \hline
      False verdicts & 12      & 12          & 12        & 23         & 3 & ?\\ 
  \hspace{0.5cm} Correct false   & 12 & 12   & 12        & 12         & 2 & ? \\ 
  \hspace{0.5cm} Incorrect false & 0 & 0   & 0        & 11         & 1 & ? \\ \hline
      True verdicts  & 12      & 12          & 11        & 1         & 0 & ? \\ 
  \hspace{0.5cm} Correct true   & 12 & 12    & 11        & 1         & 0 & ? \\ 
  \hspace{0.5cm} Incorrect true & 0 & 0    & 0        & 0         & 0 & ? \\ \hline
      Unknowns       & 8      & 8          & 9        & 8         & 29 & ? \\ \hline
      CPU time (s)   & 10 200 & 9 960      & 11 000   & 11 000    & 23 500 & ? \\
      Wall time (s)  & 9 060  & 9 040      & 9 650    & 9 640     & 22 200 & ? \\
      \hline
    \end{tabular}
  \end{table}
\end{center}

{\em Discussion.}

{\textbf EmptyLockset} -- without lockset condition you can not prove conditions on global variables under locks. It produces too much incorrect false verdicts.

{\textbf MergeJoin}, {\textbf MergeEq}, {\textbf MergeSep} all having the similar results. {\textbf MergeSep} does not prove one true and a bit slower.

%To see how much time/resources is spent on applying inference objects we have collected the following characteristics:
%\begin{itemize}
%\item {\bf inference objects} -- number of inference objects in the final reached set;
%\item {\bf merges} -- number of times when merge for inference objects returns different state;
%\item {\bf refinements} -- of predicate analysis;
%\item {\bf states from objects} -- number of inference object applications in transfer;
%\item {\bf stops} -- stop operator for inference objects returns true;
%\item {\bf passes} -- stop operator for inference objects returns false;
%\item {\bf total states} -- number of states in the final reached set.
%\end{itemize}

{\em See full tables of results in repository.}

{\textbf Threading} work too slow for the device drivers, so all benchmarks except three are timeouts. TODO: What is the reason of incorrect false?

TODO: Preliminary results of the other tools.
YogarCBMC -- segmentation fails and incorrect results.
SMACK -- timeouts.

%----------------------------------------------------
\subsection{Evaluation on SV-COMP Benchmarks}
Category {\em ConcurrencySafety}.

For SV-COMP we should generate a witness file, but thread-modular approach gives only a singlethreaded trace with inference objects. To find the full path to the error we need to find interleaving of the other threads.

{\em Future solution. The witness checker should follow the singlethreaded trace and should do context switches at the points of inference objects. It should explore the remaining threads to find a suitable interleaving.}

Right now witness checkers do not support thread identifiers in the error witnesses.

\begin{center}
  \begin{table}[h]\footnotesize
    \caption{Evaluation on SV-COMP benchmarks}
    \begin{tabular}{ | l | c | c | c | c | c | c |}
      \hline
      Approach         & MergeJoin   & MergeEq & MergeSep & EmptyLockset & Threading & Other tool \\ \hline
      False verdicts & 988    & 988        & 988      & 989       & 769       & ?\\ 
  \hspace{0.5cm} Correct false & 789 & 789 & 789      & 789       & 767       & ?\\ 
  \hspace{0.5cm} Incorrect false & 199 & 199 & 199    & 200       & 2         & ? \\ \hline
      True verdicts  & 33      & 33        & 33       & 32        & 163       & ? \\ 
  \hspace{0.5cm} Correct true & 33 & 33    & 33       & 32        & 163       & ? \\
  \hspace{0.5cm} Incorrect true & 0 & 0    & 0        & 0         & 0         & ? \\ \hline
      Unknowns       & 26     & 26         & 26       & 26        & 115       & ? \\ \hline
      CPU time (s)   & 28 400 & 38 200     & 46 900  & 39 900    & 63 000    & ? \\ 
      Wall time (s)  & 17 300 & 24 100     & 29 100   & 23 500    & 38 600    & ? \\
      \hline
    \end{tabular}
  \end{table}
\end{center}

{\em Discussion.}
The thread-modular approach do not produce incorrect false result confirming soundness of the approach. 

For the {\textbf EmptyLockset} configuration we do not see much more incorrect false results compare to the other options while on device drivers it gave more difference.
Hence not that many benchmarks are related to protecting data with locks.

{\textbf MergeJoin}, {\textbf MergeEq} and {\textbf MergeSep} have the same number of correct results. {\textbf MergeSep} is the most slow, then {\textbf MergeEq} is slower than the most fastest {\textbf MergeJoin}.

Comparing Thread-Modular with Inference Objects to Classic Analysis ({\textbf Threading}). All 33 true proved by thread-modular approach were not proved by {\textbf Threading}. That is one of the contributions of the method.
Thread-modular approach has many incorrect false.
Most of them (157) are due to unsupported atomic constructions like compare and swap {\em \_\_VERIFIER\_atomic\_CAS}. In 23 cases we do not support happens-before ordering by thread creation (the child thread can not interfere parent before creation). The current limitations of refinement procedure (see Section~\ref{sect_impl_refinement}) do not allow to discover interpolants to the thread of inference object in 12 cases. 7 cases are sensitive for exploring interleavings.

%
%\section{Таблица обыкновенная} \label{sect3_1}
%
%Так размещается таблица:
%
%\begin{table} [htbp]
%  \centering
%  \changecaptionwidth\captionwidth{15cm}
%  \caption{Название таблицы}\label{Ts0Sib}%
%  \begin{tabular}{| p{3cm} || p{3cm} | p{3cm} | p{4cm}l |}
%  \hline
%  \hline
%  Месяц   & \centering $T_{min}$, К & \centering $T_{max}$, К &\centering  $(T_{max} - T_{min})$, К & \\
%  \hline
%  Декабрь &\centering  253.575   &\centering  257.778    &\centering      4.203  &   \\
%  Январь  &\centering  262.431   &\centering  263.214    &\centering      0.783  &   \\
%  Февраль &\centering  261.184   &\centering  260.381    &\centering     $-$0.803  &   \\
%  \hline
%  \hline
%  \end{tabular}
%\end{table}
%
%\begin{table} [htbp]% Пример записи таблицы с номером, но без отображаемого наименования
%	\centering
%	\parbox{9cm}{% чтобы лучше смотрелось, подбирается самостоятельно
%        \captiondelim{}% должен стоять до самого пустого caption
%        \caption{}%
%        \label{tbl:test1}%
%        \begin{SingleSpace}
%    	\begin{tabular}{ | c | c | c | c |}
%    	\hline
%    	Оконная функция	& ${2N}$ & ${4N}$	& ${8N}$	\\ \hline
%    	Прямоугольное 	& 8.72 	 & 8.77		& 8.77		\\ \hline
%    	Ханна		& 7.96 	 & 7.93		& 7.93		\\ \hline
%    	Хэмминга	& 8.72 	 & 8.77		& 8.77		\\ \hline
%    	Блэкмана	& 8.72 	 & 8.77		& 8.77		\\ \hline
%    	\end{tabular}%
%    	\end{SingleSpace}
%	}
%\end{table}
%
%Таблица \ref{tbl:test2} "--- пример таблицы, оформленной в~классическом книжном варианте или~очень близко к~нему. \mbox{ГОСТу} по~сути не~противоречит. Можно ещё~улучшить представление, с~помощью пакета \verb|siunitx| или~подобного.
%
%\begin{table} [htbp]%
%    \centering
%	\caption{Наименование таблицы, очень длинное наименование таблицы, чтобы посмотреть как оно будет располагаться на~нескольких строках и~переноситься}%
%	\label{tbl:test2}% label всегда желательно идти после caption
%    \renewcommand{\arraystretch}{1.5}%% Увеличение расстояния между рядами, для улучшения восприятия.
%    \begin{SingleSpace}
%	\begin{tabular}{@{}@{\extracolsep{20pt}}llll@{}} %Вертикальные полосы не используются принципиально, как и лишние горизонтальные (допускается по ГОСТ 2.105 пункт 4.4.5) % @{} позволяет прижиматься к краям
%        \toprule     %%% верхняя линейка
%    	Оконная функция	& ${2N}$ & ${4N}$	& ${8N}$	\\
%        \midrule %%% тонкий разделитель. Отделяет названия столбцов. Обязателен по ГОСТ 2.105 пункт 4.4.5 
%    	Прямоугольное 	& 8.72 	 & 8.77		& 8.77		\\
%    	Ханна		& 7.96 	 & 7.93		& 7.93		\\
%    	Хэмминга	& 8.72 	 & 8.77		& 8.77		\\
%    	Блэкмана	& 8.72 	 & 8.77		& 8.77		\\
%        \bottomrule %%% нижняя линейка
%	\end{tabular}%
%   	\end{SingleSpace}
%\end{table}

%\newpage
%============================================================================================================================

\clearpage