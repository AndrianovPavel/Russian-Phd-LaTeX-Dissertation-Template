\input{Dissertation/appendixsetup}   % Предварительные настройки для правильного подключения Приложений
\chapter{Доказательство теорем} \label{AppendixA}

\section{Доказательство теоремы~\ref{thrm_soundness}}
\label{sect_thrm_soundness_proof}

Нам нужно доказать следующее утверждение:
$$\conc{CPA_{PS}(\mathbb{D}, e_0, \pi_0)} \supseteq Reach_{\tcarrow}(\conc{\{e_0\}})$$

Доказательство состоит из двух частей. Сначала доказывается, что алгоритм~\ref{cpata_algorithm_ps_without_waitlist} без использов
ания структуры waitlist аппроксимирует сверху множество достижимых конкретных состояний. Вторым шагом показывается, что алгоритм~\ref{cpata_algorithm_ps} эквивалентен алгоритму~\ref{cpata_algorithm_ps_without_waitlist}.

\begin{algorithm}[H]
% General Algorithm without waitlist
 \KwData{
 адаптивный статический анализ с частичными состояниями $\mathbb{D}$,
 {начальное абстрактное состояние $e_0$ с точностью $\pi_0 \in \Pi$},
 {множество $reached$ из элементов $E \times \Pi$}
 }
\KwResult{множество достижимых абстрактных состояний $reached$}
 $reached := \{e_0,\pi_0\} $\;
 
 \For{each $e \in reached$ }{
  \For{each $e'$: $e \tat{reached} (e', \pi')$ }{
   $(\widehat{e}, \widehat{\pi})=prec(e',\pi',reached)$\;
   \For{each $(e'', \pi'') \in reached$}{
    $e_{new} = merge(\widehat{e}, e'', \widehat{\pi})$\;
    \If{$e_{new} \neq e''$}{
     $reached := reached \setminus{\{(e'', \pi'')\}} \cup \{(e_{new}, \widehat{\pi})\}$\;
    }
   }
   \If{$!stop(\widehat{e}, reached, \widehat{\pi})$}{
    $reached := reached \cup \{(\widehat{e}, \widehat{\pi})\}$\;
   }
  }
 }%repeat

 \caption{$\overline{CPA}(\mathbb{D}, e_0, \pi_0)$}
 \label{cpata_algorithm_ps_without_waitlist}
\end{algorithm}

\begin{proof}
Нужно доказать, что $\conc{\overline{CPA}(\mathbb{D}, e_0, \pi_0)} \supseteq Reach_{\tcarrow}(\conc{\{e_0\}})$.
Определим рекурсивную функцию $\overline{CPA}_n(\mathbb{D}, e_0, \pi_0)$, которая будет вычислять множество достижимых состояний алгоритма на n-той итерации.
 Обозначим ее для краткости $reached_n \equiv \overline{CPA}_n(\mathbb{D}, e_0, \pi_0)$.

 \begin{equation}
 \label{cpa_recursive_algorithm}
 \begin{aligned}
 & reached_0 = \{e_0\} \\
 & s_n = \{e' \mid e \in reached_n \land e \tat{reached_n} e'\} \\
 & s'_n = \{\widehat{e} \mid \widehat{e} = prec(e',\pi',reached_n) \land e' \in s_n\} \\
 & \widehat{s}_n = \{\overline{e} \mid \exists e^1 \in reached_n, \widehat{e} \in s'_n \ldotp \overline{e} = merge(\widehat{e}, e^1, \pi))\} \\
 & \widehat{s}^1_n = reached_n \setminus \widehat{s}_n \\
 & \widehat{s}^2_n = \widehat{s}_n \setminus reached_n \\
 & \tilde{s}_n = \{e \mid e \in s' \land !(stop(e, reached_n, \pi))\} \\
 & reached_{n+1} = reached_n \setminus \widehat{s}^1_n \cup \widehat{s}^2_n \cup \tilde{s}_n = \widehat{s}_n \cup \tilde{s}_n\\
 \end{aligned}
 \end{equation}

Покажем, что 
 \begin{equation}
 \label{algorithm_lemma_1}
 \begin{aligned}
 \conc{reached_{n+1} \cup \hat{R}} \supseteq \conc{Reach(reached_n) \cup \hat{R}} \end{aligned}
 \end{equation}

 \begin{align*}
 & \conc{reach_{n+1} \cup \hat{R}} = (def.) = \conc{reached_n \setminus \widehat{s}^1_n \cup \widehat{s}^2_n \cup \tilde{s}_n \cup \hat{R}} \supseteq \\
 & (eq.~\ref{cpa_conc_eq_2},\ref{cpa_merge_eq}) \supseteq \conc{reached_n  \cup \tilde{s}_n \cup \hat{R}}  
  \supseteq (eq.~\ref{cpa_conc_eq_2},\ref{cpa_stop_eq})\\
  & \supseteq \conc{reached_n  \cup s'_n \cup \hat{R}}  \supseteq (eq.~\ref{cpa_conc_eq_2},\ref{cpa_prec_eq}) \supseteq \conc{reached_n  \cup s_n \cup \hat{R}} =\\
  &= (def.~\ref{reach_k}) = \conc{Reach(reached_n) \cup \hat{R}}
 \end{align*}
 
 Теперь докажем по индукции, что 
 
 \begin{equation}
 \label{algorithm_invariant_1}
 \begin{aligned}
 \forall \hat{R} \in E, k \in \mathbb{N}, \conc{reached_{n+k} \cup \hat{R}} \supseteq \conc{Reach^k(reached_n) \cup \hat{R}}
 \end{aligned}
 \end{equation}
 
 При $k=0$ отношение тривиально: $\conc{reached_n \cup \hat{R}} \supseteq \conc{reached_n \cup \hat{R}}$.
 Пусть теперь инвариант~\ref{algorithm_invariant_1} выполнен при некотором $k$, рассмотрим его при $k+1$:
 \begin{align*}
 & \conc{reach_{n+k+1} \cup \hat{R}} \supseteq (eq.~\ref{algorithm_lemma_1}) \supseteq \conc{Reach(reached_{n+k}) \cup \hat{R}} = \\
 & = (def.) = \conc{reached_{n+k} \cup \{e' \mid e \tat{reached_{n+k}} e' \land e \in reached_{n+k}} \cup \hat{R}\} \\
 & \supseteq (eq.~\ref{algorithm_invariant_1}) \supseteq \conc{Reach^k(reached_n) \cup \\
 & \{e' \mid e \tat{reached_{n+k}} e' \land e \in reached_{n+k}\} \cup \hat{R}} \supseteq (eq.~\ref{eq_suc}) \\
 & \supseteq \conc{Reach^k(reached_n) \cup \{e' \mid e \tat{Reach^k(reached_n)} e' \land e \in Reach^k(reached_n)\} \\ 
 & \cup \hat{R}} = (def.~\ref{reach_k}) = \conc{Reach^{k+1}(reached_n) \cup \hat{R}}
 \end{align*}

Таким образом, мы имеем,
\begin{equation}
 \label{algorithm_invariant_2}
 \begin{aligned}
 \conc{reach_{n+k}} \supseteq \conc{Reach^k(reached_n)} \supseteq (eq.~\ref{cpa_transfer_eq}) \supseteq \bigcup_{\tau \in \conc{reached_n}}{\{\tau' \mid \tau \tcarrow \tau'\}}
 \end{aligned}
 \end{equation}


Когда алгоритм заканчивает свое выполнение (доходит до неподвижной точки) $reached_n = reached_{n+1}$. 
Обозначим финальное множество, как $Reached = lim_{n \rightarrow \infty}(reached_n)$.
Нам остается доказать, что $\bigcup_{\tau \in \conc{Reached}}{\{\tau' \mid \tau \tcarrow \tau'\}} \supseteq Reach_{\tcarrow}(\conc{\{e_0\}})$.


Покажем, что $\forall \{\tau_i\} \in \mathcal{T}: \tau_0 \in \conc{\{e_0\}} \land \forall 1 \leq k \leq N: \tau_{k} \tcarrow \tau_{k+1} \implies \forall 1 \leq k \leq N: \tau_k \in \conc{Reached}$ по индукции.
Алгоритм~\ref{cpata_algorithm_ps_without_waitlist} не удаляет состояния, поэтому $e_0 \in Reached \lor e_0 \sqsubseteq e' \in Reached$.
Таким образом, $ \tau_0 \in \conc{Reached}$, то есть, базис индукции выполнен.
Предположим, что утверждение выполнено для некоторого $n \in \mathbb{N}$:

\begin{align}
& \forall 1 \leq k \leq n: \tau_k \in \conc{Reached} \nonumber \\ 
& \forall \tau \in \conc{Reached}: \{\tau' \mid \tau \tcarrow \tau'\} \subseteq (eq.~\ref{cpa_transfer_eq}) \subseteq \conc{Reach^k(Reached)} = \\
& = \conc{Reached} \implies \tau_{k+1} \in \{\tau' \mid \tau \tcarrow \tau'\} \subseteq \conc{Reached} \nonumber
\end{align}

Так, мы показали, что $\conc{\overline{CPA}(\mathbb{D}, e_0, \pi_0)} \supseteq Reach_{\tcarrow}(\conc{\{e_0\}})$.


Теперь нам нужно показать, что алгоритм использующий очередь состояний ($waitlist$) эквивалентен алгоритму без нее, то есть, что $\forall e \in \overline{CPA}(\mathbb{D}, e_0, \pi_0) \exists e' \in CPA(\mathbb{D}, e_0, \pi_0): e \sqsubseteq e'$.

Заметим, что итерации алгоритмов абсолютно одинаковые, поэтому изменения полученных состояний являются эквивалентными.
Единственным отличием является то, что $CPA(\mathbb{D}, e_0, \pi_0)$ итерируется по очереди состояний, а $\overline{CPA}(\mathbb{D}, e_0, \pi_0)$ -- по всем возможным подмножествам R.
Два алгоритма могут быть синхронизированы следующим образом: они начинают с одного и того же начального множества $reached$. 
Каждая итерация алгоритма $CPA(\mathbb{D}, e_0, \pi_0)$ повторяется алгоритмом $\overline{CPA}(\mathbb{D}, e_0, \pi_0)$. 
Так как они выполняют одинаковые действия и модифицируют множество $reached$ одинаково, получаемое множество $reached$ будет совпадать у обоих алгоритмов.
Когда $CPA(\mathbb{D}, e_0, \pi_0)$ заканчивает выполнение, $\overline{CPA}(\mathbb{D}, e_0, \pi_0)$ продолжает выполнять все оставшиеся итерации.

Теперь нужно показать, что рассмотренных итераций алгоритма $CPA(\mathbb{D}, e_0, \pi_0)$ будет достаточно для построения финального множества $reached$.

По аналогии с предыдущим пунктом нам потребуется рекурсивное определение функций $reached_n$ и $waitlist_n$:
 \begin{equation}
 \label{cpa_recursive_algorithm_2}
 \begin{aligned}
 & reached_0 = \{e_0\} \\
 & waitlist_0 = \{e_0\} \\
 & e_n = get(waitlist_n) \text{ выдает первое состояние в очереди} \\
 & s_n = \{e'_n \mid e_n \tat{reached_n} e'_n\} \\
 & s'_n = \{\widehat{e}_n \mid \widehat{e}_n = prec(e'_n,\pi',reached_n) \land e'_n \in s_n\} \\
 & \widehat{s}_n = \{\overline{e_n} \mid \exists e^1_n \in reached_n, \widehat{e}_n \in s'_n \ldotp \overline{e_n} = merge(\widehat{e}_n, e^1_n, \pi))\} \\
 & \widehat{s}^1_n = reached_n \setminus \widehat{s}_n \\
 & \widehat{s}^2_n = \widehat{s}_n \setminus reached_n \\
 & \tilde{s}_n = \{e_n \mid e_n \in s'_n \land !(stop(e_n, reached_n, \pi))\} \\
 & reached_{n+1} = reached_n \setminus \widehat{s}^1_n \cup \widehat{s}^2_n \cup \tilde{s}_n = \widehat{s}_n \cup \tilde{s}_n\\
 & waitlist_{n+1} = waitlist_n \setminus \{e_n\} \setminus \widehat{s}^1_n \cup \widehat{s}^2_n \cup \tilde{s}_n
 \end{aligned}
 \end{equation}

Докажем следующий инвариант для всех итераций алгоритмов.

\begin{equation}
\label{cpa_algorithm_equivalance_invariant}
\begin{aligned}
& \forall e \in reached_n: \\
& e \in waitlist_n \lor (\forall \widehat{R} \subseteq E : \conc{\{e' \mid e \tat{reached_n} e'\} \cup reached_n \cup \widehat{R}} \subseteq \conc{reached_n \cup \widehat{R}}) 
\end{aligned}
\end{equation}

Этот инвариант означает, что для любого перехода из $reached_n$ либо он находится в $waitlist_n$, либо все следующие переходы тоже находятся в $reached_n$, то есть в процессе анализа переходы не теряются.

Докажем инвариант~\ref{cpa_algorithm_equivalance_invariant} по индукции.
Для начального шага инвариант выполнен: $reached_0 = \{e_0\} \land waitlist_0 = \{e_0\} \implies e_0 \in waitlist_0$.
Пусть теперь инвариант выполнен для некоторой итерации $k$. Рассмотрим следующую итерацию $k+1$ и возьмем случайный переход $e \in reached_{k+1}$.
Возможны четыре варианта:

\begin{enumerate}
\item $e \in (reached_{k+1} \setminus reached_n) \setminus \{e_n\}$. Для этого состояния ничего не меняется, для него был выполнен инвариант на предыдущей итерации, и будет выполнен на этой.
\item $e = e_k$. На прошлой итерации инвариант был выполнен, так как $e_k \in waitlist_k$, но $e_k \notin waitlist_{k+1}$.
По определению $reached_{k+1} = reached_k \setminus \widehat{s}^1_k \cup \widehat{s}^2_k \cup \tilde{s}_k$ и $\conc{reached_{k+1} \cup \widehat{R}} = \conc{reached_k \setminus \widehat{s}^1_k \cup \widehat{s}^2_k \cup \tilde{s}_k \cup \widehat{R}} \supseteq (eq.~\ref{cpa_merge_eq}) \supseteq \conc{reached_k \cup \tilde{s}_k \cup \widehat{R}} \supseteq (eq.~\ref{cpa_stop_eq}) \supseteq \conc{reached_k \cup s'_k \cup \widehat{R}} \supseteq (eq.~\ref{cpa_prec_eq}) \supseteq \conc{reached_k \cup \{e' \mid e \tat{reached_k} e'\} \cup \widehat{R}}$. Откуда следует $\conc{\{e' \mid e \tat{reached_{k+1}} e'\} \cup reached_{k+1} \cup \widehat{R}} \subseteq \conc{reached_{k+1} \cup \widehat{R}})$.
\item $e \in \widehat{s}^2_{k+1}$. По определению $waitlist_{k+1}$ это означает, что $\widehat{s}^2_{k+1} \subseteq waitlist_{k+1} \implies e \in waitlist_{k+1}$. А значит, инвариант выполнен.
\item $e \in \tilde{s}_k$. Опять же, по определению $waitlist_{k+1}$ это означает, что $\tilde{s}_k \subseteq waitlist_{k+1} \implies e \in waitlist_{k+1}$. А значит, инвариант выполнен.
\end{enumerate}

Инвариант~\ref{cpa_algorithm_equivalance_invariant} выполнен на всех итерациях алгоритма $CPA(\mathbb{D}, e_0, \pi_0)$, включая последнюю, в которой $waitlist_k = \emptyset$, $reached_k = Reached$, поэтому $\conc{\{e' \mid e \tat{Reached} e'\} \cup Reached} \subseteq \conc{Reached}$.
Отсюда следует, что все остальные операции алгоритма $\overline{CPA}(\mathbb{D}, e_0, \pi_0)$ являются бесполезными.
Действительно, пусть $\overline{CPA}(\mathbb{D}, e_0, \pi_0)$ нашел переход $e \in Reached: e \tat{Reached} e' \land e' \notin Reached$.
Используя инвариант мы получаем, что $\conc{\{e'\} \cup Reached} \subseteq \conc{Reached}$. 
То есть, этот абстрактный переход не дает новых конкретных переходов.
Отсюда следует, что

$\conc{\overline{CPA}(\mathbb{D}, e_0, \pi_0)} \subseteq (req.~\ref{cpa_conc_eq_2}) \subseteq \conc{CPA(\mathbb{D}, e_0, \pi_0)}$

Мы доказали следующее неравенство:

$$Reach_{\tcarrow}(\conc{\{e_0\}}) \subseteq \conc{\overline{CPA}(\mathbb{D}, e_0, \pi_0)} \subseteq \conc{CPA(\mathbb{D}, e_0, \pi_0)}$$

\end{proof}

\section{Доказательство условия~\ref{cpa_transfer_inner_req} для анализа без абстракции}
\label{sect_transfer_q_proof}

Нам нужно доказать условие~\ref{cpa_transfer_inner_req}, связывающее операторы $transfer$, $\epp$ и $apply$.
\begin{proof}

Рассмотрим случайный переход $\tau \tcarrow \tau'$, где $\tau=(c,t,g), \tau' = (c',t',g')$ и предположим, что 
$\exists t_0, t_1, \dots, t_n \in T, t_i \neq t_j$ \\
$\tau \in \epp_I((e_0,t_0), \{(e_1,t_1), \dots, (e_n, t_n)\})$.
По определению~\ref{epp_tm_na} это означает, что $s_i = (t_i, c(t_i),c_l(t_i), c_g,c_s)$. 
По определению $\tatarrow_Q$ существует состояние $s_0' = (t_0, c'(t_0),c_l'(t_0), c'_g,c'_s)$.
% Кроме того, переход e'_0 может быть спроецирован: $e'_0|_p = e'_p = (s, t, (gl', s', gl'', s''))$.
Так как $\{e_i\}$ были совместны  (оператор $check_C(\{e_0,\dots,e_n\})$, это означает, что все переходы в окружении соответствует переходу: $\forall 1 \le i \le n: q_i = (gl, s, gl', s')$.
Таким образом, следующие состояния имеют вид $e_i \tatarrow e'_i$, $e'_i = (s'_i, t_i, q'_i)$, где $s'_i = (t_i, c(t_i),c_l(t_i), c'_g,c'_s)$.
То есть, глобальные части других частичных состояний меняются в соответствии с переходом в основном потоке.
При этом по определению $\tatarrow_Q$, который выдает все возможные переходы, $\exists 0 \le k \le n: g' = q'_k$, то есть новый переход в потоке должен содержатся среди полученных после переходов в окружении. 
Зафиксируем это значение $k$ и рассмотрим проекцию этого нового перехода в потоке: $e'_k|_p = e'_p = (s'_k, t_k, (gl', s', gl'', cs'')) $.
Глобальное часть состояний $s'_i$ была получена с помощью одинакового перехода, поэтому она удовлетворяет условию совместности $compatible_Q$. 
Поэтому можно применить оператор $apply$ $\forall 0 \le i \le n, i \neq k: \tilde{e_i} = apply(e'_i, e'_p) = (s'_i, t_i, (gl', s', gl'', cs''))$.
И тогда, состояния $e'_k, \tilde{e'_0}, \dots, \tilde{e'_{k-1}}, \tilde{e'_{k+1}}, \dots, \tilde{e'_n}$ удовлетворяют условию $check_C$. 
А значит, могут быть объединены в глобальный переход оператором $\epp$, при этом $\tau'$ будет в этом множестве по его построению.

В доказательстве опущен второй случай с операцией создания потока thread\_create. 
Рассуждения полностью повторяют описанные выше с той лишь поправкой, что они проводятся для $n+1$ элемента.
\end{proof}

\section{Доказательство условия~\ref{cpa_strengthen_req} оператора $strengthen$}
\label{sect_composite_strengthen_proof}

Покажем, что требование~\ref{cpa_strengthen_req} выполнено для простого оператора $\downarrow$ (определение~\ref{simple_strengthen_def}).

\begin{proof}
Дакажем условие от противного.
Предположим, что 
\begin{align*}
& \exists \tau \in \mathcal{T}, e_0, \dots, e_n \in E, t_0, \dots, t_n \in T : \\
& \tau \in \epp_{\mathscr{C}}((e_0, t_0), \{(apply_{\mathscr{C}}(e_1, e_0), t_1), \dots, (apply_{\mathscr{C}}(e_n, e_0), t_n)\}) \\
& \tau \notin \epp_{\mathscr{C}}((\downarrow e_0, t_0), \{(apply_{\mathscr{C}}(\downarrow e_1, \downarrow e_0), t_1), \dots, (apply_{\mathscr{C}}(\downarrow e_n, \downarrow e_0), t_n)\}) 
\end{align*}
По определению~\ref{composite_epp_def} это означает, что существует некоторый внутренний CPA, для которого 
\begin{align*}
& \exists \tau \in \mathcal{T}, e^i_0, \dots, e^i_n \in E_I, t_0, \dots, t_n \in T : \\
& \tau \in \epp_{I}((e^i_0, t_0), \{(e^i_1, t_1), \dots, (e^i_n, t_n)\}) \\
& \tau \notin \epp_{I}((\tilde{e}^i_0, t_0), \{(apply_{I}(\tilde{e}^i_1, \tilde{e}^i_0), t_1), \dots, ((apply_{I}(\tilde{e}^i_n, \tilde{e}^i_0), t_n)\}) 
\end{align*}
Где $\downarrow e_j = (\tilde{e}^1_j, \dots, \tilde{e}^m_j)$.

Учитывая явный вид перехода $e=(s,q)$ и вид оператора $epp_I$(определение~\ref{epp_split}), а также то, что оператор усиления не меняет абстрактного состояния (определение~\ref{simple_strengthen_def}), получаем, что можно исключить из рассмотрения абстрактные состояния:
\begin{align*}
& \exists g \in G, e^i_0, \dots, e^i_n \in E_I, t_0, \dots, t_n \in T : \\
& g \in \econc{q^i_0} \cap \econc{q^i_1} \cap \dots \cap \econc{q^i_n} \\
& g \notin \econc{\tilde{q}^i_0} \cap \econc{\tilde{q}^i_1} \cap \dots \cap \econc{\tilde{q}^i_n} 
\end{align*}

Это означает, что $\exists 0 \le j \le n: q^i_j \neq \tilde{q}^i_j$. При этом не обязательно единственный.
По определению~\ref{simple_strengthen_def} отсюда следует, что $\exists g' \in G, g \neq g': \exists 0 \le j \le n: \tilde{q}^i_j = g'$, так как только в этом случае дуга может измениться.
Однако, по тому же определению~\ref{simple_strengthen_def} такое изменение возможно только если дуга $g'$ является единственно возможной, а значит, $\tau = (c, g', t)$, что противоречит условию $g' \neq g$. 
А значит,предположение было неверно, и условие~\ref{cpa_strengthen_req} выполнено.
\end{proof}

\section{Доказательство условия~\ref{cpa_transfer_inner_req} для CompositeCPA}
\label{sect_composite_transfer_proof}

Нам нужно показать, что требование~\ref{cpa_transfer_inner_req} выполнено для $\mathscr{C}$, если требование~\ref{cpa_transfer_composite_req} выполнены для всех его внутренних элементов. 

\begin{proof}
Рассмотрим случайный конкретный переход:
$$\tau \in  \epp_{\mathscr{C}}
\left(
\begin{pmatrix}
e_0 \\
t_0 
\end{pmatrix},
\left\{
\begin{pmatrix}
e_1 \\
t_1 
\end{pmatrix},
\dots ,
\begin{pmatrix}
e_m \\
t_m 
\end{pmatrix}
\right\}
\right)$$
где $e_i=(e^1_i, \dots, e^n_i)$.

По определению~\ref{composite_epp_def}, это означает что
$$\forall j: 1\le j \le m: \tau \in  \epp_{\Delta_j}
\left(
\begin{pmatrix}
e^j_0 \\
t_0 
\end{pmatrix},
\left\{
\begin{pmatrix}
e^j_1 \\
t_1 
\end{pmatrix},
\dots ,
\begin{pmatrix}
e^j_m \\
t_m 
\end{pmatrix}
\right\}
\right)$$

Возьмем случайный переход
$ \tau \tcarrow \tau', \tau=(c,g,t), g \in G, t \in T, $ и покажем, что 
$\exists e_0', \dots, e_m', \tilde{e}'_0, \dots, \tilde{e}'_{k-1}, \tilde{e}'_{k+1}, \dots, \tilde{e}'_m \in E$ и
$\tau' \in  \epp_{\mathscr{C}}
\left(
\begin{pmatrix}
e'_k \\
t_k 
\end{pmatrix},
\left\{
\begin{pmatrix}
\tilde{e'_0} \\
t_0 
\end{pmatrix},
\dots ,
\begin{pmatrix}
\tilde{e'_m} \\
t_m 
\end{pmatrix}
\right\}
\right)$ 
и при этом $\tilde{e}_0', \dots, \tilde{e}_m'$ будут получены из $e_0', \dots, e_m'$, а они -- из $e_0, \dots, e_m$ указанными в~\ref{cpa_transfer_inner_req} способами.

%enumerate
Сначала рассмотрим случай $g \neq thread\_create$

Используя требование~\ref{cpa_transfer_composite_req} для внутренних элементов, получаем, что 
\begin{align*}
& \forall 0 \le j \le m: \exists {e^j_0}', {e^j_1}', \dots, {e^j_n}' \in E: \\
& \forall 1 \leq i \leq n: e^j_i \tat{R} {e^j_i}': \exists 1 \le k \le n: t_k = t' \land \\
& \tau' \in \epp_I(({e^j_k}', t_k), \{(\tilde{e^j_i}, t_i) \mid \tilde{e^j_i} = apply({e^j_i}', {e^j_k}') \land i \neq k)\})
\end{align*}

Заметим, что $k$ является одним и тем же для всех внутренних элементов, так как есть только одно $t_k = t'$, а множество $t_0, t_1, \dots, t_n$ -- общее для всех CPA.
Поэтому можно объединить все внутренние переходы следующим способом:
$ \forall 1 \le i \le n: e_i' = ({e^1_i}', \dots, {e^m_i}')$
Таким образом, $\tau' \in \epp_{\mathscr{C}}((e_k', t_k), \{(\tilde{e_i}, t_i) \mid \tilde{e_i} = apply(e'_i, e'_k) \land i \neq k)\})$.
По условию~\ref{cpa_strengthen_req} $\tau' \in \epp_{\mathscr{C}}((\downarrow e_k', t_k), \{(\tilde{e_i}, t_i) \mid \tilde{e_i} = apply(\downarrow  e'_i, \downarrow  e'_k) \land i \neq k)\})$.

По определению $transfer$ в композитном анализе, получаем, что $\forall 1 \le i \le n: e_i \tatarrow \downarrow e_i'$, а по определению $apply$: $\forall 1 \le i \le n: \tilde{e_i} = apply(e_i', e_k') = (\tilde{e^0_i}, \dots, \tilde{e^m_i})$.
А это означает, что выполнено условие~\ref{cpa_transfer_inner_req}.
Это как раз то, что нам нужно было доказать. 

Рассмотрим случай $g = thread\_create$.

Используя требование~\ref{cpa_transfer_composite_req} для внутренних элементов, получаем, что 
\begin{align*}
& \exists e'_0, e'_1, \dots, e'_{n+1} \in E, \forall 0 \leq i \leq n: \\
& \begin{cases}
& (s_0, (l, tc_{parent}, l')) \tat{R} e'_0 \\
& (s_0, (l, tc_{child}, l')) \tat{R} e'_{n+1} \\
& e_i \tatarrow e'_i, \text{ если } i \neq 0
\end{cases} \\
& \exists 0 \le k \le n + 1:  t_k = t' \land\\
& \tau' \in \epp_I((e_k', t_k), \{(\tilde{e}_i, t_i) \mid \tilde{e}_i = apply(e_i', e'_k) \land i \neq k)\}) 
\end{align*}

Заметим, что $k$ является одним и тем же для всех внутренних элементов, так как есть только одно $t_k = t'$, а множество $t_0, t_1, \dots, t_n$ -- общее для всех CPA.
Поэтому можно объединить все внутренние переходы следующим способом:
$ \forall 1 \le i \le n+1: e_i' = ({e^1_i}', \dots, {e^m_i}')$.
Таким образом, $\tau' \in \epp_{\mathscr{C}}((e_k', t_k), \{(\tilde{e_i}, t_i) \mid \tilde{e_i} = apply(e'_i, e'_k) \land i \neq k)\})$.
По условию~\ref{cpa_strengthen_req} $\tau' \in \epp_{\mathscr{C}}((\downarrow e_k', t_k), \{(\tilde{e_i}, t_i) \mid \tilde{e_i} = apply(\downarrow  e'_i, \downarrow  e'_k) \land i \neq k)\})$.

По определению $transfer$ в композитном анализе, получаем, что $\forall 1 \le i \le n: e_i \tatarrow \downarrow e_i' \land e_0 \tatarrow \downarrow e'_{n+1}$, а по определению $apply$: $\forall 1 \le i \le n: \tilde{e_i} = apply(e_i', e_k') = (\tilde{e^0_i}, \dots, \tilde{e^m_i})$.
А это означает, что выполнено условие~\ref{cpa_transfer_inner_req}.
Это как раз то, что нам нужно было доказать. 
\end{proof}

\section{Доказательство условия~\ref{cpa_transfer_inner_req} для ThreadCPA}
\label{sect_transfer_thread_proof}

Нам нужно доказать, что определенные в разделе~\ref{sect_thread_analysis} операторы $transfer$ и $apply$ удовлетворяют условию~\ref{cpa_transfer_composite_req}.

\begin{proof}
Рассмотрим случайный конкретный переход:
$$\tau \in  \epp_{T}
\left(
\begin{pmatrix}
e_0 \\
t_0 
\end{pmatrix},
\left\{
\begin{pmatrix}
e_1 \\
t_1 
\end{pmatrix},
\dots ,
\begin{pmatrix}
e_m \\
t_m 
\end{pmatrix}
\right\}
\right)$$.

Если $op \neq thread\_create$, по определению $transfer$ $e'_i = (s_i, \top^T_T)$, то есть состояние не изменяется, а переходы рассматриваются все возможные. 
А значит, 
$$\tau' \in  \epp_{T}
\left(
\begin{pmatrix}
e'_0 \\
t_0 
\end{pmatrix},
\left\{
\begin{pmatrix}
e'_1 \\
t_1 
\end{pmatrix},
\dots ,
\begin{pmatrix}
e'_m \\
t_m 
\end{pmatrix}
\right\}
\right)$$

Если $op = tc_{Child} \lor op = tc_{Parent}$ состояние меняется только у дочернего потока: $e_{m+1} = (l_{\nu}, \top^T_T)$.
То есть, 
$$\tau' \in  \epp_{T}
\left(
\begin{pmatrix}
e'_k \\
t_k 
\end{pmatrix},
\left\{
\begin{pmatrix}
e'_0 \\
t_0 
\end{pmatrix},
\dots ,
\begin{pmatrix}
e'_m \\
t_m 
\end{pmatrix} ,
\begin{pmatrix}
e'_{m+1} \\
l_{\nu} 
\end{pmatrix}
\right\}
\right)$$

\end{proof}

\section{Доказательство условия~\ref{cpa_transfer_composite_req} для LocationCPA}
\label{sect_transfer_location_proof}

Нам нужно доказать, что определенные в разделе~\ref{sect_location_analysis} операторы $transfer$ и $apply$ удовлетворяют условию~\ref{cpa_transfer_composite_req}.

\begin{proof}
Рассмотрим случайный конкретный переход $\tau \tcarrow \tau'$ и такие абстрактные переходы $e_0, \dots, e_n$, что :
$$\tau \in \epp_{L}
\left(
\begin{pmatrix}
e_0 \\
t_0 
\end{pmatrix},
\left\{
\begin{pmatrix}
e_1 \\
t_1 
\end{pmatrix},
\dots ,
\begin{pmatrix}
e_m \\
t_m 
\end{pmatrix}
\right\}
\right)$$.

По определениям~\ref{epp_split},~\ref{location_epp_def} это означает, что $\tau = (c, g, t_0)$,  
$$ c_{pc} = 
\left\lbrace
\begin{array}{c}
t_0 \to l_0,\\
\dots\\
t_m \to l_m\\
\end{array}
\right\rbrace.$$
где $e_i = (s_i, q_i)$, $l_i \in loc(s_i)$, $q_0 \in G$, $q_0 = (l_0, op, l_0')$.

%TODO thread\_join
Рассмотрим случай $g \neq thread\_create$. После перехода изменяется счетчик комманд только нулевого потока: $l_0' \in loc(s_0')$. $$ c'_{pc} = 
\left\lbrace
\begin{array}{c}
t_0 \to l_0',\\
\dots,\\
t_m \to l_m\\
\end{array}
\right\rbrace.$$

Это означает, что $$c' \in \epp^S_L 
\begin{pmatrix}
\begin{pmatrix}
s_0 \\
t_0 
\end{pmatrix}& ,
\dots& ,
\begin{pmatrix}
s_n \\
t_n 
\end{pmatrix}
\end{pmatrix}$$

Теперь рассмотрим поток $k$ в котором совершается переход $\tau' = (c', g', t_k)$. 
По определению оператора $transfer$ $e_k \tatarrow e_k'$, $e_k' = (s_k', g_k)$, где $g_k$ -- включает в себя все дуги, по которым возможны переходы из данной точки программы, то есть и $g'$. 
По определению $compatible$ полученные состояния будут совместны, а значит, можно будет применить оператор $apply$, то есть
$\tau' \in \epp_I((e_k', t_k), \{(\tilde{e}_i, t_i) \mid \tilde{e}_i = apply(e_i', e'_k) \land i \neq k)\})$.

Для случая $g = tc_{Child}(l_\nu)$ или $g = tc_{Parent}(l_\nu)$ доказательство принципиально ничем не отличается от предыдущего случая, так как никакой новой семантике в этих дугах нет. 
\end{proof}

\section{Доказательство условия~\ref{cpa_transfer_composite_req} для PredicateCPA}
\label{sect_transfer_predicate_proof}

Нам нужно доказать, что определенные в разделе~\ref{sect_predicate_analysis} операторы $transfer$ и $apply$ удовлетворяют условию~\ref{cpa_transfer_composite_req}.

\begin{proof}

Рассмотрим случайное множество абстрактных переходов $e_0, e_1, \dots, e_n$ и конкретный переход $
\tau \in \epp_P
\left(
\begin{pmatrix}
e_0 \\
t_0 
\end{pmatrix},
\left\{
\begin{pmatrix}
e_1 \\
t_1 
\end{pmatrix},
\dots ,
\begin{pmatrix}
e_n \\
t_n 
\end{pmatrix}
\right\}
\right)$

Возьмем некоторый следующий переход $\tau \tcarrow \tau'$, где $\tau = (c, g, t_0)$
Нам нужно показать, что 
$\exists e_0', e_1', \dots, e_n' \in E_P$ и  $
\tau' \in \epp_P
\left(
\begin{pmatrix}
e_0 \\
t_0 
\end{pmatrix},
\left\{
\begin{pmatrix}
e_1 \\
t_1 
\end{pmatrix},
\dots ,
\begin{pmatrix}
e_n \\
t_n 
\end{pmatrix}
\right\}
\right)$.

Не будем подробно рассматривать случаи $g = thread\_create$
%, $g = thread\_join$
, так как этот переход не меняет состояния анализа предикатов и не меняет состояние памяти ($c_l$ и $c_g$).
А значит, $s_i = s_i'$.
Поэтому в этом случае $c' \in \epp^S_P
\left(
\begin{pmatrix}
s_k' \\
t_k 
\end{pmatrix},
\left\{
\begin{pmatrix}
s_0' \\
t_0 
\end{pmatrix},
\dots ,
\begin{pmatrix}
s_n' \\
t_n 
\end{pmatrix}
\right\}
\right) = 
\epp^S_P
\left(
\begin{pmatrix}
s_k \\
t_k 
\end{pmatrix},
\left\{
\begin{pmatrix}
s_0 \\
t_0 
\end{pmatrix},
\dots ,
\begin{pmatrix}
s_n \\
t_n 
\end{pmatrix}
\right\}
\right)$.

Рассмотрим теперь случай $op = assign(x, exp)$. Зафиксируем значение $k$ такое, что $t_k = t'$. Будем рассматривать переход в потоке из состояния $s_k$.

$e_k \tatarrow e_k'$, $e_k'|_p = (\theta_{X^{local},env}(s_k'),\theta_{X^{local},env}(SP_{q'}(s_k'))$. 
Начальные состояния $s_i$ были совместными, так как к ним был применим оператор $\epp$. 
Отсюда следует, что существовала общая модель $\hat{v_g} \models s_i$.
По определению оператора $assign$, который меняет только значение переменной $x$ на новое, $c_g' = \hat{v_g}'$, где $\hat{v_g}'$ совпадает с $\hat{v_g}$ для всех переменных, кроме $x$.
По определению оператора перехода $s_k' = SP_{op}(s_k)$, а $s_i' = s_i'[x \rightarrow \hat x] \land SP_{op}(true) = SP_{op}(s_i)$.
Таким образом, частичные состояния всех потоков меняются согласованно, и по определению $SP_{op}(\varphi)$ новое $\hat{v_g}'$ должно являться моделью для новых абстрактных состояний.
Теперь необходимо проверить, что для новых переходов будет выполнен оператор $apply$, то есть, переходы будут совместны.
Мы показали, что состояния останутся совместными, и осталось показать, что совместными останутся дуги, то есть, согласно~\ref{epp_split}, что $\forall g \in G: g \in \econc{g} \cap \econc{\theta_{X^{local},env}(SP_{g}(true))}$. 
Очевидно, что $g \in \econc{\theta_{X^{local},env}(SP_{g}(true))}$, и условие выполнено.
Таким образом мы показали, что условие~\ref{cpa_transfer_inner_req} выполнено для анализа предикатов.

\end{proof}


\section{Доказательство условия~\ref{cpa_transfer_composite_req} для LockCPA}
\label{sect_transfer_lock_proof}

Нам нужно доказать, что определенные в разделе~\ref{sect_lock_analysis} операторы $transfer$ и $apply$ удовлетворяют условию~\ref{cpa_transfer_composite_req}.


\begin{proof}
Рассмотрим случайное множество абстрактных переходов $e_0, e_1, \dots, e_n$ и конкретный переход $
\tau \in \epp_S
\left(
\begin{pmatrix}
e_0 \\
t_0 
\end{pmatrix},
\left\{
\begin{pmatrix}
e_1 \\
t_1 
\end{pmatrix},
\dots ,
\begin{pmatrix}
e_n \\
t_n 
\end{pmatrix}
\right\}
\right)$

Возьмем некоторый следующий переход $\tau \tcarrow \tau'$, где $\tau = (c, g, t_0)$
Так как предикатный анализ выдает все возможные следующие дуги, достаточно показать, что 
$\exists s_0', s_1', \dots, s_n' \in E^S_P$ и  $с' \in \epp^S_S
\left(
\begin{pmatrix}
s_k' \\
t_k 
\end{pmatrix},
\left\{
\begin{pmatrix}
s_0' \\
t_0 
\end{pmatrix},
\dots ,
\begin{pmatrix}
s_n' \\
t_n 
\end{pmatrix}
\right\}
\right)$.

Не будем подробно рассматривать случаи $op = thread\_create$, $op = assign$, $op = assume$
%, $g = thread\_join$
, так как эти переходы не меняет состояния анализа примитивов синхронизации и не меняют состояние $c_s$.
А значит, $s_i = s_i'$.
Поэтому в этом случае $c' \in \epp^S_S
\left(
\begin{pmatrix}
s_k' \\
t_k 
\end{pmatrix},
\left\{
\begin{pmatrix}
s_0' \\
t_0 
\end{pmatrix},
\dots ,
\begin{pmatrix}
s_n' \\
t_n 
\end{pmatrix}
\right\}
\right) = 
\epp^S_P
\left(
\begin{pmatrix}
s_k \\
t_k 
\end{pmatrix},
\left\{
\begin{pmatrix}
s_0 \\
t_0 
\end{pmatrix},
\dots ,
\begin{pmatrix}
s_n \\
t_n 
\end{pmatrix}
\right\}
\right)$.

Рассмотрим теперь случай $op = acquire(\hat s)$. Зафиксируем значение $k$ такое, что $t_k = t'$. Будем рассматривать переход в потоке из состояния $s_k$.

$c \tc{t, acquire(\hat s)} c'$, $s_k \tat{acquire(\hat s)} s_k'$, нужно показать, что 
$c' \in \epp^S_P
\left(
\begin{pmatrix}
s_k' \\
t_k 
\end{pmatrix},
\left\{
\begin{pmatrix}
s_0' \\
t_0 
\end{pmatrix},
\dots ,
\begin{pmatrix}
s_n' \\
t_n 
\end{pmatrix}
\right\}
\right)$

По определению отношения переходов $c_s'(\hat s) = t_k$, $s_k' = s_k \cup \{\hat s\}$.
Заметим, что если переход есть на конкретных состояниях, это означает, что блокировка $\hat s$ не захвачена ни одним из потоков $t_1, \dots, t_n$.
А это, в свою очередь, означает, что $s \notin s_i, 1 \le i \le n, i \neq k$.
Таким образом, новое состояние $s_k'$ остается совместным с остальными частичными состояниями $s_i' = s_i, i \neq k$.
По определению $\epp_S$
\begin{align*}
\epp^S_P
\left(
\begin{pmatrix}
s_k' \\
t_k 
\end{pmatrix},
\left\{
\begin{pmatrix}
s_0' \\
t_0 
\end{pmatrix},
\dots ,
\begin{pmatrix}
s_n' \\
t_n 
\end{pmatrix}
\right\}
\right) = \{ c \in C \mid \hat s \in s_i \implies c_s(\hat s) = t_i \} = C_0
\end{align*}

Очевидно, $c' \in C_0$.
Аналогично можно проверить, что операция $g = release(\hat s)$ удовлетворяет условию~\ref{cpa_transfer_composite_req}. 

\end{proof}


\section{Доказательство условия~\ref{cpa_transfer_composite_req} для ValueCPA}
\label{sect_transfer_value_proof}

Нам нужно доказать, что определенные в разделе~\ref{sect_value_analysis} операторы $transfer$ и $apply$ удовлетворяют условию~\ref{cpa_transfer_composite_req}.


\begin{proof}
Рассмотрим случайное множество абстрактных переходов $e_0, e_1, \dots, e_n$ и конкретный переход $
\tau \in \epp_V
\left(
\begin{pmatrix}
e_0 \\
t_0 
\end{pmatrix},
\left\{
\begin{pmatrix}
e_1 \\
t_1 
\end{pmatrix},
\dots ,
\begin{pmatrix}
e_n \\
t_n 
\end{pmatrix}
\right\}
\right)$

Возьмем некоторый следующий переход $\tau \tcarrow \tau'$, где $\tau = (c, g, t_0)$
Так как анализ явных значений выдает все возможные следующие дуги, достаточно показать, что 
$\exists s_0', s_1', \dots, s_n' \in E^S_V$ и  $с' \in \epp^S_V
\left(
\begin{pmatrix}
s_k' \\
t_k 
\end{pmatrix},
\left\{
\begin{pmatrix}
s_0' \\
t_0 
\end{pmatrix},
\dots ,
\begin{pmatrix}
s_n' \\
t_n 
\end{pmatrix}
\right\}
\right)$.

Не будем подробно рассматривать случаи $op = thread\_create$, $op = acquire$, $op = release$
%, $g = thread\_join$
, так как эти переходы не меняет состояния анализа явных значений и не меняют состояние памяти $c_g$ и $c_l$.
А значит, $s_i = s_i'$.
Поэтому в этом случае $c' \in \epp^S_V
\left(
\begin{pmatrix}
s_k' \\
t_k 
\end{pmatrix},
\left\{
\begin{pmatrix}
s_0' \\
t_0 
\end{pmatrix},
\dots ,
\begin{pmatrix}
s_n' \\
t_n 
\end{pmatrix}
\right\}
\right) = 
\epp^S_V
\left(
\begin{pmatrix}
s_k \\
t_k 
\end{pmatrix},
\left\{
\begin{pmatrix}
s_0 \\
t_0 
\end{pmatrix},
\dots ,
\begin{pmatrix}
s_n \\
t_n 
\end{pmatrix}
\right\}
\right)$.

Рассмотрим теперь случай $op = assign(x, e)$. Зафиксируем значение $k$ такое, что $t_k = t'$. Будем рассматривать переход в потоке из состояния $s_k$.

$c \tc{t, assign(x, e)} c'$, $s_k \tat{assign(x, e)} s_k'$, нужно показать, что 
$c' \in \epp^S_V
\left(
\begin{pmatrix}
s_k' \\
t_k 
\end{pmatrix},
\left\{
\begin{pmatrix}
s_0' \\
t_0 
\end{pmatrix},
\dots ,
\begin{pmatrix}
s_n' \\
t_n 
\end{pmatrix}
\right\}
\right)$

Начальные состояния $s_i$ были совместными, так как к ним был применим оператор $\epp$. 
Отсюда следует, что существовала общая модель $\hat{v_g} \models s_i$.
Заметим, что если переход есть на конкретных состояниях, это означает, что $c_g(x) = e_{/c}$, если $x \in X^{global}$ или $c_l(t_k)(x) = e_{/c}$, если $x \in X^{local}$.
Рассмотрим присваивание, меняющее значение глобальной переменной, как наиболее сложный.
По определению отношения переходов $s_k'(x) = e_{/s}$, а значения остальных переменных не изменяется.
По определению перехода в окружении $s_i'(x) = q(x) = e$, а значения остальных переменных не изменяются.
Таким образом, частичные состояния всех потоков меняются согласованно, и новое отображение $\hat{v_g}'$ должно являться моделью для новых абстрактных состояний, так как и состояния, и модель изменили только одно значение переменной $x$.
Теперь необходимо проверить, что для новых переходов будет выполнен оператор $apply$, то есть, переходы будут совместны.
Мы показали, что состояния останутся совместными, и осталось показать, что совместными останутся дуги, то есть, согласно~\ref{epp_split}, что $\forall g \in G: g \in \econc{g} \cap \econc{x \to e}$. 
Очевидно, что $(\cdot, assign(x, e), \cdot) \in \econc{x \to e}$, и условие выполнено.
Таким образом мы показали, что условие~\ref{cpa_transfer_inner_req} выполнено для анализа предикатов.

\end{proof}


\section{Доказательство условия~\ref{cpa_transfer_composite_req} для ThreadCPA c эффектами окружения}
\label{sect_transfer_thread_env_proof}

Нам нужно доказать, что определенные в разделе~\ref{sect_thread_analysis_env} операторы $transfer$ и $apply$ удовлетворяют условию~\ref{cpa_transfer_composite_req}.

\begin{proof}
Рассмотрим случайный конкретный переход $\tau = (c, g, t_0)$, $g=(\cdot, op, \cdot)$:
$$\tau \in  \epp_{T}
\left(
\begin{pmatrix}
e_0 \\
t_0 
\end{pmatrix},
\left\{
\begin{pmatrix}
e_1 \\
t_1 
\end{pmatrix},
\dots ,
\begin{pmatrix}
e_m \\
t_m 
\end{pmatrix}
\right\}
\right)$$.

Если $op \neq thread\_create$, по определению $transfer$ $e'_i = (s_i, \top^T_T)$, то есть состояние не изменяется, а переходы рассматриваются все возможные. 
Таким образом, 
$$с' \in  \epp^S_{T}
\left(
\begin{pmatrix}
s_0 \\
t_0 
\end{pmatrix},
\left\{
\begin{pmatrix}
s_1 \\
t_1 
\end{pmatrix},
\dots ,
\begin{pmatrix}
s_m \\
t_m 
\end{pmatrix}
\right\}
\right) = \epp^S_{T}
\left(
\begin{pmatrix}
s'_0 \\
t_0 
\end{pmatrix},
\left\{
\begin{pmatrix}
s'_1 \\
t_1 
\end{pmatrix},
\dots ,
\begin{pmatrix}
s'_m \\
t_m 
\end{pmatrix}
\right\}
\right)$$
А значит, $\tau' = (c', g', t_k)$:
$$\tau'  \in \epp_{T}
\left(
\begin{pmatrix}
e'_k \\
t_k 
\end{pmatrix},
\left\{
\begin{pmatrix}
e'_0 \\
t_0 
\end{pmatrix},
\dots ,
\begin{pmatrix}
e'_m \\
t_m 
\end{pmatrix}
\right\}
\right)$$

Рассмотрим теперь случай $op = thread\_create(l_{\nu})$.
При обычном переходе меняются состояния нулевого потока: $C'_0 = C_0 \cup \{l_{\nu}\}$, и появляется новый переход, соответствующий созданному потоку: $e_{m+1} = ((l_{\nu},C_0 \cup \{l_{\nu}\}) , \top^T_T)$.

По определению $\epp_T$ все остальные переходы в окружении должны были содержать в себе абстрактную дугу, которая соответсвует переходу в потоке, то есть $q_i = (create, l_{\nu})$.
Таким образом, $\forall 1 \le i \le m, C_i = C'_0$, то есть состояния остаются совместными, а значит, оператор $apply$ будет применим.
Поэтому, 
$$\tau' \in  \epp_{T}
\left(
\begin{pmatrix}
e'_k \\
t_k 
\end{pmatrix},
\left\{
\begin{pmatrix}
e'_0 \\
t_0 
\end{pmatrix},
\dots ,
\begin{pmatrix}
e'_m \\
t_m 
\end{pmatrix} ,
\begin{pmatrix}
e'_{m+1} \\
l_{\nu} 
\end{pmatrix}
\right\}
\right)$$

\end{proof}


\section{Доказательство условия~\ref{cpa_transfer_composite_req} для расширенного ThreadCPA, инвариантного к эффектам окружения}
\label{sect_transfer_thread_ext_proof}

Нам нужно доказать, что определенные в разделе~\ref{sect_thread_analysis_ext} операторы $transfer$ и $apply$ удовлетворяют условию~\ref{cpa_transfer_composite_req}.

\begin{proof}
Рассмотрим случайный конкретный переход $\tau = (c, g, t_0)$, $g=(\cdot, op, \cdot)$:
$$\tau \in  \epp_{T}
\left(
\begin{pmatrix}
e_0 \\
t_0 
\end{pmatrix},
\left\{
\begin{pmatrix}
e_1 \\
t_1 
\end{pmatrix},
\dots ,
\begin{pmatrix}
e_m \\
t_m 
\end{pmatrix}
\right\}
\right)$$.

Если $op \neq thread\_create$, по определению $transfer$ $e'_i = (s_i, \top^T_T)$, то есть состояние не изменяется, а переходы рассматриваются все возможные. 
Таким образом, 
$$с' \in  \epp^S_{T}
\left(
\begin{pmatrix}
s_0 \\
t_0 
\end{pmatrix},
\left\{
\begin{pmatrix}
s_1 \\
t_1 
\end{pmatrix},
\dots ,
\begin{pmatrix}
s_m \\
t_m 
\end{pmatrix}
\right\}
\right) = \epp^S_{T}
\left(
\begin{pmatrix}
s'_0 \\
t_0 
\end{pmatrix},
\left\{
\begin{pmatrix}
s'_1 \\
t_1 
\end{pmatrix},
\dots ,
\begin{pmatrix}
s'_m \\
t_m 
\end{pmatrix}
\right\}
\right)$$
А значит, $\tau' = (c', g', t_k)$:
$$\tau'  \in \epp_{T}
\left(
\begin{pmatrix}
e'_k \\
t_k 
\end{pmatrix},
\left\{
\begin{pmatrix}
e'_0 \\
t_0 
\end{pmatrix},
\dots ,
\begin{pmatrix}
e'_m \\
t_m 
\end{pmatrix}
\right\}
\right)$$

Рассмотрим теперь случай $op = thread\_create(l_{\nu})$.
При обычном переходе меняются состояния нулевого потока: $C'_0 = C_0 \cup \{l_{\nu}\}$, и появляется новый переход, соответствующий созданному потоку: $e_{m+1} = ((l_{\nu},C_0 \cup \{l_{\nu}\}) , \top^T_T)$.

По определению $\epp_T$ все остальные переходы в окружении должны были содержать в себе абстрактную дугу, которая соответсвует переходу в потоке, то есть $q_i = (create, l_{\nu})$.
Таким образом, $\forall 1 \le i \le m, C_i = C'_0$, то есть состояния остаются совместными, а значит, оператор $apply$ будет применим.
Поэтому, 
$$\tau' \in  \epp_{T}
\left(
\begin{pmatrix}
e'_k \\
t_k 
\end{pmatrix},
\left\{
\begin{pmatrix}
e'_0 \\
t_0 
\end{pmatrix},
\dots ,
\begin{pmatrix}
e'_m \\
t_m 
\end{pmatrix} ,
\begin{pmatrix}
e'_{m+1} \\
l_{\nu} 
\end{pmatrix}
\right\}
\right)$$

\end{proof}


%_____________________

%Для крупных листингов есть два способа. Первый красивый, но в нём могут быть проблемы с поддержкой кириллицы (у вас может встречаться в комментариях и
%печатаемых сообщениях), он представлен на листинге~\ref{list:hwbeauty}.
%\begin{ListingEnv}[!h]% настройки floating аналогичны окружению figure
%    \captiondelim{ } % разделитель идентификатора с номером от наименования
%    \caption{Программа ,,Hello, world`` на \protect\cpp}
%    % далее метка для ссылки:
%    \label{list:hwbeauty}
%    % окружение учитывает пробелы и табуляции и применяет их в сответсвии с настройками
%    \begin{lstlisting}[language={[ISO]C++}]
%	#include <iostream>
%	using namespace std;
%
%	int main() //кириллица в комментариях при xelatex и lualatex имеет проблемы с пробелами
%	{
%		cout << "Hello, world" << endl; //latin letters in commentaries
%		system("pause");
%		return 0;
%	}
%    \end{lstlisting}
%\end{ListingEnv}%
%Второй не~такой красивый, но без ограничений (см.~листинг~\ref{list:hwplain}).
%\begin{ListingEnv}[!h]
%    \captiondelim{ } % разделитель идентификатора с номером от наименования
%    \caption{Программа ,,Hello, world`` без подсветки}
%    \label{list:hwplain}
%    \begin{Verb}
%        
%        #include <iostream>
%        using namespace std;
%        
%        int main() //кириллица в комментариях
%        {
%            cout << "Привет, мир" << endl;
%        }
%    \end{Verb}
%\end{ListingEnv}
%
%Можно использовать первый для вставки небольших фрагментов
%внутри текста, а второй для вставки полного
%кода в приложении, если таковое имеется.
%
%Если нужно вставить совсем короткий пример кода (одна или две строки),
%то~выделение  линейками и нумерация может смотреться чересчур громоздко.
%В таких случаях можно использовать окружения \texttt{lstlisting} или
%\texttt{Verb} без \texttt{ListingEnv}. Приведём такой пример
%с указанием языка программирования, отличного от~заданного по умолчанию:
%\begin{lstlisting}[language=Haskell]
%fibs = 0 : 1 : zipWith (+) fibs (tail fibs)
%\end{lstlisting}
%Такое решение~--- со вставкой нумерованных листингов покрупнее
%и вставок без выделения для маленьких фрагментов~--- выбрано,
%например, в книге Эндрю Таненбаума и Тодда Остина по архитектуре
%%компьютера~\autocite{TanAus2013} (см.~рис.~\ref{fig:tan-aus}).
%
%Наконец, для оформления идентификаторов внутри строк
%(функция \lstinline{main} и~тому подобное) используется
%\texttt{lstinline} или, самое простое, моноширинный текст
%(\texttt{\textbackslash texttt}).
%
%
%Пример~\ref{list:internal3}, иллюстрирующий подключение переопределённого языка. Может быть полезным, если подсветка кода работает криво. Без дополнительного окружения, с подписью и ссылкой, реализованной встроенным средством.
%\begingroup
%\captiondelim{ } % разделитель идентификатора с номером от наименования
%\begin{lstlisting}[language={Renhanced},caption={Пример листинга c подписью собственными средствами},label={list:internal3}]
%## Caching the Inverse of a Matrix
%
%## Matrix inversion is usually a costly computation and there may be some
%## benefit to caching the inverse of a matrix rather than compute it repeatedly
%## This is a pair of functions that cache the inverse of a matrix.
%
%## makeCacheMatrix creates a special "matrix" object that can cache its inverse
%
%makeCacheMatrix <- function(x = matrix()) {#кириллица в комментариях при xelatex и lualatex имеет проблемы с пробелами
%    i <- NULL
%    set <- function(y) {
%        x <<- y
%        i <<- NULL
%    }
%    get <- function() x
%    setSolved <- function(solve) i <<- solve
%    getSolved <- function() i
%    list(set = set, get = get,
%    setSolved = setSolved,
%    getSolved = getSolved)
%    
%}
%
%
%## cacheSolve computes the inverse of the special "matrix" returned by
%## makeCacheMatrix above. If the inverse has already been calculated (and the
%## matrix has not changed), then the cachesolve should retrieve the inverse from
%## the cache.
%
%cacheSolve <- function(x, ...) {
%    ## Return a matrix that is the inverse of 'x'
%    i <- x$getSolved()
%    if(!is.null(i)) {
%        message("getting cached data")
%        return(i)
%    }
%    data <- x$get()
%    i <- solve(data, ...)
%    x$setSolved(i)
%    i  
%}
%\end{lstlisting} %$ %Комментарий для корректной подсветки синтаксиса
%                 %вне листинга
%\endgroup
%
%Листинг~\ref{list:external1} подгружается из внешнего файла. Приходится загружать без окружения дополнительного. Иначе по страницам не переносится.
%\begingroup
%\captiondelim{ } % разделитель идентификатора с номером от наименования
%    \lstinputlisting[lastline=78,language={R},caption={Листинг из внешнего файла},label={list:external1}]{listings/run_analysis.R}
%\endgroup

%
%\chapter{Очень длинное название второго приложения, в~котором продемонстрирована работа с~длинными таблицами} \label{AppendixB}
%
% \section{Подраздел приложения}\label{AppendixB1}
%Вот размещается длинная таблица:
%\fontsize{10pt}{10pt}\selectfont
%\begin{longtable*}[c]{|l|c|l|l|} %longtable* появляется из пакета ltcaption и даёт ненумерованную таблицу
%% \caption{Описание входных файлов модели}\label{Namelists} 
%%\\ 
% \hline
% %\multicolumn{4}{|c|}{\textbf{Файл puma\_namelist}}        \\ \hline
% Параметр & Умолч. & Тип & Описание               \\ \hline
%                                              \endfirsthead   \hline
% \multicolumn{4}{|c|}{\small\slshape (продолжение)}        \\ \hline
% Параметр & Умолч. & Тип & Описание               \\ \hline
%                                              \endhead        \hline
%% \multicolumn{4}{|c|}{\small\slshape (окончание)}        \\ \hline
%% Параметр & Умолч. & Тип & Описание               \\ \hline
%%                                             \endlasthead        \hline
% \multicolumn{4}{|r|}{\small\slshape продолжение следует}  \\ \hline
%                                              \endfoot        \hline
%                                              \endlastfoot
% \multicolumn{4}{|l|}{\&INP}        \\ \hline 
% kick & 1 & int & 0: инициализация без шума ($p_s = const$) \\
%      &   &     & 1: генерация белого шума                  \\
%      &   &     & 2: генерация белого шума симметрично относительно \\
%  & & & экватора    \\
% mars & 0 & int & 1: инициализация модели для планеты Марс     \\
% kick & 1 & int & 0: инициализация без шума ($p_s = const$) \\
%      &   &     & 1: генерация белого шума                  \\
%      &   &     & 2: генерация белого шума симметрично относительно \\
%  & & & экватора    \\
% mars & 0 & int & 1: инициализация модели для планеты Марс     \\
%kick & 1 & int & 0: инициализация без шума ($p_s = const$) \\
%      &   &     & 1: генерация белого шума                  \\
%      &   &     & 2: генерация белого шума симметрично относительно \\
%  & & & экватора    \\
% mars & 0 & int & 1: инициализация модели для планеты Марс     \\
%kick & 1 & int & 0: инициализация без шума ($p_s = const$) \\
%      &   &     & 1: генерация белого шума                  \\
%      &   &     & 2: генерация белого шума симметрично относительно \\
%  & & & экватора    \\
% mars & 0 & int & 1: инициализация модели для планеты Марс     \\
%kick & 1 & int & 0: инициализация без шума ($p_s = const$) \\
%      &   &     & 1: генерация белого шума                  \\
%      &   &     & 2: генерация белого шума симметрично относительно \\
%  & & & экватора    \\
% mars & 0 & int & 1: инициализация модели для планеты Марс     \\
%kick & 1 & int & 0: инициализация без шума ($p_s = const$) \\
%      &   &     & 1: генерация белого шума                  \\
%      &   &     & 2: генерация белого шума симметрично относительно \\
%  & & & экватора    \\
% mars & 0 & int & 1: инициализация модели для планеты Марс     \\
%kick & 1 & int & 0: инициализация без шума ($p_s = const$) \\
%      &   &     & 1: генерация белого шума                  \\
%      &   &     & 2: генерация белого шума симметрично относительно \\
%  & & & экватора    \\
% mars & 0 & int & 1: инициализация модели для планеты Марс     \\
%kick & 1 & int & 0: инициализация без шума ($p_s = const$) \\
%      &   &     & 1: генерация белого шума                  \\
%      &   &     & 2: генерация белого шума симметрично относительно \\
%  & & & экватора    \\
% mars & 0 & int & 1: инициализация модели для планеты Марс     \\
%kick & 1 & int & 0: инициализация без шума ($p_s = const$) \\
%      &   &     & 1: генерация белого шума                  \\
%      &   &     & 2: генерация белого шума симметрично относительно \\
%  & & & экватора    \\
% mars & 0 & int & 1: инициализация модели для планеты Марс     \\
%kick & 1 & int & 0: инициализация без шума ($p_s = const$) \\
%      &   &     & 1: генерация белого шума                  \\
%      &   &     & 2: генерация белого шума симметрично относительно \\
%  & & & экватора    \\
% mars & 0 & int & 1: инициализация модели для планеты Марс     \\
%kick & 1 & int & 0: инициализация без шума ($p_s = const$) \\
%      &   &     & 1: генерация белого шума                  \\
%      &   &     & 2: генерация белого шума симметрично относительно \\
%  & & & экватора    \\
% mars & 0 & int & 1: инициализация модели для планеты Марс     \\
%kick & 1 & int & 0: инициализация без шума ($p_s = const$) \\
%      &   &     & 1: генерация белого шума                  \\
%      &   &     & 2: генерация белого шума симметрично относительно \\
%  & & & экватора    \\
% mars & 0 & int & 1: инициализация модели для планеты Марс     \\
%kick & 1 & int & 0: инициализация без шума ($p_s = const$) \\
%      &   &     & 1: генерация белого шума                  \\
%      &   &     & 2: генерация белого шума симметрично относительно \\
%  & & & экватора    \\
% mars & 0 & int & 1: инициализация модели для планеты Марс     \\
%kick & 1 & int & 0: инициализация без шума ($p_s = const$) \\
%      &   &     & 1: генерация белого шума                  \\
%      &   &     & 2: генерация белого шума симметрично относительно \\
%  & & & экватора    \\
% mars & 0 & int & 1: инициализация модели для планеты Марс     \\
%kick & 1 & int & 0: инициализация без шума ($p_s = const$) \\
%      &   &     & 1: генерация белого шума                  \\
%      &   &     & 2: генерация белого шума симметрично относительно \\
%  & & & экватора    \\
% mars & 0 & int & 1: инициализация модели для планеты Марс     \\
% \hline
%  %& & & $\:$ \\ 
% \multicolumn{4}{|l|}{\&SURFPAR}        \\ \hline
%kick & 1 & int & 0: инициализация без шума ($p_s = const$) \\
%      &   &     & 1: генерация белого шума                  \\
%      &   &     & 2: генерация белого шума симметрично относительно \\
%  & & & экватора    \\
% mars & 0 & int & 1: инициализация модели для планеты Марс     \\
%kick & 1 & int & 0: инициализация без шума ($p_s = const$) \\
%      &   &     & 1: генерация белого шума                  \\
%      &   &     & 2: генерация белого шума симметрично относительно \\
%  & & & экватора    \\
% mars & 0 & int & 1: инициализация модели для планеты Марс     \\
%kick & 1 & int & 0: инициализация без шума ($p_s = const$) \\
%      &   &     & 1: генерация белого шума                  \\
%      &   &     & 2: генерация белого шума симметрично относительно \\
%  & & & экватора    \\
% mars & 0 & int & 1: инициализация модели для планеты Марс     \\
%kick & 1 & int & 0: инициализация без шума ($p_s = const$) \\
%      &   &     & 1: генерация белого шума                  \\
%      &   &     & 2: генерация белого шума симметрично относительно \\
%  & & & экватора    \\
% mars & 0 & int & 1: инициализация модели для планеты Марс     \\
%kick & 1 & int & 0: инициализация без шума ($p_s = const$) \\
%      &   &     & 1: генерация белого шума                  \\
%      &   &     & 2: генерация белого шума симметрично относительно \\
%  & & & экватора    \\
% mars & 0 & int & 1: инициализация модели для планеты Марс     \\
%kick & 1 & int & 0: инициализация без шума ($p_s = const$) \\
%      &   &     & 1: генерация белого шума                  \\
%      &   &     & 2: генерация белого шума симметрично относительно \\
%  & & & экватора    \\
% mars & 0 & int & 1: инициализация модели для планеты Марс     \\
%kick & 1 & int & 0: инициализация без шума ($p_s = const$) \\
%      &   &     & 1: генерация белого шума                  \\
%      &   &     & 2: генерация белого шума симметрично относительно \\
%  & & & экватора    \\
% mars & 0 & int & 1: инициализация модели для планеты Марс     \\
%kick & 1 & int & 0: инициализация без шума ($p_s = const$) \\
%      &   &     & 1: генерация белого шума                  \\
%      &   &     & 2: генерация белого шума симметрично относительно \\
%  & & & экватора    \\
% mars & 0 & int & 1: инициализация модели для планеты Марс     \\
%kick & 1 & int & 0: инициализация без шума ($p_s = const$) \\
%      &   &     & 1: генерация белого шума                  \\
%      &   &     & 2: генерация белого шума симметрично относительно \\
%  & & & экватора    \\
% mars & 0 & int & 1: инициализация модели для планеты Марс     \\ 
% \hline 
%\end{longtable*}
%
%\normalsize% возвращаем шрифт к нормальному
%
%\section{Использование длинных таблиц с окружением \textit{longtabu}} \label{AppendixB2a}
%
%В таблице~\ref{tbl:test-functions} более книжный вариант 
%длинной таблицы, используя окружение \verb!longtabu! и разнообразные
%\verb!toprule! \verb!midrule! \verb!bottomrule! из пакета
%\verb!booktabs!. Чтобы визуально таблица смотрелась лучше, можно
%использовать следующие параметры: в самом начале задаётся расстояние
%между строчками с~помощью \verb!arraystretch!. Таблица задаётся на
%всю ширину, \verb!longtabu! позволяет делить ширину колонок
%пропорционально "--- тут три колонки в пропорции 1.1:1:4 "--- для каждой
%колонки первый параметр в описании \verb!X[]!. Кроме того, в~таблице
%убраны отступы слева и справа с помощью \verb!@{}! в
%преамбуле таблицы. К первому и~второму столбцу применяется
%модификатор 
%
%\verb!>{\setlength{\baselineskip}{0.7\baselineskip}}!,
%
%\noindent который уменьшает межстрочный интервал в для текста таблиц (иначе
%заголовок второго столбца значительно шире, а двухстрочное имя
%сливается с~окружающими). Для первой и второй колонки текст в ячейках
%выравниваются по~центру как по вертикали, так и по горизонтали "---
%задаётся буквами \verb!m!~и~\verb!c!~в~описании столбца \verb!X[]!. 
%
%Так как формулы большие "--- используется окружение \verb!alignedat!,
%чтобы отступ был одинаковый у всех формул "--- он сделан для всех, хотя
%для большей части можно было и не использовать.  Чтобы формулы
%занимали поменьше места в~каждом столбце формулы (где надо)
%используется \verb!\textstyle! "--- он~делает дроби меньше, у знаков
%суммы и произведения "--- индексы сбоку. Иногда формулы слишком большая,
%сливается со следующей, поэтому после неё ставится небольшой
%дополнительный отступ \verb!\vspace*{2ex}!  Для штрафных функций "---
%размер фигурных скобок задан вручную \verb!\Big\{!, т.к. не умеет
%\verb!alignedat! работать с~\verb!\left! и \verb!\right! через
%несколько строк/колонок.
%
%
%В примечании к таблице наоборот, окружение \verb!cases! даёт слишком
%большие промежутки между вариантами, чтобы их уменьшить, в конце
%каждой строчки окружения использовался отрицательный дополнительный
%отступ \verb!\\[-0.5em]!.
%
%
%
%\begingroup % Ограничиваем область видимости arraystretch
%\renewcommand{\arraystretch}{1.6}%% Увеличение расстояния между рядами, для улучшения восприятия.
%\begin{longtabu} to \textwidth
%{%
%@{}>{\setlength{\baselineskip}{0.7\baselineskip}}X[1.1mc]%
%>{\setlength{\baselineskip}{0.7\baselineskip}}X[mc]%
%X[4]@{}%
%}
%        \caption{Тестовые функции для оптимизации, $D$ "---
%          размерность. Для всех функций значение в точке глобального
%          минимума равно нулю.\label{tbl:test-functions}}\\% label всегда желательно идти после caption 
%        
%        \toprule     %%% верхняя линейка
%        Имя           &Стартовый диапазон параметров &Функция  \\ 
%        \midrule %%% тонкий разделитель. Отделяет названия столбцов. Обязателен по ГОСТ 2.105 пункт 4.4.5 
%        \endfirsthead
%
%        \multicolumn{3}{c}{\small\slshape (продолжение)}        \\ 
%        \toprule     %%% верхняя линейка
%        Имя           &Стартовый диапазон параметров &Функция  \\ 
%        \midrule %%% тонкий разделитель. Отделяет названия столбцов. Обязателен по ГОСТ 2.105 пункт 4.4.5 
%        \endhead
%        
%        \multicolumn{3}{c}{\small\slshape (окончание)}        \\ 
%        \toprule     %%% верхняя линейка
%        Имя           &Стартовый диапазон параметров &Функция  \\ 
%        \midrule %%% тонкий разделитель. Отделяет названия столбцов. Обязателен по ГОСТ 2.105 пункт 4.4.5 
%        \endlasthead
%
%        \bottomrule %%% нижняя линейка
%        \multicolumn{3}{r}{\small\slshape продолжение следует}  \\ 
%        \endfoot   
%        \endlastfoot
%
%        сфера         &$\left[-100,\,100\right]^D$   &
%        $\begin{aligned}\textstyle f_1(x)=\sum_{i=1}^Dx_i^2\end{aligned}$                                                        \\
%        Schwefel 2.22 &$\left[-10,\,10\right]^D$     &
%        $\begin{aligned}\textstyle f_2(x)=\sum_{i=1}^D|x_i|+\prod_{i=1}^D|x_i|\end{aligned}$                                     \\
%        Schwefel 1.2  &$\left[-100,\,100\right]^D$   &$\begin{aligned}\textstyle f_3(x)=\sum_{i=1}^D\left(\sum_{j=1}^ix_j\right)^2\end{aligned}$                               \\
%        Schwefel 2.21 &$\left[-100,\,100\right]^D$   &$\begin{aligned}\textstyle f_4(x)=\max_i\!\left\{\left|x_i\right|\right\}\end{aligned}$                             \\
%        Rosenbrock    &$\left[-30,\,30\right]^D$     &$\begin{aligned}\textstyle f_5(x)=\sum_{i=1}^{D-1}\left[100\!\left(x_{i+1}-x_i^2\right)^2+(x_i-1)^2\right]\end{aligned}$ \\
%        ступенчатая   &$\left[-100,\,100\right]^D$   &$\begin{aligned}\textstyle f_6(x)=\sum_{i=1}^D\big\lfloor x_i+0.5\big\rfloor^2\end{aligned}$                             \\ 
%зашумлённая квартическая  &$\left[-1.28,\,1.28\right]^D$ &$\begin{aligned}\textstyle f_7(x)=\sum_{i=1}^Dix_i^4+rand[0,1)\end{aligned}$\vspace*{2ex}\\
%        Schwefel 2.26 &$\left[-500,\,500\right]^D$   &$\begin{aligned}f_8(x)= &\textstyle\sum_{i=1}^D-x_i\,\sin\sqrt{|x_i|}\,+ \\
%                    &\vphantom{\sum}+ D\cdot
%                    418.98288727243369 \end{aligned}$\\
%        Rastrigin     &$\left[-5.12,\,5.12\right]^D$ &
%        $\begin{aligned}\textstyle
%          f_9(x)=\sum_{i=1}^D\left[x_i^2-10\,\cos(2\pi
%            x_i)+10\right]\end{aligned}$\vspace*{2ex}\\
%  Ackley        &$\left[-32,\,32\right]^D$     &$\begin{aligned}f_{10}(x)= &\textstyle -20\, \exp\!\left(-0.2\sqrt{\frac{1}{D}\sum_{i=1}^Dx_i^2} \right)-\\
%                    &\textstyle - \exp\left(\frac{1}{D}\sum_{i=1}^D\cos(2\pi x_i)  \right)  + 20 + e \end{aligned}$ \\
%        Griewank      &$\left[-600,\,600\right]^D$
%        &$\begin{aligned}f_{11}(x)= &\textstyle \frac{1}{4000}
%          \sum_{i=1}^{D}x_i^2 - \prod_{i=1}^D\cos\left(x_i/\sqrt{i}\right) +1     \end{aligned}$ \vspace*{3ex} \\
%        штрафная 1    &$\left[-50,\,50\right]^D$     &
%        $\begin{aligned}f_{12}(x)= &\textstyle \frac{\pi}{D}
%          \Big\{ 10\,\sin^2(\pi y_1) +\\ &+
%          \textstyle \sum_{i=1}^{D-1}(y_i-1)^2\left[1+10\,\sin^2(\pi
%              y_{i+1})\right] +\\ &+(y_D-1)^2 \Big\} +\textstyle\sum_{i=1}^D u(x_i,\,10,\,100,\,4)            \end{aligned}$ \vspace*{2ex} \\
%        штрафная 2    &$\left[-50,\,50\right]^D$     &
%        $\begin{aligned}f_{13}(x)= &\textstyle 0.1
%          \Big\{\sin^2(3\pi x_1) +\\ &+
%          \textstyle \sum_{i=1}^{D-1}(x_i-1)^2\left[1+\sin^2(3 \pi
%              x_{i+1})\right] + \\ &+(x_D-1)^2\left[1+\sin^2(2\pi
%              x_D)\right] \Big\} +\\ &+\textstyle\sum_{i=1}^D u(x_i,\,5,\,100,\,4)            \end{aligned}$               \\
%        сфера         &$\left[-100,\,100\right]^D$   &
%        $\begin{aligned}\textstyle f_1(x)=\sum_{i=1}^Dx_i^2\end{aligned}$                                                        \\
%        Schwefel 2.22 &$\left[-10,\,10\right]^D$     &
%        $\begin{aligned}\textstyle f_2(x)=\sum_{i=1}^D|x_i|+\prod_{i=1}^D|x_i|\end{aligned}$                                     \\
%        Schwefel 1.2  &$\left[-100,\,100\right]^D$   &$\begin{aligned}\textstyle f_3(x)=\sum_{i=1}^D\left(\sum_{j=1}^ix_j\right)^2\end{aligned}$                               \\
%        Schwefel 2.21 &$\left[-100,\,100\right]^D$   &$\begin{aligned}\textstyle f_4(x)=\max_i\!\left\{\left|x_i\right|\right\}\end{aligned}$                             \\
%        Rosenbrock    &$\left[-30,\,30\right]^D$     &$\begin{aligned}\textstyle f_5(x)=\sum_{i=1}^{D-1}\left[100\!\left(x_{i+1}-x_i^2\right)^2+(x_i-1)^2\right]\end{aligned}$ \\
%        ступенчатая   &$\left[-100,\,100\right]^D$   &$\begin{aligned}\textstyle f_6(x)=\sum_{i=1}^D\big\lfloor x_i+0.5\big\rfloor^2\end{aligned}$                             \\ 
%зашумлённая квартическая  &$\left[-1.28,\,1.28\right]^D$ &$\begin{aligned}\textstyle f_7(x)=\sum_{i=1}^Dix_i^4+rand[0,1)\end{aligned}$\vspace*{2ex}\\
%        Schwefel 2.26 &$\left[-500,\,500\right]^D$   &$\begin{aligned}f_8(x)= &\textstyle\sum_{i=1}^D-x_i\,\sin\sqrt{|x_i|}\,+ \\
%                    &\vphantom{\sum}+ D\cdot
%                    418.98288727243369 \end{aligned}$\\
%        Rastrigin     &$\left[-5.12,\,5.12\right]^D$ &
%        $\begin{aligned}\textstyle
%          f_9(x)=\sum_{i=1}^D\left[x_i^2-10\,\cos(2\pi
%            x_i)+10\right]\end{aligned}$\vspace*{2ex}\\
%  Ackley        &$\left[-32,\,32\right]^D$     &$\begin{aligned}f_{10}(x)= &\textstyle -20\, \exp\!\left(-0.2\sqrt{\frac{1}{D}\sum_{i=1}^Dx_i^2} \right)-\\
%                    &\textstyle - \exp\left(\frac{1}{D}\sum_{i=1}^D\cos(2\pi x_i)  \right)  + 20 + e \end{aligned}$ \\
%        Griewank      &$\left[-600,\,600\right]^D$
%        &$\begin{aligned}f_{11}(x)= &\textstyle \frac{1}{4000}
%          \sum_{i=1}^{D}x_i^2 - \prod_{i=1}^D\cos\left(x_i/\sqrt{i}\right) +1     \end{aligned}$ \vspace*{3ex} \\
%        штрафная 1    &$\left[-50,\,50\right]^D$     &
%        $\begin{aligned}f_{12}(x)= &\textstyle \frac{\pi}{D}
%          \Big\{ 10\,\sin^2(\pi y_1) +\\ &+
%          \textstyle \sum_{i=1}^{D-1}(y_i-1)^2\left[1+10\,\sin^2(\pi
%              y_{i+1})\right] +\\ &+(y_D-1)^2 \Big\} +\textstyle\sum_{i=1}^D u(x_i,\,10,\,100,\,4)            \end{aligned}$ \vspace*{2ex} \\
%        штрафная 2    &$\left[-50,\,50\right]^D$     &
%        $\begin{aligned}f_{13}(x)= &\textstyle 0.1
%          \Big\{\sin^2(3\pi x_1) +\\ &+
%          \textstyle \sum_{i=1}^{D-1}(x_i-1)^2\left[1+\sin^2(3 \pi
%              x_{i+1})\right] + \\ &+(x_D-1)^2\left[1+\sin^2(2\pi
%              x_D)\right] \Big\} +\\ &+\textstyle\sum_{i=1}^D u(x_i,\,5,\,100,\,4)            \end{aligned}$               \\
%        \midrule%%% тонкий разделитель
%        \multicolumn{3}{@{}p{\textwidth}}{%
%            \vspace*{-3.5ex}% этим подтягиваем повыше
%            \hspace*{2.5em}% абзацный отступ - требование ГОСТ 2.105
%            Примечание "---  Для функций $f_{12}$ и $f_{13}$
%            используется $y_i = 1 + \frac{1}{4}(x_i+1)$
%            и~$u(x_i,\,a,\,k,\,m)=\begin{cases}
%k(x_i-a)^m,\quad &x_i >a\\[-0.5em]
%0,\quad &-a\leq x_i \leq a\\[-0.5em]
%k(-x_i-a)^m,\quad &x_i <-a
%\end{cases}$  }   \\        \bottomrule %%% нижняя линейка 
%\end{longtabu} 
%\endgroup
%
%
%\section{Форматирование внутри таблиц} \label{AppendixB3}
%
%В таблице~\ref{tbl:other-row} пример с чересстрочным
%форматированием. В~файле \verb+userstyles.tex+  задаётся счётчик
%\verb+\newcounter{rowcnt}+ который увеличивается на 1 после каждой
%строчки (как указано в преамбуле таблицы). Кроме того, задаётся
%условный макрос \verb+\altshape+ который выдаёт одно
%из~двух типов форматирования в~зависимости от чётности счётчика.
%
%В таблице~\ref{tbl:other-row} каждая чётная строчка "--- синяя,
%нечётная "--- с наклоном и~слегка поднята вверх. Визуально это приводит
%к тому, что среднее значение и~среднеквадратичное изменение
%группируются и хорошо выделяются взглядом в~таблице. Сохраняется
%возможность отдельные значения в таблице выделить цветом или
%шрифтом. К первому и второму столбцу форматирование не применяется
%по~сути таблицы, к шестому общее форматирование не применяетсся для
%наглядности.
%
%Так как заголовок таблицы тоже считается за строчку, то перед ним (для
%первого, промежуточного и финального варианта) счётчик обнуляется,
%а~в~\verb+\altshape+ для нулевого значения счётчика форматирования
%не~применяется.
%
%
%\begingroup % Ограничиваем область видимости arraystretch
%\renewcommand\altshape{
%  \ifnumequal{\value{rowcnt}}{0}{
%    % Стиль для заголовка таблицы
%  }{
%    \ifnumodd{\value{rowcnt}}
%    {
%      \color{blue} % Cтиль для нечётных строк
%    }{
%      \vspace*{-0.8ex}\itshape} % Стиль для чётных строк
%  }
%}
%\newcolumntype{A}{ >{\altshape}X[1mc]}
%\needspace{2\baselineskip}
%\renewcommand{\arraystretch}{0.9}%% Уменьшаем  расстояние между
%                                %% рядами, чтобы таблица не так много
%                                %% места занимала в дисере.
%\begin{longtabu} to \textwidth {@{}X[0.2ml]X[0.9mc]AAAX[0.99mc]>{\setlength{\baselineskip}{0.7\baselineskip}}AA<{\stepcounter{rowcnt}}@{}}
%% \begin{longtabu} to \textwidth {@{}X[0.2ml]X[1mc]X[1mc]X[1mc]X[1mc]X[1mc]>{\setlength{\baselineskip}{0.7\baselineskip}}X[1mc]X[1mc]@{}}
%  \caption{Длинная таблица с примером чересстрочного форматирования\label{tbl:other-row}}\vspace*{1ex}\\% label всегда желательно идти после caption
%  % \vspace*{1ex}     \\
%
%  \toprule %%% верхняя линейка  
%\setcounter{rowcnt}{0} &Итерации & JADE\texttt{++} & JADE & jDE & SaDE
%& DE/rand /1/bin & PSO \\ 
% \midrule %%% тонкий разделитель. Отделяет названия столбцов. Обязателен по ГОСТ 2.105 пункт 4.4.5 
% \endfirsthead
%
% \multicolumn{8}{c}{\small\slshape (продолжение)} \\ 
% \toprule %%% верхняя линейка
%\setcounter{rowcnt}{0} &Итерации & JADE\texttt{++} & JADE & jDE & SaDE
%& DE/rand /1/bin & PSO \\ 
% \midrule %%% тонкий разделитель. Отделяет названия столбцов. Обязателен по ГОСТ 2.105 пункт 4.4.5 
% \endhead
% 
% \multicolumn{8}{c}{\small\slshape (окончание)} \\ 
% \toprule %%% верхняя линейка
%\setcounter{rowcnt}{0} &Итерации & JADE\texttt{++} & JADE & jDE & SaDE
%& DE/rand /1/bin & PSO \\ 
% \midrule %%% тонкий разделитель. Отделяет названия столбцов. Обязателен по ГОСТ 2.105 пункт 4.4.5 
% \endlasthead
%
% \bottomrule %%% нижняя линейка
% \multicolumn{8}{r}{\small\slshape продолжение следует}     \\ 
% \endfoot 
% \endlastfoot
% 
%f1  & 1500 & \textbf{1.8E-60}   & 1.3E-54   & 2.5E-28   & 4.5E-20   & 9.8E-14   & 9.6E-42   \\\nopagebreak
%    &      & (8.4E-60) & (9.2E-54) & \color{red}(3.5E-28) & (6.9E-20) & (8.4E-14) & (2.7E-41) \\
%f2  & 2000 & 1.8E-25   & 3.9E-22   & 1.5E-23   & 1.9E-14   & 1.6E-09   & 9.3E-21   \\\nopagebreak
%    &      & (8.8E-25) & (2.7E-21) & (1.0E-23) & (1.1E-14) & (1.1E-09) & (6.3E-20) \\
%f3  & 5000 & 5.7E-61   & 6.0E-87   & 5.2E-14   & \color{green}9.0E-37   & 6.6E-11   & 2.5E-19   \\\nopagebreak
%    &      & (2.7E-60) & (1.9E-86) & (1.1E-13) & (5.4E-36) & (8.8E-11) & (3.9E-19) \\
%f4  & 5000 & 8.2E-24   & 4.3E-66   & 1.4E-15   & 7.4E-11   & 4.2E-01   & 4.4E-14   \\\nopagebreak
%    &      & (4.0E-23) & (1.2E-65) & (1.0E-15) & (1.8E-10) & (1.1E+00) & (9.3E-14) \\
%f5  & 3000 & 8.0E-02   & 3.2E-01   & 1.3E+01   & 2.1E+01   & 2.1E+00   & 2.5E+01   \\\nopagebreak
%    &      & (5.6E-01) & (1.1E+00) & (1.4E+01) & (7.8E+00) & (1.5E+00) & (3.2E+01) \\
%f6  & 100  & 2.9E+00   & 5.6E+00   & 1.0E+03   & 9.3E+02   & 4.7E+03   & 4.5E+01   \\\nopagebreak
%    &      & (1.2E+00) & (1.6E+00) & (2.2E+02) & (1.8E+02) & (1.1E+03) & (2.4E+01) \\
%f7  & 3000 & 6.4E-04   & 6.8E-04   & 3.3E-03   & 4.8E-03   & 4.7E-03   & 2.5E-03   \\\nopagebreak
%    &      & (2.5E-04) & (2.5E-04) & (8.5E-04) & (1.2E-03) & (1.2E-03) & (1.4E-03) \\
%f8  & 1000 & 3.3E-05   & 7.1E+00   & 7.9E-11   & 4.7E+00   & 5.9E+03   & 2.4E+03   \\\nopagebreak
%    &      & (2.3E-05) & (2.8E+01) & (1.3E-10) & (3.3E+01) & (1.1E+03) & (6.7E+02) \\
%f9  & 1000 & 1.0E-04   & 1.4E-04   & 1.5E-04   & 1.2E-03   & 1.8E+02   & 5.2E+01   \\\nopagebreak
%    &      & (6.0E-05) & (6.5E-05) & (2.0E-04) & (6.5E-04) & (1.3E+01) & (1.6E+01) \\
%f10 & 500  & 8.2E-10   & 3.0E-09   & 3.5E-04   & 2.7E-03   & 1.1E-01   & 4.6E-01   \\\nopagebreak
%    &      & (6.9E-10) & (2.2E-09) & (1.0E-04) & (5.1E-04) & (3.9E-02) & (6.6E-01) \\
%f11 & 500  & 9.9E-08   & 2.0E-04   & 1.9E-05   & 7.8E-04)  & 2.0E-01   & 1.3E-02   \\\nopagebreak
%    &      & (6.0E-07) & (1.4E-03) & (5.8E-05) & (1.2E-03  & (1.1E-01) & (1.7E-02) \\
%f12 & 500  & 4.6E-17   & 3.8E-16   & 1.6E-07   & 1.9E-05   & 1.2E-02   & 1.9E-01   \\\nopagebreak
%    &      & (1.9E-16) & (8.3E-16) & (1.5E-07) & (9.2E-06) & (1.0E-02) & (3.9E-01) \\
%f13 & 500  & 2.0E-16   & 1.2E-15   & 1.5E-06   & 6.1E-05   & 7.5E-02   & 2.9E-03   \\\nopagebreak
%    &      & (6.5E-16) & (2.8E-15) & (9.8E-07) & (2.0E-05) & (3.8E-02) & (4.8E-03) \\
%f1  & 1500 & \textbf{1.8E-60}   & 1.3E-54   & 2.5E-28   & 4.5E-20   & 9.8E-14   & 9.6E-42   \\\nopagebreak
%    &      & (8.4E-60) & (9.2E-54) & \color{red}(3.5E-28) & (6.9E-20) & (8.4E-14) & (2.7E-41) \\
%f2  & 2000 & 1.8E-25   & 3.9E-22   & 1.5E-23   & 1.9E-14   & 1.6E-09   & 9.3E-21   \\\nopagebreak
%    &      & (8.8E-25) & (2.7E-21) & (1.0E-23) & (1.1E-14) & (1.1E-09) & (6.3E-20) \\
%f3  & 5000 & 5.7E-61   & 6.0E-87   & 5.2E-14   & 9.0E-37   & 6.6E-11   & 2.5E-19   \\\nopagebreak
%    &      & (2.7E-60) & (1.9E-86) & (1.1E-13) & (5.4E-36) & (8.8E-11) & (3.9E-19) \\
%f4  & 5000 & 8.2E-24   & 4.3E-66   & 1.4E-15   & 7.4E-11   & 4.2E-01   & 4.4E-14   \\\nopagebreak
%    &      & (4.0E-23) & (1.2E-65) & (1.0E-15) & (1.8E-10) & (1.1E+00) & (9.3E-14) \\
%f5  & 3000 & 8.0E-02   & 3.2E-01   & 1.3E+01   & 2.1E+01   & 2.1E+00   & 2.5E+01   \\\nopagebreak
%    &      & (5.6E-01) & (1.1E+00) & (1.4E+01) & (7.8E+00) & (1.5E+00) & (3.2E+01) \\
%f6  & 100  & 2.9E+00   & 5.6E+00   & 1.0E+03   & 9.3E+02   & 4.7E+03   & 4.5E+01   \\\nopagebreak
%    &      & (1.2E+00) & (1.6E+00) & (2.2E+02) & (1.8E+02) & (1.1E+03) & (2.4E+01) \\
%f7  & 3000 & 6.4E-04   & 6.8E-04   & 3.3E-03   & 4.8E-03   & 4.7E-03   & 2.5E-03   \\\nopagebreak
%    &      & (2.5E-04) & (2.5E-04) & (8.5E-04) & (1.2E-03) & (1.2E-03) & (1.4E-03) \\
%f8  & 1000 & 3.3E-05   & 7.1E+00   & 7.9E-11   & 4.7E+00   & 5.9E+03   & 2.4E+03   \\\nopagebreak
%    &      & (2.3E-05) & (2.8E+01) & (1.3E-10) & (3.3E+01) & (1.1E+03) & (6.7E+02) \\
%f9  & 1000 & 1.0E-04   & 1.4E-04   & 1.5E-04   & 1.2E-03   & 1.8E+02   & 5.2E+01   \\\nopagebreak
%    &      & (6.0E-05) & (6.5E-05) & (2.0E-04) & (6.5E-04) & (1.3E+01) & (1.6E+01) \\
%f10 & 500  & 8.2E-10   & 3.0E-09   & 3.5E-04   & 2.7E-03   & 1.1E-01   & 4.6E-01   \\\nopagebreak
%    &      & (6.9E-10) & (2.2E-09) & (1.0E-04) & (5.1E-04) & (3.9E-02) & (6.6E-01) \\
%f11 & 500  & 9.9E-08   & 2.0E-04   & 1.9E-05   & 7.8E-04)  & 2.0E-01   & 1.3E-02   \\\nopagebreak
%    &      & (6.0E-07) & (1.4E-03) & (5.8E-05) & (1.2E-03  & (1.1E-01) & (1.7E-02) \\
%f12 & 500  & 4.6E-17   & 3.8E-16   & 1.6E-07   & 1.9E-05   & 1.2E-02   & 1.9E-01   \\\nopagebreak
%    &      & (1.9E-16) & (8.3E-16) & (1.5E-07) & (9.2E-06) & (1.0E-02) & (3.9E-01) \\
%f13 & 500  & 2.0E-16   & 1.2E-15   & 1.5E-06   & 6.1E-05   & 7.5E-02   & 2.9E-03   \\\nopagebreak
%    &      & (6.5E-16) & (2.8E-15) & (9.8E-07) & (2.0E-05) & (3.8E-02) & (4.8E-03) \\
%
%    % \vspace*{1ex}     \\
%%         \midrule%%% тонкий разделитель
%%         \multicolumn{3}{@{}p{\textwidth}}{%
%%             % \vspace*{-4ex}% этим подтягиваем повыше
%%             % \hspace*{2.5em}% абзацный отступ - требование ГОСТ 2.105
%%             Примечание "---  Для функций $f_{12}$ и $f_{13}$
%%             используется $y_i = 1 + \frac{1}{4}(x_i+1)$ и
%%             $u(x_i,\,a,\,k,\,m)=\begin{cases}
%% k(x_i-a)^m,\quad  & x_i >a     \\[-0.5em]
%% 0,\quad           & -a\leq x_i \leq a        \\[-0.5em]
%% k(-x_i-a)^m,\quad & x_i <-a
%% \end{cases}$  }     \\
%\bottomrule %%% нижняя линейка 
%\end{longtabu} \endgroup

