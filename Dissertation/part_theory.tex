\chapter{Теория}
\label{chapter_theory}

\section{Одиночное изображение} \label{sect2_1}

\begin{figure}[ht] 
  \centering
  \includegraphics [scale=0.27] {latex}
  \caption{TeX.}
  \label{img:latex}
\end{figure}

%\newpage
%============================================================================================================================
\section{Длинное название параграфа, в котором мы узнаём как сделать две картинки с~общим номером и названием} \label{sect2_2}

А это две картинки под общим номером и названием:
\begin{figure}[ht]
  \begin{minipage}[ht]{0.49\linewidth}\centering
    \includegraphics[width=0.5\linewidth]{knuth1} \\ а)
  \end{minipage}
  \hfill
  \begin{minipage}[ht]{0.49\linewidth}\centering
    \includegraphics[width=0.5\linewidth]{knuth2} \\ б)
  \end{minipage}
  \caption{Очень длинная подпись к изображению, на котором представлены две фотографии Дональда Кнута}
  \label{img:knuth}  
\end{figure}

Те~же~две картинки под~общим номером и~названием, но с автоматизированной нумерацией подрисунков:
\begin{figure}[ht]
    {\centering
        \hfill
        \subbottom[List-of-Figures entry][Первый подрисунок\label{img:knuth_2_1}]{%
            \includegraphics[width=0.25\linewidth]{knuth1}}
        \hfill
        \subbottom[\label{img:knuth_2_2}]{%
            \includegraphics[width=0.25\linewidth]{knuth2}}
        \hfill
        \subbottom[Третий подрисунок]{%
            \includegraphics[width=0.3\linewidth]{example-image-c}}
        \hfill
    }
    \legend{Подрисуночный текст, описывающий обозначения, например. Согласно
    ГОСТ 2.105, пункт 4.3.1, располагается перед наименованием рисунка.}
    \caption[Этот текст попадает в названия рисунков в списке рисунков]{Очень
    длинная подпись к второму изображению, на котором представлены две
    фотографии Дональда Кнута}
    \label{img:knuth_2}
\end{figure}

На рисунке~\ref{img:knuth_2_1} показан Дональд Кнут без головного убора. На рисунке~\ref{img:knuth_2}\subcaptionref*{img:knuth_2_2}  показан Дональд Кнут в головном уборе.

Возможно вставлять векторные картинки, рассчитываемые \LaTeX\ <<на~лету>>
с~их~предварительной компиляцией. Надписи в таких рисунках будут выполнены
тем же~шрифтом, который указан для документа в целом.
На рисунке~\ref{img:tikz_example} на~странице~\pageref{img:tikz_example} представлен пример схемы, рассчитываемой пакетом \verb|tikz| <<на~лету>>.
Для ускорения компиляции, подобные рисунки могут быть <<кешированы>>, что
определяется настройками в~\verb|common/setup.tex|.
Причём имя предкомпилированного
файла и папка расположения таких файлов могут быть отдельно заданы,
что удобно, если не для подготовки диссертации,
то для подготовки научных публикаций.
\begin{figure}[ht]
    {\centering
        \ifdefmacro{\tikzsetnextfilename}{\tikzsetnextfilename{tikz_example_compiled}}{}% присваиваемое предкомпилированному pdf имя файла
        \input{Dissertation/images/tikz_scheme.tikz}

    }
    \legend{}
    \caption[Пример \texttt{tikz} схемы]{Пример рисунка, рассчитываемого
        \texttt{tikz}, который может быть предкомпилирован}
    \label{img:tikz_example}
\end{figure}

Множество программ имеют либо встроенную возможность экспортировать векторную
графику кодом \verb|tikz|, либо соответствующий пакет расширения.
Например, в GeoGebra есть встроенный экспорт,
для Inkscape есть пакет svg2tikz,
для Python есть пакет matplotlib2tikz,
для R есть пакет tikzdevice.


\section{Пример вёрстки списков} \label{sect2_3}

\noindent Нумерованный список:
\begin{enumerate}
  \item Первый пункт.
  \item Второй пункт.
  \item Третий пункт.
\end{enumerate}

\noindent Маркированный список:
\begin{itemize}
  \item Первый пункт.
  \item Второй пункт.
  \item Третий пункт.
\end{itemize}

\noindent Вложенные списки:
\begin{itemize}
  \item Имеется маркированный список.
  \begin{enumerate}
    \item В нём лежит нумерованный список,
    \item в котором
    \begin{itemize}
      \item лежит ещё один маркированный список.
    \end{itemize}    
  \end{enumerate}
\end{itemize}

\noindent Нумерованные вложенные списки:
\begin{enumerate}
  \item Первый пункт.
  \item Второй пункт.
  \item Вообще, по ГОСТ 2.105 первый уровень нумерации
  (при необходимости ссылки в тексте документа на одно из перечислений)
  идёт буквами русского или латинского алфавитов,
  а второй "--- цифрами со скобками.
  Здесь отходим от ГОСТ.
    \begin{enumerate}
      \item в нём лежит нумерованный список,
      \item в котором
        \begin{enumerate}
          \item ещё один нумерованный список,
          \item третий уровень нумерации не нормирован ГОСТ 2.105;
          \item обращаем внимание на строчность букв,
          \item в этом списке
          \begin{itemize}
            \item лежит ещё один маркированный список.
          \end{itemize}    
        \end{enumerate}

    \end{enumerate}

  \item Четвёртый пункт.
\end{enumerate}

\section{Традиции русского набора}

Много полезных советов приведено в материале
<<\href{http://www.dropbox.com/s/x4hajy4pkw3wdql/wholesome-typesetting.pdf?dl=1\&pv=1}{Краткий курс благородного набора}>> (автор А.\:В.~Костырка).
Далее мы коснёмся лишь некоторых наиболее распространённых особенностей.

\subsection{Пробелы}

В~русском наборе принято:
\begin{itemize}
    \item единицы измерения, знак процента отделять пробелами от~числа: 10~кВт, 15~\% (согласно ГОСТ 8.417, раздел 8);
    \item $\tg 20^\circ$, но: 20~${}^\circ$C (согласно ГОСТ 8.417, раздел 8);
    \item знак номера, параграфа отделять от~числа: №~5, \S~8;
    \item стандартные сокращения: т.\:е., и~т.\:д., и~т.\:п.;
    \item неразрывные пробелы в~предложениях.
\end{itemize}

\subsection{Математические знаки и символы}

Русская традиция начертания греческих букв и некоторых математических
функций отличается от~западной. Это исправляется серией
\verb|\renewcommand|.
\begin{itemize}
%Все \original... команды заранее, ради этого примера, определены в Dissertation\userstyles.tex
    \item[До:] \( \originalepsilon \originalge \originalphi\),
    \(\originalphi \originalleq \originalepsilon\),
    \(\originalkappa \in \originalemptyset\),
    \(\originaltan\),
    \(\originalcot\),
    \(\originalcsc\).
    \item[После:] \( \epsilon \ge \phi\),
    \(\phi \leq \epsilon\),
    \(\kappa \in \emptyset\),
    \(\tan\),
    \(\cot\),
    \(\csc\).
\end{itemize}

Кроме того, принято набирать греческие буквы вертикальными, что
решается подключением пакета \verb|upgreek| (см. закомментированный
блок в~\verb|userpackages.tex|) и~аналогичным переопределением в
преамбуле (см.~закомментированный блок в \verb|userstyles.tex|). В
этом шаблоне такие переопределения уже включены.

Знаки математических операций принято переносить. Пример переноса
в~формуле \eqref{eq:equation3}.

\subsection{Кавычки}
В английском языке приняты одинарные и двойные кавычки в~виде ‘...’ и~“...”. В России приняты французские («...») и~немецкие („...“) кавычки (они называются «ёлочки» и~«лапки», соответственно). <<Лапки>> обычно используются внутри ,,ёлочек``, например, <<... наш гордый ,,Варяг``...>>.

Французкие левые и правые кавычки набираются
как лигатуры \verb|<<| и \verb|>>|, а~немецкие левые и правые кавычки набираются как лигатуры \verb|,,| и \verb|‘‘| (\verb|``|).

Вместо лигатур или команд с~активным символом "\ можно использовать команды \verb|\glqq| и \verb|\grqq| для набора немецких кавычек и команды \verb|\flqq| и~\verb|\frqq| для набора французских кавычек. Они определены в пакете \verb|babel|.

\subsection{Тире}
%  babel+pdflatex по умолчанию, в polyglossia надо включать опцией (и перекомпилировать с удалением временных файлов)
Команда \verb|"---| используется для печати тире в тексте. Оно несколько короче английского длинного тире. Кроме того, команда задаёт небольшую жёсткую отбивку от слова, стоящего перед тире. При этом, само тире не отрывается от~слова. После тире следует такая же отбивка от текста, как и перед тире. При наборе текста между словом и командой, за которым она следует, должен стоять пробел.

В составных словах, таких, как <<Закон Менделеева"--~Клапейрона>>, для печати тире надо использовать команду \verb|"--~|. Она ставит более короткое, по~сравнению с~английским, тире и позволяет делать переносы во втором слове. При~наборе текста команда \verb|"--~| не отделяется пробелом от слова, за которым она следует (\verb|Менделеева"--~|). Следующее за командой слово может быть  отделено от~неё пробелом или перенесено на другую строку.

Если прямая речь начинается с~абзаца, то перед началом её печатается тире командой
\verb|"--*|. Она печатает русское тире и жёсткую отбивку нужной величины перед текстом.

\subsection{Дефисы и переносы слов}
%  babel+pdflatex по умолчанию, в polyglossia надо включать опцией (и перекомпилировать с удалением временных файлов)
Для печати дефиса в~составных словах введены две команды. Команда~\verb|"~| печатает дефис и~запрещает делать переносы в~самих словах, а~команда \verb|"=| печатает дефис, оставляя \TeX ’у право делать переносы в~самих словах.

В отличие от команды \verb|\-|, команда \verb|"-| задаёт место в~слове, где можно делать перенос, не~запрещая переносы и~в~других местах слова.

Команда \verb|""| задаёт место в~слове, где можно делать перенос, причём дефис при~переносе в~этом месте не~ставится.

Команда \verb|",| вставляет небольшой пробел после инициалов с~правом переноса в~фамилии.

\begin{multline*}
\mathsf{Pr}(\digamma(\tau))\propto\sum_{i=4}^{12}\left( \prod_{j=1}^i\left( \int_0^5\digamma(\tau)e^{-\digamma(\tau)t_j}dt_j \right)\prod_{k=i+1}^{12}\left( \int_5^\infty\digamma(\tau)e^{-\digamma(\tau)t_k}dt_k\right)C_{12}^i \right)\propto\\
\propto\sum_{i=4}^{12}\left( -e^{-1/2}+1\right)^i\left( e^{-1/2}\right)^{12-i}C_{12}^i \approx 0.7605,\quad \forall\tau\neq\overline{\tau}
\end{multline*}


%Большая фигурная скобка только справа
\[\left.                                                          %ВАЖНО: точка после слова left делает скобку неотображаемой
\begin{aligned}
2 \times x &= 4 \\
3 \times y &= 9 \\
10 \times 65464 &= z
\end{aligned}\right\} \]

\[ \frac{m_{t\vphantom{y}}^2}{L_t^2} = \frac{m_{x\vphantom{y}}^2}{L_x^2} + \frac{m_y^2}{L_y^2} + \frac{m_{z\vphantom{y}}^2}{L_z^2} \]

