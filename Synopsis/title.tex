\thispagestyle{empty}

\noindent%
\begin{tabularx}{\textwidth}{@{}lXr@{}}%
    & & \large{На правах рукописи}\\
    \IfFileExists{images/logo.pdf}{\includegraphics[height=2.5cm]{logo}}{\rule[0pt]{0pt}{2.5cm}}  & &
    \ifnumequal{\value{showperssign}}{0}{%
        \rule[0pt]{0pt}{1.5cm}
    }{
        %\includegraphics[height=1.5cm]{personal-signature.png}
    }\\
\end{tabularx}

\vspace{0pt plus1fill} %число перед fill = кратность относительно некоторого расстояния fill, кусками которого заполнены пустые места
\begin{center}
\textbf {\large \thesisAuthor}
\end{center}

\vspace{0pt plus3fill} %число перед fill = кратность относительно некоторого расстояния fill, кусками которого заполнены пустые места
\begin{center}
\textbf {\Large %\MakeUppercase
\thesisTitle}

\vspace{0pt plus3fill} %число перед fill = кратность относительно некоторого расстояния fill, кусками которого заполнены пустые места
{\large Специальность \thesisSpecialtyNumber\ "---\par <<\thesisSpecialtyTitle>>}

\vspace{0pt plus1.5fill} %число перед fill = кратность относительно некоторого расстояния fill, кусками которого заполнены пустые места
\Large{Автореферат}\par
\large{диссертации на соискание учёной степени\par \thesisDegree}
\end{center}

\vspace{0pt plus4fill} %число перед fill = кратность относительно некоторого расстояния fill, кусками которого заполнены пустые места
{\centering\thesisCity~--- \thesisYear\par}

\newpage
{
% оборотная сторона обложки
\thispagestyle{empty}
\SingleSpacing
\noindent Работа выполнена в {Федеральном государственном бюджетном учреждении науки Институте системного программирования им. В.П. Иванникова Российской академии наук} и в {Федеральном государственном автономном образовательном учреждении высшего образования <<Московский физико-технический институт (национальный исследовательский университет)>>}.

\vspace{1mm}
\noindent%
\begin{tabularx}{\textwidth}{@{}lX@{}}
    Научный руководитель:   & \textbf{\supervisorFio}\par
                              \supervisorRegalia 
\\
%    \vspace{1mm}\\
    Официальные оппоненты:  &
    \ifnumequal{\value{showopplead}}{0}{\vspace{1mm}}{%
        \textbf{\opponentOneFio,}\par
        \opponentOneRegalia,\par
        \opponentOneJobPlace,\par
        \opponentOneJobPost\par
            \vspace{1mm}
        \textbf{\opponentTwoFio,}\par
        \opponentTwoRegalia,\par
        \opponentTwoJobPlace,\par
        \opponentTwoJobPost
    }%
    \vspace{1mm} \\
    Ведущая организация:    &
    \ifnumequal{\value{showopplead}}{0}{\vspace{1mm}}{%
        \leadingOrganizationTitle
    }%
\end{tabularx}
\vspace{1mm}

\noindent Защита состоится \defenseDate~на~заседании диссертационного совета \defenseCouncilNumber~при \defenseCouncilTitle~по адресу: \defenseCouncilAddress.

\vspace{1mm}
\noindent С диссертацией можно ознакомиться в библиотеке и на сайте Федерально­го государственного бюджетного учреждения науки Институте системногопрограммирования им. В. П. Иванникова РАН.

%\vspace{0.008\paperheight plus1fill}
%\noindent Отзывы на автореферат в двух экземплярах, заверенные печатью учреждения, просьба направлять по адресу: \defenseCouncilAddress, ученому секретарю диссертационного совета~\defenseCouncilNumber.

\vspace{1mm}
\noindent{Автореферат разослан \synopsisDate.}

%\noindent Телефон для справок: \defenseCouncilPhone.

\vspace{1pt}
\noindent%
\begin{tabularx}{\textwidth}{@{}%
>{\raggedright\arraybackslash}b{18em}@{}
>{\centering\arraybackslash}X
r
@{}}
    Ученый секретарь\par
    диссертационного совета
    \defenseCouncilNumber,\par
    \defenseSecretaryRegalia
    &
    \ifnumequal{\value{showsecrsign}}{0}{}{%
        %\includegraphics[width=2cm]{secretary-signature.png}%
    }%
    &
    \defenseSecretaryFio
\end{tabularx} 

}