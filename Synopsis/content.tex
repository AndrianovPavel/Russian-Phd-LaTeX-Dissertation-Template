
\section*{Общая характеристика работы}

\newcommand{\actuality}{\underline{\textbf{\actualityTXT}}}
\newcommand{\progress}{\underline{\textbf{\progressTXT}}}
\newcommand{\aim}{\underline{{\textbf\aimTXT}}}
\newcommand{\tasks}{\underline{\textbf{\tasksTXT}}}
\newcommand{\novelty}{\underline{\textbf{\noveltyTXT}}}
\newcommand{\influence}{\underline{\textbf{\influenceTXT}}}
\newcommand{\methods}{\underline{\textbf{\methodsTXT}}}
\newcommand{\defpositions}{\underline{\textbf{\defpositionsTXT}}}
\newcommand{\reliability}{\underline{\textbf{\reliabilityTXT}}}
\newcommand{\probation}{\underline{\textbf{\probationTXT}}}
\newcommand{\contribution}{\underline{\textbf{\contributionTXT}}}
\newcommand{\publications}{\underline{\textbf{\publicationsTXT}}}


{\actuality} Информационные технологии являются важной составляющей инфраструктуры современного общества.
Они позволяют автоматизировать различные процессы жизнедеятельности человека и обеспечить возможность коммуникации.
%В настоящее время невозможно представить себе высокотехнологичное производство или сервис услуг, которые бы обходились без использования информационных технологий.
%Более того, сейчас активно развиваются технологии, которые позволяют автоматизировать бытовые потребности человека.
%К таким технологиям относится «интернет вещей» (англ. Internet of Things, IoT).
%Это означает, что жизнь человека становится все более и более зависимой от программных систем.
%Для эффективного использования многоядерных систем широко используется многопоточное программное обеспечение, в том числе и системное программное обеспечение.
Таким образом, задача повышения производительности компьютерных систем становится чрезвычайно важной.
Развитие технологий многоядерности и многопоточности являются важнейшими направлениями решения данной задачи.
Например, в ядре операционной системы может одновременно выполняться большое число (несколько десятков) совершенно различных активностей: обработчики прерываний, системные вызовы от пользовательских программ, внутренние службы ядра, например, планировщик, драйвера внешних устройств.
Только за счет использования многопоточности современные системы могут обеспечивать необходимые показатели скорости и производительности.
%

%Информационные системы непрерывно развиваются и усложняются. 
%С ростом сложности программных продуктов росло и количество ошибок в них.
%Повышение роли программного обеспечения в жизнедеятельности человека приводит к увеличению величины последствий от отказа или некорректного поведения.
%Некоторые ошибки способны сильно исказить правильное выполнение программы и привести к серьезным последствиям.
%В 2017 году компания Coverity опубликовала очередной отчет, в котором были приведены результаты исследования изучения 760 млн строк кода.
%В среднем, в программном продукте содержится около 1.4 ошибок на 1000 строк кода.
%Это число ошибок значительно вырасло по сравнению с отчетами предыдущих лет.
%Таким образом, число проблем в проекте размером миллион строк кода исчисляется тысячами.
%Степень критичности ошибки также зависит и от той области, в которой применяется программное обеспечение.
%Ошибки в программных системах, используемых в авиации или на атомных электростанциях, могут привести к значительным человеческим жертвам или колоссальным финансовым затратам.
%Известно множество случаев, в которых ошибки в программном обеспечении приводили к серьезным последствиям, например, случаи с аппаратом лучевой терапии Therac-25 и с космическим аппаратом 
%Phobos\citeinsynopsis{Leveson:1995}.
%Таким образом, с момента возникновения первых вычислительных систем перед людьми всегда стояла задача проверки правильности программы. 

%Под верификацией будем понимать проверку соответствия одних создаваемых в ходе разработки и сопровождения программного обеспечения сущностей другим, ранее созданным или используемым в качестве исходных данных, а также соответствие этих сущностей и процессов их разработки правилам и стандартам [4]. В качестве сущностей могут выступать непосредственно программы, модели, документы и др.
%Существует несколько методов верификации: экспертиза, статический анализ, формальные методы, динамические методы, синтетические методы. 

%Поэтому с самого начала развития информатики развиваются и совершенствуются методы проверки программ и доказательства их корректности. Их уже достаточно много: от ручных математических доказательств, до динамического тестирования. Они прошли долгий путь от формальных математических доказательств, которые были доступны только специалистам, обладающим хорошим математическим образованием, до прикладных инструментов, понятных даже инженерам и разработчикам прикладного программного обеспечения.

%Развитие параллельных вычислений привело к быстрому разрастанию кода, который предполагает параллельное исполнение.
%%Цифры? 
%Параллельные алгоритмы позволяют более эффективно использовать доступные вычислительные ресурсы.
%Основной сложностью таких алгоритмов является обеспечение синхронизации между потоками и корректным использованием разделяемых ресурсов.
%В связи с этим появляются новые типы ошибок, характерные только для параллельно исполняемого кода. 

В дополнении к ошибкам, которые встречаются во всех видах программного обеспечения, многопоточные программы могут содержать специфические ошибки, связанные с параллельным выполнением: например, состояния взаимной блокировки и состояния <<гонки>>.
В общем случае состоянием гонки называют ситуацию, при которой поведение программы зависит от порядка или времени выполнения некоторых неконтролируемых событий.
Важное уточнение заключается в том, что такое выполнение не всегда является ошибкой.
Проблемы возникают тогда, когда разработчик не предусматривает некоторое из возможных поведений программы.
Часто рассматривают более узкий класс -- <<состояния гонки по данным>>. Эта ситуация возникает при одновременном доступе к данным из разных потоков (процессов).
%Здесь и далее не будем различать потоки и процессы, так как их различия не имеют отношения к теме работы.
Состояния гонки по данным становятся опасными в случае, если имеет место хотя бы один доступ на запись в разделяемую область памяти.
%Пример!
В этом случае результирующее значение переменной зависит от порядка выполнения инструкций, а в параллельно выполняемом коде, в общем случае, последовательность выполнения инструкций не определена. 

%Стоит упомянуть о высокоуровневых состояниях гонки по данным.
%Такие состояния гонки по данным отличается тем, что доступ производится к разным разделяемым данным, которые тем не менее являются семантически связанными.
%Например, это может быть реализация сложных структур данных, таких как двусвязные списки, деревья, графы и т. п.
%В случае модификации такой структуры данных должна быть обеспечена атомарность, в противном случае, данные могут стать неконсистентными, например, не нарушится целостность списка. 
 
Состояния взаимных блокировок являются вторым большим классом ошибок в многопоточных программах.
Они возникают при некорректном использовании блокирующих механизмов синхронизации.
В этом случае все потоки системы находятся в ожидании некоторого разблокирующего действия от других потоков и не могут продолжить свое выполнение. 

Оба класса ошибок так или иначе связаны с некорректной синхронизацией параллельно выполняющихся потоков.
Такие ошибки, связанные с параллельным исполнением кода, искать и исправлять гораздо сложнее, чем ошибки в последовательном коде, так как необходимо анализировать все возможные сценарии взаимодействия потоков.
Кроме того, поиск и исправление таких ошибок осложняется случайным характером их проявлений, ошибка может проявляться очень редко.
Это связано с тем, что для проявления ошибки необходимы некоторая конкретная последовательность и порядок действий различных потоков. 
В системном программном обеспечении могут использоваться не только обыкновенные примитивы синхронизации, но и специальные низкоуровневые, например, запреты прерываний и планирования. 
Таким образом, обнаружить ошибку, связанную с некорректной синхронизацией потоков, вручную практически невозможно, поэтому для их поиска используются различные автоматизированные подходы.
Кроме этого, для анализа системного программного обеспечения, например, операционных систем, требутся специализированные инструменты, которые будут учитывать их особенности.

Обычно выделяют два класса подходов к анализу программ: статические, которые проводят анализ исходного кода программы, и динамические, которые анализируют поведение программы в процессе ее выполнения.
Каждый имеет свои достоинства и недостатки, например, методы динамического анализа обычно характеризуются высоким уровнем истинных предупреждений, в то время как статические методы позволяют покрыть больший объем кода.
На данный момент не существует универсального подхода для анализа программ.

В прикладных пользовательских программах может быть достаточно провести тщательное тестирование, возможно, с помощью инструментов динамического анализа, чтобы проверить основные сценарии поведения программы.
%Однако, такое тестирование не дает гарантий корректного поведения даже при тех же самых условиях и входных данных, что для пользовательских приложений является приемлемым вариантом.
В случае же системного программного обеспечения цена пропущенной ошибки может быть слишком велика.
Кроме того, некоторые сценарии поведения программы слишком сложно воспроизвести при реальном выполнении.
Устройство системного программного обеспечения отличается от прикладных программ, что затрудняет анализ, так как далеко не все методы принимают во внимание специфику системного программного обеспечения.
Поэтому применение только динамического анализу становится недостаточным, и в дополнение к ним применяются методы статической верификации.

Методы статической верификации способны обеспечить доказательство отсутствия ошибок в некоторых заранее заданных предположениях, но при этом традиционно испытывают сложности при масштабировании на большие объемы исходного кода.
Тем не менее, такие методы показывают высокую точность, что является важной особенностью при анализе такого программного обеспечения, к которому предъявляются высокие требования качества.
%Развитие подобных методов позволит повысить качество системного программного обеспечения.
Таким образом, разработка масштабируемого метода статической верификации для поиска состояний гонки является актуальной задачей.

%В ядре операционной системы может одновременно выполняться большое число (несколько десятков) совершенно различных функций: обработчики прерываний, системные вызовы от пользовательских программ, внутренние службы ядра, например, планировщик, драйвера внешних устройств.
%Для синхронизации всех этих функций используются не только обыкновенные примитивы синхронизации, но и специальные низкоуровневые, которые характерны только для системного программного обеспечения, например, запреты прерываний и планирования. 
%Различные исследования показывают, что ошибки, связанные с параллельным выполнением, в системном программном обеспечении являются достаточно многочисленными, например, к ним относятся около 20\% всех ошибок в файловых
%системах~\citeinsynopsis{Palix11}.
%
%Верификация многопоточных программ всегда являлась более сложной задачей, чем верификация последовательных программ.
%Точное вычисление всех возможных чередований (англ. interleavings), приводит к комбинаторному взрыву числа состояний.
%Поэтому, большинство инструментов статической верификации применяют арзличные техники оптимизации: редукция частичных порядков (англ. partial order reduction)~\citeinsynopsis{Abdulla:2014},\citeinsynopsis{Godefroid:1996}, абстракция счетчика (англ. counter abstraction)~\citeinsynopsis{Basler:2009} и другие.
%Тем не менее, большинство современных инструментов статической верификации плохо масштабируются на промышленное программное обеспечение.
%Этот факт подтверждается соревнованием по статической верификации~\citeinsynopsis{svcomp19}.
%Задачи из категории «многопоточность», основанные на драйверах операционной системы Linux вызывают значительные сложности для всех инструментов статической верификации.
%
%Противоположностью методам проверки моделей являются методы статического анализа, которые нацелены на быстрый поиск ошибок без какой-либо уверенности в финальном вердикте.
%Такие инструменты применяют различные фильтры и эвристики для ускорения анализа и поэтому не могут гарантировать корректность, то есть отсутствие ошибок.
%В данной работе представлен подход к статической верификации многопоточного программного обеспечения, который позволяет легко настраивать баланс между скоростью и точностью проводимого анализа. Кроме того, он легко нацеливается на конкретную задачу.

%В частности, анализ типовых ошибок, исправленных за год в ядре операционной системы Linux, показал, что ошибки, связанные с состоянием гонки образуют самый многочисленный класс и составляют около 17\% от всех 
%\ifsynopsis
%ошибок.
%\else
%ошибок~\cite{commit_analysis_12}.
%\fi

%Таким образом, задача поиска ошибок синхронизации в ядрах операционных систем, в том числе, состояний гонки по данным, является важной и актуальной задачей.

%\ifsynopsis
%%Этот абзац появляется только в~автореферате.
%\else
%
%%Динамический анализ системного программного обеспечения не обеспечивает должного качества для применения целевого ПО в критически важных областях. В таких случаях необходимо применение формальных методов верификации. Основной сложностью данного подхода является высокая сложность точного моделирования системного программного обеспечения, так как неизбежно возникает проблема комбинаторного взрыва. Тем не менее, сейчас активно применяется формальная верификация для доказательств корректности отдельных подсистем (модулей) программной системы с формулировкой требований на корректное использование его интерфейсов другими частями системы.
%
%Даже при анализе одного потока сложной программной системы возникают сложности с анализом циклов, битовых операций, адресной арифметики. 
%На сколько учитывать окружение и другие потоки?
%\fi

% {\progress} 
% Этот раздел должен быть отдельным структурным элементом по
% ГОСТ, но он, как правило, включается в описание актуальности
% темы. Нужен он отдельным структурынм элемементом или нет ---
% смотрите другие диссертации вашего совета, скорее всего не нужен.

{\aim} данной работы является разработка метода поиска состояний гонок, который будет масштабироваться на большие объемы кода, будет обладать приемлемым уровнем ложных предупреждений и будет учитывать специфику ядра операционных систем.

Для~достижения поставленной цели необходимо было решить следующие {\tasks}:
\begin{enumerate}
  \item Разработать общий алгоритм, позволяющий реализовать подход к верификации программного обеспечения с раздельным анализом потоков, и доказать его корректность;
  \item Разработать метод поиска состояний гонки, как частный случай общего алгоритма, который может эффективно применяться к большим объемам исходного кода, и доказать его корректность;
  \item Реализовать разработанные алгоритмы;
  \item Провести эксперименты и сравнить результаты с другими инструментами статической верификации;
\end{enumerate}

{\novelty}
\begin{enumerate}
  \item Был предложен новый алгоритм, который является обобщением существующего алгоритма CPA, и доказана его корректность;
  \item Был предложен частный случай обобщенного алгоритма, который реализует подход с раздельным анализом потоков, и доказана его корректность;
  \item Разработан метод поиска состояний гонки и доказана его корректность;
  %\item Были проведены запуски инструмента на модулях ядра операционной системы Linux, а также на двух операционных системах реального времени;
\end{enumerate}

%{\influence} \ldots

%{\methods} \ldots

%{\defpositions}
\textbf{Положения, выносимые на публичное представление}
\begin{enumerate}
  \item Обобщенный алгоритм анализа программ, позволяющий использовать различные виды анализа, и доказательство его корректности;
  \item Алгоритм, позволяющий проводить раздельный анализ потоков многопоточных программ, и доказательство его корректности;
  \item Метод поиска состояний гонки в многопоточных программах.
\end{enumerate}

%{\reliability} полученных результатов обеспечивается \ldots \ Результаты находятся в соответствии с результатами, полученными другими авторами.

{\probation}
Основные результаты работы докладывались~на:
\begin{itemize}
  % \item Весенний коллоквиум молодых исследователей в области программной инженерии (SYRCoSE: Spring Young Researchers Colloquium on Software Engineering), Санкт-Петербург, 2014 г.;
  % \item Научно-исследовательский семинар лаборатории «Software and Computational Systems Lab» Университета Пассау, Германия, 2014 г.
  \item Международная научно-практическая конференция <<Инструменты и методы анализа программ>> (TMPA: Tools and Methods for Program Analysis), Кострома, 2014 г.
  \item Международный семинар разработчиков CPAchecker, Москва, 2015 г.
  \item Летняя научная школа компании Microsoft (Microsoft Summer School), Кэмбридж, Англия, 2015 г.
  \item Научно-практическая Открытая конференция ИСП РАН, Москва, 2016 г.
  \item Научно-исследовательский семинар лаборатории <<Software and Computational Systems Lab>> Университета Пассау, Германия, 2016 г.
  \item Международная научно-практическая конференция <<Инструменты и методы анализа программ>> (TMPA: Tools and Methods for Program Analysis), Москва, 2017 г.
  \item Международный семинар разработчиков CPAchecker, Падерборн, Германия, 2017 г.
  \item Международный семинар разработчиков CPAchecker, Москва, Россия, 2018 г.
  \item Соревнования по статической верификации SV-COMP, Прага, Чехия, 2019 г.
  \item Международный семинар разработчиков CPAchecker, Фрауенинзель, Германия, 2019 г.
  \item Семинар <<Математические вопросы информатики>> Мехмат МГУ, Москва, Россия, 2019 г.
\end{itemize}

% {\contribution} Автор принимал активное участие \ldots

\publications\ Основные результаты по теме диссертации изложены в 6 печатных изданиях~\cite{lockatorVAK,lockatorVAK2,TMPA2017,theoryVAK,lockatorSyrcose,lockatorTMPA}, 
    4 из которых изданы в журналах, рекомендованных ВАК~\cite{lockatorVAK,lockatorVAK2,TMPA2017,theoryVAK}, из них 1 находится в базе Scopus~\cite{TMPA2017},
    2 "--- в тезисах докладов~\cite{lockatorSyrcose,lockatorTMPA}.

В статье~\cite{lockatorVAK} автором описана основная идея метода (глава 3) и его реализация (глава 5).
В статье~\cite{lockatorVAK2} автором написаны разделы, посвященные общей идее метода (глава 3), его реализации (глава 4), процессу уточнения (глава 5) и анализу потоков (глава 6).
В статье~\cite{TMPA2017} автором написаны разделы, посвященные разработанному методу и его реализации (главы 3--6).
В статье~\cite{lockatorSyrcose} автором написаны разделы, в которых описывается ключевые особенности метода (главы 3--5).
В статье~\cite{lockatorTMPA} автором написаны разделы, посвященные разработанному методу (главы 3--5).

 % Характеристика работы по структуре во введении и в автореферате не отличается (ГОСТ Р 7.0.11, пункты 5.3.1 и 9.2.1), потому её загружаем из одного и того же внешнего файла, предварительно задав форму выделения некоторым параметрам

%\underline{\textbf{Объем и структура работы.}} Диссертация состоит из~введения, четырех глав, заключения и~приложения. Полный объем диссертации \textbf{ХХХ}~страниц текста с~\textbf{ХХ}~рисунками и~5~таблицами. Список литературы содержит \textbf{ХХX}~наименование.

%\newpage
\section*{Содержание работы}
Во \underline{\textbf{введении}} обосновывается актуальность
исследований, проводимых в~рамках данной диссертационной работы,
формулируется цель, ставятся задачи работы, излагается научная новизна
и практическая значимость представляемой работы.
В~первой главе приводится обзор научной литературы по изучаемой проблеме, во второй главе описывается теоретические основы метода, доказывается его корректность в некоторых предположениях.
В третьей главе описывается схема реализации и архитектура разработанного инструмента.
В четвертой главе представлены результаты апробации на частных примерах: как на множестве тестовых примеров, так и на реальном системном коде.

\underline{\textbf{Первая глава}} посвящена обзору существующих методов поиска ошибок использования различных примитивов синхронизации в многопоточном программном обеспечении.
В разделе 1 рассматриваются различные методы статической верификации (англ. software model checking).
Такие методы основаны на том, что автоматически строится некоторая формальная модель программы, а затем эта модель проверяется на соответствие заданным свойствам.
Такие методы являются достаточно точными при условии, что модель достаточно хорошо соответствует исходной программе.
Одним из важных минусов таких методов являются высокие требования к ресурсам.
Другим минусом статической верификации является то, что на реальном программном обеспечении достаточно сложно достигнуть высокого уровня соответствия модели и исходной программы, что приводит к большому количеству ложных срабатываний.
К плюсам можно отнести возможность формального доказательства, что в программе отсутствуют дефекты определенного типа, опять же, при условии адекватности построенной модели.

В разделе 2 рассматриваются методы статического анализа.
Эти методы похожи на те, которые были рассмотрены в предыдущем разделе, однако, в отличие от методов проверки моделей, методы статического анализа не ставят целью возможность доказательства отсутствия ошибок в программе.
Основной задачей таких методов является нахождение как можно большего числа ошибок при наименьших затратах ресурсов. 
Многие из методов статического анализа являются корректными, то есть способны не пропускать ошибки, но в этом случае процент ложных срабатываний будет слишком большой.
Большое количество ложных срабатываний сильно затрудняет ручной анализ предупреждений, поэтому зачастую применяются различные фильтры, которые сокращают число ложных сообщений об ошибках.
Однако такие фильтры являются лишь неточными эвристиками, которые снижают корректность метода в целом и способны привести к пропуску реальной ошибки.

В разделе 3 рассматриваются методы динамического анализа.
Такие методы применяются для поиска ошибок синхронизации во время работы программы, поэтому одна из основных характеристик таких методов - это замедление целевой программы. 
Одной из характерных особенностей данного метода является возможность гарантировать, что обнаруженное предупреждение является реальной ошибкой, а не ложной. 
Однако, такая возможность доступна, если поиск ошибок производится только на тех путях, которые были получены при конкретном выполнении программы.
Перебрать все варианты выполнения даже небольшой программы невозможно за разумное время, поэтому применяются различные механизмы анализа множества путей, похожих на те, которые возникают при реальном выполнении.
Такой подход значительно ускоряет процесс анализа, но приводит к получению ложных предупреждений. 
Тем не менее, процент ложных предупреждений у инструментов, реализующих методы динамического анализа, обычно значительно ниже, чем у тех, что основаны на других подходах.

В разделе 4 подводятся основные итоги обзора и делаются выводы.
Результаты обзора позволяют заключить, что основные усилия сейчас сосредаточены на анализе пользовательских приложений, 
системное программное обеспечение является слишком объемным и слишком сложным для применения общецелевых методов.
Применение методов динамического анализа к системному программному обеспечению осложняется тем, что требуется трудоемкая настройка тестового окружения, в том числе аппаратуры.
Кроме того, методы динамического анализа не способны обеспечить гарантию отсутствия ошибок.
Методы статического анализа успешно применяются к любым объемам кода любой сложности. 
В случае применения таких методов к большому объему сложного кода будет получено огромное количество предупреждений.
Анализ этих предупреждений вручную может потребовать большого количества времени.
Методы статической верификации способны дать гарантию отсутствия дефектов определенного типа, в некоторых разумных предположениях, и, что важно, заранее известных.
Однако, такие методы не способны в настоящее время успешно применяться к большим объемам исходного кода.

Таким образом, можно заключить, что в настоящее время отсутствуют такие методы анализа больших объемов системного кода, в том числе, операционных систем, которые могут обеспечить высокий уровень надежности.
Данная работа посвящена описанию разработанного метода поиска состояний гонки, который может применяться к реальным программным системам.

\underline{\textbf{Вторая глава}} посвящена описанию теории разработанного метода проверки многопоточных программ.
Основная идея заключается в том, что каждый поток в программе анализируется независимо от остальных.
В этом случае удастся избежать комбинаторного взрыва состояний, который бы неизбежно присутствовал, если бы анализ проводился с учетом всех взаимодействий между потоками.
Для того, чтобы обеспечить достаточно высокую точность метода, поток анализируется в некотором окружении, которое формируется другими потоками. 
Уровень точности окружения, который требуется для анализа, может гибко настраиваться.

Метод с раздельным анализом потоков может внутри себя использовать уже существующие техники и подходы, например, такие как CEGAR.
В этом случае необходимо предложить обобщенный алгоритм анализа программ, частным случаем которого уже будут как обычные подходы к анализу последовательных программ, так и различные подходы к анализу многопоточных программ, в частности, предложенный подход с раздельным анализом потоков. 

Для того чтобы доказать корректность предложенного подхода, необходимо определить формальную семантику программы, то есть ее математическую модель.
С помощью предложенного подхода строится некоторая абстракцию этой математической модели. 
Адекватность такой абстракции определяется тем, что каждый из возможных поведений модели должен присутствовать и в абстракции.
Таким образом, ошибка не может быть пропущена.
Именно в этом смысле далее используется термин корректность анализа (англ. soundness).

В разделе 1 описана семантика многопоточных программ. 
Формальная модель программы определяется, как множество конкретных состояний программы, которые описываются значениями переменных каждого из потоков, а также глобальных переменных, статусом примитивов синхронизации и информацией об активных потоках.
Далее описывается семантика всех поддерживаемых операторов программы, то есть задаются правила преобразования конкретных состояний операторами программы.
В число поддерживаемых операторов входят операторы сравнения (англ. assumption), присваивания (англ. assignment),
кроме того, специальные операторы работы с примитивами синхронизации: захват блокировки и ее освобождение, а также операции с потоками: создание потока (thread\_create) и ожидания завершения потока (thread\_join).
В заключении даны определения ошибки в программе, в частности, определение состояния гонки по данным.

В разделе 2 описан обобщенный алгоритм анализа программы, который оперирует абстрактными состояниями программы, то есть такими, которым может соответствовать некоторое множество конкретных.
Обычно для доказательства отсутствия ошибок в программе не требуется строить абстракцию с большим уровнем точности, так как многие детали модели являются ненужными. 
В этом случае множество абстрактных состояний оказывается значительно меньше, чем множество конкретных состояний.
 
Одним из важных особенностей предложенного обобщенного алгоритма является то, что его абстрактные состояния являются частичными, то есть множество конкретных состояний программы, соответствующее некоторому частичному абстрактному состоянию, может зависить не только от этого состояния, но и от других частичных состояний. 
Этот алгоритм оперирует адаптивным статическим анализом (англ., configurable program analysis, CPA), который задается следующими операторами: $transfer$, $prec$, $merge$, $stop$, $update$, $frontier$.
Для того чтобы обеспечить корректность работы алгоритма, каждый из этих операторов должен удовлетворять определенным условиям.

Алгоритм с частичными состояниями является расширением классического алгоритма, используемого в теории CPA, который рассматривает переходы только из каждого абстрактного состояния по-отдельности.
Основным отличием расширенного алгоритма с частичными состояниями является возможность рассмотрения переходов из любого множества абстрактных состояний. 
Это позволяет наложить более слабые условия на операторы анализа, чем этого требует оригинальный алгоритм.
Новые слабые условия позволяют применять более эффективные техники анализа, чем это было раньше.
После этого доказывается основная теорема, которая позволяет гарантировать корректость анализа, если его операторы удовлетворяют поставленным условиям.

В разделе 3 представлен алгоритм с раздельным анализом потоков, который позволяет применять различные техники анализа программы (CPA).
Этот алгоритм является частным случаем общего алгоритма, и оперирует двумя типами объектов: абстрактными состояниями и некоторыми вспомогательными объектами, которые далее будут называться эффектами окружения.
Из абстрактного состояния возможны как обычные переходы на основании операторов программы, так и переходы "по окружению", которые описываются эффектами окружения.
Переходы "по окружению" моделируют изменение состояния отдельного потока в результате действий другого потока.
Для этого алгоритма описаны условия на его операторы $transfer$, $prec$, $merge$, $stop$, а также представлен общий вид операторов $update$, $frontier$, которые не зависят от используемых CPA.
Кроме того, вводится новый оператор $compatible$, который определяет возможность применения перехода "по окружению" к текущему состоянию.
Для каждого оператора показано соответствие поставленных условий тем, которые требуются общим алгоритмом.
После этого определены дополнительные условия на оператор $transfer$ и представлены доказательства корректности этих условий.

Также в этом разделе представлен частный случай алгоритма, который использует анализы, состояния которых инвариантны к преобразованиям окружения.
Важным следствием для такого алгоритма является то, что переходы "по окружению" могут быть опущены, так как они не могут изменить абстрактное состояние анализа.
Это позволяет значительно повысить скорость анализа, однако снижает точность, так как теряется точная информации о том, какие именно действия могут совершить другие активные потоки и приходится учитывать любой вариант их действия.

В разделе 4 представлены описания различных CPA, которые могут быть использованы совместно друг с другом.
\begin{itemize}
\item LocationCPA. Этот служебный анализ отвечает за правильную последовательность применения операторов программы. 
\item PredicateCPA. Предикатный анализ позволяет строить предикатную абстракцию на основе некоторого множества предикатов. 
Требуемые предикаты извлекаются автоматически с помощью процесса уточнения и запросов к внешнему компоненту -- решателю (англ. solver).
\item ThreadCPA. Анализ потоков определяет какие из операций текущего потока могут выполняться параллельно с другими операциями других потоков.
\item LockCPA. Анализ примитивов синхронизации позволяет определить те точки программы, которые защищены некоторыми примитивами синхронизации, что не позволяет выполнять параллельно эти участки кода.
\end{itemize}

Кроме того, описан служебный анализ CompositeCPA, позволяющий использовать несколько различных анализов вместе.
Для каждого анализа представлено формальное определение всех его операторов и доказано, что эти операторы удовлетворяют поставленным условиям корректности.

\underline{\textbf{Третья глава}} посвящена описанию реализации разработанного метода поиска состояний гонки. Далее будем называть инструмент, реализующий предложенный метод, CPALockator. В этой же главе представлены решения, которые используются для эффективного анализа программного обеспечения.

В разделе 1 представлена общая архитектура инструмента, описаны используемые компоненты: анализ разделяемых данных, который используется как предварительный анализ и основной анализ, основанный на комбинации LockCPA, ThreadCPA и PredicateCPA.
Кроме того в этом разделе представлены описания служебных CPA, которые используются  для оптимизации работы всего инструмента: BAMCPA и UsageCPA.
BAMCPA обеспечивает модульный анализ по функциям, то есть результаты анализа функции могут быть переиспользованы в дальнейшем, если эта функция встретится еще раз.
UsageCPA определяет и сохраняет информацию о доступах к памяти, сортируя их, и обеспечивает эффективный доступ к ним.

Конфигурация инструмента CPALockator позволяет настраивать инструмент для конкретного исходного кода.
Например, можно определить функции выделения памяти, что означает, что выделенная ими память является точно локальной.
Также имеется возможность аннотировать различные функции, которые могут вызвать сложности у инструмента, например, функции работы со списками.
В этом случае доступ к отдельным служебным полям, например, $next$, будет считаться, как доступ ко всему списку.

В разделе 2 описаны структуры данных, которые используются для оптимизации работы с большим множеством доступов к памяти, которые были обнаружены в программе.
Здесь описаны используемые контейнеры, которые накапливают в себе информацию о доступах к памяти в процессе анализа: локальные контейнеры, контейнеры для отдельных функций и глобальный контейнер, который содержит в себе полное множество информации.
Также описано устройство структуры данных, используемой при уточнении результатов, которое позволяет применять эффективную навигацию по множеству всех доступов к памяти.

В разделе 3 описан процесс уточнения полученных результатов.
Так как полученная абстракция программы может является аппроксимацией сверху множества состояний программы, возможны ситуации, при которых некоторые важные детали не будут учтены. 
Это приведет к тому, что будет выдано ложное сообщение об ошибке. 
Процесс уточнения позволяет повысить степень уверенности в том, что найденная ошибка является истинной.
Например, может быть проверена локальная достижимость каждого из путей, участвующих в состоянии гонки.
Этот процесс требует достаточно большого количества ресурсов, однако может быть прерван в любой момент.

В разделе 4 представлено описание формата результатов, которое основано на формате witness, уже применяющегося в других инструментах статической верификации.
В этом разделе описан механизм построения трассы, приводящей к состоянию гонки, а также показана возможность их визуализации.

%Можно сослаться на свои работы в автореферате. Для этого в файле
%\verb!Synopsis/setup.tex! необходимо присвоить положительное значение
%счётчику \verb!\setcounter{usefootcite}{1}!. В таком случае ссылки на
%работы других авторов будут подстрочными.
%\ifnumgreater{\value{usefootcite}}{0}{
%Изложенные в третьей главе результаты опубликованы в~\cite{vakbib1, vakbib2}.
%}{}

В \underline{\textbf{четвертой главе}} приведены результаты экспериментов.
Результаты демонстрируют возможности инструмента CPALockator по его гибкой настройки.
В качестве набора небольших программ был выбран открытый тестовый набор sv-comp, на нем CPALockator показал достаточно точные результаты. 
Было получено некоторое количество ложных предупреждений об ошибке, но ни одной реальной ошибки не было пропущено, что подтверждает корректность разработанного метода.
При этом, конечно, другие методы статической верификации способны выдавать меньшее количество ложных предупреждений на таком наборе небольших рукописных программ.
Для проверки масштабируемости инструмента были выбраны несколько модулей операционной системы Linux с заранее известными ошибками.
Более половины из этих ошибок были успешно обнаружены инструментами, при этом в оставшихся случаях верификация завершилась из-за нехватки вычислительных ресурсов.
Если снизить еще снизить требования к точности, все реальные ошибки успешно обнаруживаются, хотя при этом появляется некоторое количество ложных предупреждений.

Гибкая настройка требований к ресурсам позволяет использовать инструмент для проверки достаточно больших программных систем, содержащих сотни тысяч строк кода и проводить их анализ за минуты.
При необходимости для каждой конкретной задачи можно выбрать свой баланс между качеством и требуемыми ресурсами, в зависимости от поставленной цели. 

В \underline{\textbf{заключении}} приведены основные результаты работы, которые заключаются в следующем:
%% Согласно ГОСТ Р 7.0.11-2011:
%% 5.3.3 В заключении диссертации излагают итоги выполненного исследования, рекомендации, перспективы дальнейшей разработки темы.
%% 9.2.3 В заключении автореферата диссертации излагают итоги данного исследования, рекомендации и перспективы дальнейшей разработки темы.
\begin{enumerate}
  \item Был разработан общий алгоритм анализа, позволяющий объединить в себе как классический алгоритм CPA, так и алгоритм с раздельным анализом потоков;
  \item Была доказана теорема, позволяющая гарантировать корректность работы алгоритма;
  \item На основе общего алгоритма был разработан алгоритм с раздельным анализом потоков;
  \item Был разработан метод поиска состояний гонки, базирующийся на подходе с раздельным анализом потоков;
  \item Была доказана корректность предложенного метода поиска состояний гонки, который позволяет гибко балансировать точность и скорость анализа;
  \item Были проведены эксперименты, которые показали высокую степень гибкости разработанного метода.
\end{enumerate}



%\newpage
%При использовании пакета \verb!biblatex! список публикаций автора по теме
%диссертации формируется в разделе <<\publications>>\ файла
%\verb!../common/characteristic.tex!  при помощи команды \verb!\nocite! 

\ifdefmacro{\microtypesetup}{\microtypesetup{protrusion=false}}{} % не рекомендуется применять пакет микротипографики к автоматически генерируемому списку литературы
\ifnumequal{\value{bibliosel}}{0}{% Встроенная реализация с загрузкой файла через движок bibtex8
  \renewcommand{\bibname}{\large \authorbibtitle}
  \nocite{*}
  \insertbiblioauthor           % Подключаем Bib-базы
  %\insertbiblioother   % !!! bibtex не умеет работать с несколькими библиографиями !!!
}{% Реализация пакетом biblatex через движок biber
  \ifnumgreater{\value{usefootcite}}{0}{
%  \nocite{*} % Невидимая цитата всех работ, позволит вывести все работы автора
  \insertbiblioauthorcited      % Вывод процитированных в автореферате работ автора
  }{
  \insertbiblioauthor           % Вывод всех работ автора
%  \insertbiblioauthorgrouped    % Вывод всех работ автора, сгруппированных по источникам
%  \insertbiblioauthorimportant  % Вывод наиболее значимых работ автора (определяется в файле characteristic во второй section)
  \insertbiblioother            % Вывод списка литературы, на которую ссылались в тексте автореферата
  }
}
\ifdefmacro{\microtypesetup}{\microtypesetup{protrusion=true}}{}

