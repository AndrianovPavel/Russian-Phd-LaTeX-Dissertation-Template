
\section*{Общая характеристика работы}

\newcommand{\actuality}{\underline{\textbf{\actualityTXT}}}
\newcommand{\progress}{\underline{\textbf{\progressTXT}}}
\newcommand{\aim}{\underline{{\textbf\aimTXT}}}
\newcommand{\tasks}{\underline{\textbf{\tasksTXT}}}
\newcommand{\novelty}{\underline{\textbf{\noveltyTXT}}}
\newcommand{\influence}{\underline{\textbf{\influenceTXT}}}
\newcommand{\methods}{\underline{\textbf{\methodsTXT}}}
\newcommand{\defpositions}{\underline{\textbf{\defpositionsTXT}}}
\newcommand{\reliability}{\underline{\textbf{\reliabilityTXT}}}
\newcommand{\probation}{\underline{\textbf{\probationTXT}}}
\newcommand{\contribution}{\underline{\textbf{\contributionTXT}}}
\newcommand{\publications}{\underline{\textbf{\publicationsTXT}}}


{\actuality} Информационные технологии являются важной составляющей инфраструктуры современного общества.
Они позволяют автоматизировать различные процессы жизнедеятельности человека и обеспечить возможность коммуникации.
%В настоящее время невозможно представить себе высокотехнологичное производство или сервис услуг, которые бы обходились без использования информационных технологий.
%Более того, сейчас активно развиваются технологии, которые позволяют автоматизировать бытовые потребности человека.
%К таким технологиям относится «интернет вещей» (англ. Internet of Things, IoT).
%Это означает, что жизнь человека становится все более и более зависимой от программных систем.
%Для эффективного использования многоядерных систем широко используется многопоточное программное обеспечение, в том числе и системное программное обеспечение.
Таким образом, задача повышения производительности компьютерных систем становится чрезвычайно важной.
Развитие технологий многоядерности и многопоточности являются важнейшими направлениями решения данной задачи.
Например, в ядре операционной системы может одновременно выполняться большое число (несколько десятков) совершенно различных активностей: обработчики прерываний, системные вызовы от пользовательских программ, внутренние службы ядра, например, планировщик, драйвера внешних устройств.
Только за счет использования многопоточности современные системы могут обеспечивать необходимые показатели скорости и производительности.
%

%Информационные системы непрерывно развиваются и усложняются. 
%С ростом сложности программных продуктов росло и количество ошибок в них.
%Повышение роли программного обеспечения в жизнедеятельности человека приводит к увеличению величины последствий от отказа или некорректного поведения.
%Некоторые ошибки способны сильно исказить правильное выполнение программы и привести к серьезным последствиям.
%В 2017 году компания Coverity опубликовала очередной отчет, в котором были приведены результаты исследования изучения 760 млн строк кода.
%В среднем, в программном продукте содержится около 1.4 ошибок на 1000 строк кода.
%Это число ошибок значительно вырасло по сравнению с отчетами предыдущих лет.
%Таким образом, число проблем в проекте размером миллион строк кода исчисляется тысячами.
%Степень критичности ошибки также зависит и от той области, в которой применяется программное обеспечение.
%Ошибки в программных системах, используемых в авиации или на атомных электростанциях, могут привести к значительным человеческим жертвам или колоссальным финансовым затратам.
%Известно множество случаев, в которых ошибки в программном обеспечении приводили к серьезным последствиям, например, случаи с аппаратом лучевой терапии Therac-25 и с космическим аппаратом 
%Phobos\citeinsynopsis{Leveson:1995}.
%Таким образом, с момента возникновения первых вычислительных систем перед людьми всегда стояла задача проверки правильности программы. 

%Под верификацией будем понимать проверку соответствия одних создаваемых в ходе разработки и сопровождения программного обеспечения сущностей другим, ранее созданным или используемым в качестве исходных данных, а также соответствие этих сущностей и процессов их разработки правилам и стандартам [4]. В качестве сущностей могут выступать непосредственно программы, модели, документы и др.
%Существует несколько методов верификации: экспертиза, статический анализ, формальные методы, динамические методы, синтетические методы. 

%Поэтому с самого начала развития информатики развиваются и совершенствуются методы проверки программ и доказательства их корректности. Их уже достаточно много: от ручных математических доказательств, до динамического тестирования. Они прошли долгий путь от формальных математических доказательств, которые были доступны только специалистам, обладающим хорошим математическим образованием, до прикладных инструментов, понятных даже инженерам и разработчикам прикладного программного обеспечения.

%Развитие параллельных вычислений привело к быстрому разрастанию кода, который предполагает параллельное исполнение.
%%Цифры? 
%Параллельные алгоритмы позволяют более эффективно использовать доступные вычислительные ресурсы.
%Основной сложностью таких алгоритмов является обеспечение синхронизации между потоками и корректным использованием разделяемых ресурсов.
%В связи с этим появляются новые типы ошибок, характерные только для параллельно исполняемого кода. 

В дополнении к ошибкам, которые встречаются во всех видах программного обеспечения, многопоточные программы могут содержать специфические ошибки, связанные с параллельным выполнением: например, состояния взаимной блокировки и состояния <<гонки>>.
В общем случае состоянием гонки называют ситуацию, при которой поведение программы зависит от порядка или времени выполнения некоторых неконтролируемых событий.
Важное уточнение заключается в том, что такое выполнение не всегда является ошибкой.
Проблемы возникают тогда, когда разработчик не предусматривает некоторое из возможных поведений программы.
Часто рассматривают более узкий класс -- <<состояния гонки по данным>>. Эта ситуация возникает при одновременном доступе к данным из разных потоков (процессов).
%Здесь и далее не будем различать потоки и процессы, так как их различия не имеют отношения к теме работы.
Состояния гонки по данным становятся опасными в случае, если имеет место хотя бы один доступ на запись в разделяемую область памяти.
%Пример!
В этом случае результирующее значение переменной зависит от порядка выполнения инструкций, а в параллельно выполняемом коде, в общем случае, последовательность выполнения инструкций не определена. 

%Стоит упомянуть о высокоуровневых состояниях гонки по данным.
%Такие состояния гонки по данным отличается тем, что доступ производится к разным разделяемым данным, которые тем не менее являются семантически связанными.
%Например, это может быть реализация сложных структур данных, таких как двусвязные списки, деревья, графы и т. п.
%В случае модификации такой структуры данных должна быть обеспечена атомарность, в противном случае, данные могут стать неконсистентными, например, не нарушится целостность списка. 
 
Состояния взаимных блокировок являются вторым большим классом ошибок в многопоточных программах.
Они возникают при некорректном использовании блокирующих механизмов синхронизации.
В этом случае все потоки системы находятся в ожидании некоторого разблокирующего действия от других потоков и не могут продолжить свое выполнение. 

Оба класса ошибок так или иначе связаны с некорректной синхронизацией параллельно выполняющихся потоков.
Такие ошибки, связанные с параллельным исполнением кода, искать и исправлять гораздо сложнее, чем ошибки в последовательном коде, так как необходимо анализировать все возможные сценарии взаимодействия потоков.
Кроме того, поиск и исправление таких ошибок осложняется случайным характером их проявлений, ошибка может проявляться очень редко.
Это связано с тем, что для проявления ошибки необходимы некоторая конкретная последовательность и порядок действий различных потоков. 
В системном программном обеспечении могут использоваться не только обыкновенные примитивы синхронизации, но и специальные низкоуровневые, например, запреты прерываний и планирования. 
Таким образом, обнаружить ошибку, связанную с некорректной синхронизацией потоков, вручную практически невозможно, поэтому для их поиска используются различные автоматизированные подходы.
Кроме этого, для анализа системного программного обеспечения, например, операционных систем, требутся специализированные инструменты, которые будут учитывать их особенности.

Обычно выделяют два класса подходов к анализу программ: статические, которые проводят анализ исходного кода программы, и динамические, которые анализируют поведение программы в процессе ее выполнения.
Каждый имеет свои достоинства и недостатки, например, методы динамического анализа обычно характеризуются высоким уровнем истинных предупреждений, в то время как статические методы позволяют покрыть больший объем кода.
На данный момент не существует универсального подхода для анализа программ.

В прикладных пользовательских программах может быть достаточно провести тщательное тестирование, возможно, с помощью инструментов динамического анализа, чтобы проверить основные сценарии поведения программы.
%Однако, такое тестирование не дает гарантий корректного поведения даже при тех же самых условиях и входных данных, что для пользовательских приложений является приемлемым вариантом.
В случае же системного программного обеспечения цена пропущенной ошибки может быть слишком велика.
Кроме того, некоторые сценарии поведения программы слишком сложно воспроизвести при реальном выполнении.
Устройство системного программного обеспечения отличается от прикладных программ, что затрудняет анализ, так как далеко не все методы принимают во внимание специфику системного программного обеспечения.
Поэтому применение только динамического анализу становится недостаточным, и в дополнение к ним применяются методы статической верификации.

Методы статической верификации способны обеспечить доказательство отсутствия ошибок в некоторых заранее заданных предположениях, но при этом традиционно испытывают сложности при масштабировании на большие объемы исходного кода.
Тем не менее, такие методы показывают высокую точность, что является важной особенностью при анализе такого программного обеспечения, к которому предъявляются высокие требования качества.
%Развитие подобных методов позволит повысить качество системного программного обеспечения.
Таким образом, разработка масштабируемого метода статической верификации для поиска состояний гонки является актуальной задачей.

%В ядре операционной системы может одновременно выполняться большое число (несколько десятков) совершенно различных функций: обработчики прерываний, системные вызовы от пользовательских программ, внутренние службы ядра, например, планировщик, драйвера внешних устройств.
%Для синхронизации всех этих функций используются не только обыкновенные примитивы синхронизации, но и специальные низкоуровневые, которые характерны только для системного программного обеспечения, например, запреты прерываний и планирования. 
%Различные исследования показывают, что ошибки, связанные с параллельным выполнением, в системном программном обеспечении являются достаточно многочисленными, например, к ним относятся около 20\% всех ошибок в файловых
%системах~\citeinsynopsis{Palix11}.
%
%Верификация многопоточных программ всегда являлась более сложной задачей, чем верификация последовательных программ.
%Точное вычисление всех возможных чередований (англ. interleavings), приводит к комбинаторному взрыву числа состояний.
%Поэтому, большинство инструментов статической верификации применяют арзличные техники оптимизации: редукция частичных порядков (англ. partial order reduction)~\citeinsynopsis{Abdulla:2014},\citeinsynopsis{Godefroid:1996}, абстракция счетчика (англ. counter abstraction)~\citeinsynopsis{Basler:2009} и другие.
%Тем не менее, большинство современных инструментов статической верификации плохо масштабируются на промышленное программное обеспечение.
%Этот факт подтверждается соревнованием по статической верификации~\citeinsynopsis{svcomp19}.
%Задачи из категории «многопоточность», основанные на драйверах операционной системы Linux вызывают значительные сложности для всех инструментов статической верификации.
%
%Противоположностью методам проверки моделей являются методы статического анализа, которые нацелены на быстрый поиск ошибок без какой-либо уверенности в финальном вердикте.
%Такие инструменты применяют различные фильтры и эвристики для ускорения анализа и поэтому не могут гарантировать корректность, то есть отсутствие ошибок.
%В данной работе представлен подход к статической верификации многопоточного программного обеспечения, который позволяет легко настраивать баланс между скоростью и точностью проводимого анализа. Кроме того, он легко нацеливается на конкретную задачу.

%В частности, анализ типовых ошибок, исправленных за год в ядре операционной системы Linux, показал, что ошибки, связанные с состоянием гонки образуют самый многочисленный класс и составляют около 17\% от всех 
%\ifsynopsis
%ошибок.
%\else
%ошибок~\cite{commit_analysis_12}.
%\fi

%Таким образом, задача поиска ошибок синхронизации в ядрах операционных систем, в том числе, состояний гонки по данным, является важной и актуальной задачей.

%\ifsynopsis
%%Этот абзац появляется только в~автореферате.
%\else
%
%%Динамический анализ системного программного обеспечения не обеспечивает должного качества для применения целевого ПО в критически важных областях. В таких случаях необходимо применение формальных методов верификации. Основной сложностью данного подхода является высокая сложность точного моделирования системного программного обеспечения, так как неизбежно возникает проблема комбинаторного взрыва. Тем не менее, сейчас активно применяется формальная верификация для доказательств корректности отдельных подсистем (модулей) программной системы с формулировкой требований на корректное использование его интерфейсов другими частями системы.
%
%Даже при анализе одного потока сложной программной системы возникают сложности с анализом циклов, битовых операций, адресной арифметики. 
%На сколько учитывать окружение и другие потоки?
%\fi

% {\progress} 
% Этот раздел должен быть отдельным структурным элементом по
% ГОСТ, но он, как правило, включается в описание актуальности
% темы. Нужен он отдельным структурынм элемементом или нет ---
% смотрите другие диссертации вашего совета, скорее всего не нужен.

{\aim} данной работы является разработка метода поиска состояний гонок, который будет масштабироваться на большие объемы кода, будет обладать приемлемым уровнем ложных предупреждений и будет учитывать специфику ядра операционных систем.

Для~достижения поставленной цели необходимо было решить следующие {\tasks}:
\begin{enumerate}
  \item Разработать общий алгоритм, позволяющий реализовать подход к верификации программного обеспечения с раздельным анализом потоков, и доказать его корректность;
  \item Разработать метод поиска состояний гонки, как частный случай общего алгоритма, который может эффективно применяться к большим объемам исходного кода, и доказать его корректность;
  \item Реализовать разработанные алгоритмы;
  \item Провести эксперименты и сравнить результаты с другими инструментами статической верификации;
\end{enumerate}

{\novelty}
\begin{enumerate}
  \item Был предложен новый алгоритм, который является обобщением существующего алгоритма CPA, и доказана его корректность;
  \item Был предложен частный случай обобщенного алгоритма, который реализует подход с раздельным анализом потоков, и доказана его корректность;
  \item Разработан метод поиска состояний гонки и доказана его корректность;
  %\item Были проведены запуски инструмента на модулях ядра операционной системы Linux, а также на двух операционных системах реального времени;
\end{enumerate}

%{\influence} \ldots

%{\methods} \ldots

%{\defpositions}
\textbf{Положения, выносимые на публичное представление}
\begin{enumerate}
  \item Обобщенный алгоритм анализа программ, позволяющий использовать различные виды анализа, и доказательство его корректности;
  \item Алгоритм, позволяющий проводить раздельный анализ потоков многопоточных программ, и доказательство его корректности;
  \item Метод поиска состояний гонки в многопоточных программах.
\end{enumerate}

%{\reliability} полученных результатов обеспечивается \ldots \ Результаты находятся в соответствии с результатами, полученными другими авторами.

{\probation}
Основные результаты работы докладывались~на:
\begin{itemize}
  % \item Весенний коллоквиум молодых исследователей в области программной инженерии (SYRCoSE: Spring Young Researchers Colloquium on Software Engineering), Санкт-Петербург, 2014 г.;
  % \item Научно-исследовательский семинар лаборатории «Software and Computational Systems Lab» Университета Пассау, Германия, 2014 г.
  \item Международная научно-практическая конференция <<Инструменты и методы анализа программ>> (TMPA: Tools and Methods for Program Analysis), Кострома, 2014 г.
  \item Международный семинар разработчиков CPAchecker, Москва, 2015 г.
  \item Летняя научная школа компании Microsoft (Microsoft Summer School), Кэмбридж, Англия, 2015 г.
  \item Научно-практическая Открытая конференция ИСП РАН, Москва, 2016 г.
  \item Научно-исследовательский семинар лаборатории <<Software and Computational Systems Lab>> Университета Пассау, Германия, 2016 г.
  \item Международная научно-практическая конференция <<Инструменты и методы анализа программ>> (TMPA: Tools and Methods for Program Analysis), Москва, 2017 г.
  \item Международный семинар разработчиков CPAchecker, Падерборн, Германия, 2017 г.
  \item Международный семинар разработчиков CPAchecker, Москва, Россия, 2018 г.
  \item Соревнования по статической верификации SV-COMP, Прага, Чехия, 2019 г.
  \item Международный семинар разработчиков CPAchecker, Фрауенинзель, Германия, 2019 г.
  \item Семинар <<Математические вопросы информатики>> Мехмат МГУ, Москва, Россия, 2019 г.
\end{itemize}

% {\contribution} Автор принимал активное участие \ldots

\publications\ Основные результаты по теме диссертации изложены в 6 печатных изданиях~\cite{lockatorVAK,lockatorVAK2,TMPA2017,theoryVAK,lockatorSyrcose,lockatorTMPA}, 
    4 из которых изданы в журналах, рекомендованных ВАК~\cite{lockatorVAK,lockatorVAK2,TMPA2017,theoryVAK}, из них 1 находится в базе Scopus~\cite{TMPA2017},
    2 "--- в тезисах докладов~\cite{lockatorSyrcose,lockatorTMPA}.

В статье~\cite{lockatorVAK} автором описана основная идея метода (глава 3) и его реализация (глава 5).
В статье~\cite{lockatorVAK2} автором написаны разделы, посвященные общей идее метода (глава 3), его реализации (глава 4), процессу уточнения (глава 5) и анализу потоков (глава 6).
В статье~\cite{TMPA2017} автором написаны разделы, посвященные разработанному методу и его реализации (главы 3--6).
В статье~\cite{lockatorSyrcose} автором написаны разделы, в которых описывается ключевые особенности метода (главы 3--5).
В статье~\cite{lockatorTMPA} автором написаны разделы, посвященные разработанному методу (главы 3--5).

 % Характеристика работы по структуре во введении и в автореферате не отличается (ГОСТ Р 7.0.11, пункты 5.3.1 и 9.2.1), потому её загружаем из одного и того же внешнего файла, предварительно задав форму выделения некоторым параметрам

%\underline{\textbf{Объем и структура работы.}} Диссертация состоит из~введения, четырех глав, заключения и~приложения. Полный объем диссертации \textbf{ХХХ}~страниц текста с~\textbf{ХХ}~рисунками и~5~таблицами. Список литературы содержит \textbf{ХХX}~наименование.

%\newpage
\section*{Содержание работы}
Во \underline{\textbf{введении}} обосновывается актуальность
исследований, проводимых в~рамках данной диссертационной работы,
формулируется цель, ставятся задачи работы, излагается научная новизна
и практическая значимость представляемой работы.
В~первой главе приводится обзор научной литературы по изучаемой проблеме, во второй главе описывается теоретические основы метода, доказывается его корректность в некоторых предположениях.
В третьей главе описывается схема реализации и архитектура разработанного инструмента.
В четвертой главе представлены результаты апробации на частных примерах: как на множестве тестовых примеров, так и на реальном системном коде.

\underline{\textbf{Первая глава}} посвящена обзору существующих методов поиска ошибок использования различных примитивов синхронизации в многопоточном программном обеспечении.
В разделе 1.1 рассматриваются различные методы статической верификации (англ. software model checking).
Такие методы основаны на том, что автоматически строится некоторая формальная модель программы, а затем эта модель проверяется на соответствие заданным свойствам.
Такие методы являются достаточно точными при условии, что модель достаточно хорошо соответствует исходной программе.
Одним из важных минусов таких методов являются высокие требования к ресурсам.
Другим минусом статической верификации является то, что на реальном программном обеспечении достаточно сложно достигнуть высокого уровня соответствия модели и исходной программы, что приводит к большому количеству ложных срабатываний.
К плюсам можно отнести возможность формального доказательства, что в программе отсутствуют дефекты определенного типа, опять же, при условии адекватности построенной модели.

В разделе 1.2 рассматриваются методы статического анализа.
Эти методы похожи на те, которые были рассмотрены в предыдущем разделе, однако, в отличие от методов проверки моделей, методы статического анализа не ставят целью возможность доказательства отсутствия ошибок в программе.
Основной задачей таких методов является нахождение как можно большего числа ошибок при наименьших затратах ресурсов. 
Многие из методов статического анализа являются корректными, то есть способны не пропускать ошибки, но в этом случае процент ложных срабатываний будет слишком большой.
Большое количество ложных срабатываний сильно затрудняет ручной анализ предупреждений, поэтому зачастую применяются различные фильтры, которые сокращают число ложных сообщений об ошибках.
Однако такие фильтры являются лишь неточными эвристиками, которые снижают корректность метода в целом и способны привести к пропуску реальной ошибки.

В разделе 1.3 рассматриваются методы динамического анализа.
Такие методы применяются для поиска ошибок синхронизации во время работы программы, поэтому одна из основных характеристик таких методов - это замедление целевой программы. 
Одной из характерных особенностей данного метода является возможность гарантировать, что обнаруженное предупреждение является реальной ошибкой, а не ложной. 
Однако, такая возможность доступна, если поиск ошибок производится только на тех путях, которые были получены при конкретном выполнении программы.
Перебрать все варианты выполнения даже небольшой программы невозможно за разумное время, поэтому применяются различные механизмы анализа множества путей, похожих на те, которые возникают при реальном выполнении.
Такой подход значительно ускоряет процесс анализа, но приводит к получению ложных предупреждений. 
Тем не менее, процент ложных предупреждений у инструментов, реализующих методы динамического анализа, обычно значительно ниже, чем у тех, что основаны на других подходах.

В разделе 1.4 подводятся основные итоги обзора и делаются выводы.
Результаты обзора позволяют заключить, что основные усилия сейчас сосредаточены на анализе пользовательских приложений, 
системное программное обеспечение является слишком объемным и слишком сложным для применения общецелевых методов.
Применение методов динамического анализа к системному программному обеспечению осложняется тем, что требуется трудоемкая настройка тестового окружения, в том числе аппаратуры.
Кроме того, методы динамического анализа не способны обеспечить гарантию отсутствия ошибок.
Методы статического анализа успешно применяются к любым объемам кода любой сложности. 
В случае применения таких методов к большому объему сложного кода будет получено огромное количество предупреждений.
Анализ этих предупреждений вручную может потребовать большого количества времени.
Методы статической верификации способны дать гарантию отсутствия дефектов определенного типа, в некоторых разумных предположениях, и, что важно, заранее известных.
Однако, такие методы не способны в настоящее время успешно применяться к большим объемам исходного кода.

Таким образом, можно заключить, что в настоящее время отсутствуют такие методы анализа больших объемов системного кода, в том числе, операционных систем, которые могут обеспечить высокий уровень надежности.
Данная работа посвящена описанию разработанного метода поиска состояний гонки, который может применяться к реальным программным системам.

\underline{\textbf{Вторая глава}} посвящена описанию теории разработанного метода проверки многопоточных программ.
Основная идея заключается в том, что каждый поток в программе анализируется независимо от остальных.
В этом случае удастся избежать комбинаторного взрыва состояний, который бы неизбежно присутствовал, если бы анализ проводился с учетом всех взаимодействий между потоками.
Для того, чтобы обеспечить достаточно высокую точность метода, поток анализируется в некотором окружении, которое формируется другими потоками. 
Уровень точности окружения, который требуется для анализа, может гибко настраиваться.

Метод с раздельным анализом потоков может внутри себя использовать уже существующие техники и подходы, например, такие как CEGAR.
В этом случае необходимо предложить обобщенный алгоритм анализа программ, частным случаем которого уже будут как обычные подходы к анализу последовательных программ, так и различные подходы к анализу многопоточных программ, в частности, предложенный подход с раздельным анализом потоков. 

Для того чтобы доказать корректность предложенного подхода, необходимо определить формальную семантику программы, то есть ее математическую модель.
С помощью предложенного подхода строится некоторая абстракцию этой математической модели. 
Адекватность такой абстракции определяется тем, что каждый из возможных поведений модели должен присутствовать и в абстракции.
Таким образом, ошибка не может быть пропущена.
Именно в этом смысле далее используется термин корректность анализа (англ. soundness).

В разделе 2.1 описана семантика многопоточных программ. 
Формальная модель программы определяется, как множество конкретных состояний программы, которые описываются значениями переменных каждого из потоков, а также глобальных переменных, статусом примитивов синхронизации и информацией об активных потоках.
Далее описывается семантика всех поддерживаемых операторов программы, то есть задаются правила преобразования конкретных состояний операторами программы.
В число поддерживаемых операторов входят операторы сравнения (англ. assumption), присваивания (англ. assignment),
кроме того, специальные операторы работы с примитивами синхронизации: захват блокировки и ее освобождение, а также создание потока (thread\_create)
% и ожидания завершения потока (thread\_join).
В заключении даны определения ошибки в программе, в частности, определение состояния гонки.

В разделе 2.2 описан адаптивный статический анализ (англ. Configurable Program Analysis, CPA), который является формальной математической моделью некоторого статического анализа программы.
CPA определяется доменом абстрактных переходов и операторами над этими абстрактными переходами: оператором перехода $transfer$, оператором объединения $merge$ и оператором останова $stop$.
Каждый из этих операторов должен удовлетворять некоторым условиям для обеспечения корректности.
Одним из важных отличий данного определения от классического варианта теории является использование абстрактных переходов вместо абстрактных состояний.
Это необходимо для возможности построения абстракции не только над конкретными состояниями, но и над конкретными дугами, то есть операциями программы.
Еще одним важным отличием данного определения является ослабление условия на оператор $transfer$, что позволяет определять более сложные CPA.

В разделе 2.3 представлен алгоритм вычисления множества достижимых переходов с помощью некоторого CPA.
Одним из важных особенностей предложенного обобщенного алгоритма является то, что его абстрактные переходы являются частичными, то есть множество конкретных переходов программы, соответствующее некоторому частичному абстрактному переходу, может зависить не только от этого перехода, но и от других частичных переходов. 

Сам алгоритм остается практически неизменным по сравнению с классическим вариантом, за исключением небольшого изменения оператора $transfer$, что является необходимым для обеспечения возможности рассмотрения переходов из любого множества абстрактных состояний. 
Основное отличие возникает при доказательстве корректности этого алгоритма, то есть, утверждения о том, что построенное множество абстрактных переходов аппроксимирует сверху множество конкретных переходов.
Доказательство этой теоремы приведено в приложении.
 
В разделе 2.4 представлен CPA, реализующий функциональность подхода с раздельным анализом потоков.
Этот CPA выполняет не только обычные переходы (т.н. \textit{переходы в потоке}), но и строит специальные \textit{переходы в окружении}, которые представляют собой влияние потоков друг на друга.
Для этого базовый набор операторов CPA расширяется еще тремя: оператором проекции $\cdot|_p$, оператором составления перехода $compose$ и оператором проверки совместности $compatible$.

Переходы в окружении моделируют изменение состояния отдельного потока в результате действий другого потока.
Для построения окружения используется оператор проекции, который представляет как выглядит тот или иной переход для другого потока.
Таким образом, если переход не модифицирует разделяемые данные, его проекцией будет тождественный переход, означающий, что он не может повлиять на состояния других потоков.
Каждая полученная проекция должна быть применена к каждому переходу потока, если это возможно.
Возможность применения задает оператор $compatible$, который, по сути, проверяет могут ли проекция и переход в потоке выполняться параллельно. 
Обычно, эта проверка означает то, что два частичных состояния могут соответствовать одному конкретному состоянию.
В частности, отсюда следует, что все значения разделяемых данных должны совпадать.

Кроме того, в разделе рассмотрен предельный случай CPA, который является инвариантным к переходам в окружении.
Такое свойство позволяет значительно повысить скорость анализа за счет применения более эффективных оптимизаций.

В разделе 2.5 показано, как классический вариант подхода с раздельным рассмотрением потоков без абстракции может быть выражен с использованием предложенной теории. 
Это демонстрирует ее выразительность.

В разделе 2.6 представлено формальное определение CompositeCPA, который обеспечивает композицию различных CPA между собой.
Его домен представляет собой декартово произведение доменов внутренних CPA, а все операторы используют параллельную композицию соответствующих операторов внутренних CPA.
Кроме того, CompositeCPA может усиливать переходы одних CPA за счет информации из других.
Для этого служит специальный оператор $strengthen$.
В данном случае, если некоторый CPA не может определить единственную CFA дугу для следующего перехода, данный CPA может подсказать ему, используя информацию о том, по какой дуге будут переходить следующие CPA.
Такая реализация оператора $strengthen$ является достаточно тривиальной и в дальнейшем может быть усилена.

Разделы 2.7, 2.8, 2.9 посвящены описанию различных вариантов ThreadCPA, который отслеживает множество активных потоков.
Простой его вариант может лишь отличить один поток от другого, но не может определить, может ли он выполняться параллельно, что приводит к снижению точности анализа.
Вариант с использованием эффектов окружения позволяет учитывать зависимости по созданию потоков, а значит, позволяет вычислять совместные переходы более точно, однако не является инвариантным к переходам в окружении.
Наконец, расширенный вариант ThreadCPA позволяет обеспечить необходимую точность анализа при сохранении инвариантности к переходам в окружении.

Раздел 2.10 представляет формальную модель LocationCPA, который отвечает за синтаксическую достижимость точек программы.
Данный CPA обеспечивает связь с исходным кодом программы, а также за определение следующих переходов, которые задаются CFA дугами.

Раздел 2.11 посвящен описанию PredicateCPA, который реализует анализ предикатов в случае подхода с раздельным рассмотрением потоков.
Кроме ожидаемых изменений, связанных с дополнительными операторами, и поддержки переходов в окружении, было необходимо обеспечить возможность анализа нескольких потоков, исполняющих одну функцию. 
Это значит, что при переходах в окружении необходимо переименовывать все встречающиеся локальные переменные для избежания коллизии имен.

Раздел 2.12 представляет описание LockCPA, который отслеживает множество испльзуемых примитивов синхронизации.
Данный CPA является инвариантным к переходам в окружении, и его проекции служат лишь для повышения точности оператора $compatible$, так как два перехода не могут выполняться параллельно друг с другом, если был захвачен одна и та же блокировка.

Раздел 2.13 описывает очень простой анализ явных значений ValueCPA, который может отслеживать лишь присваивания явных значений в переменные.
Он является простым анализом по сравнению с анализом предикатов.

В разделе 2.14 представлен основной метод поиска состояний гонки. 
Он основан на алгоритме Lockset, который определяет состояние гонки, как ситуацию, при которой происходит параллельный доступ к разделяемой памяти с непересекающимся множеством блокировок.
Предложенный метод использует оператор $compatible$ и определяет состояние гонки, как пару \textit{совместных} переходов, которые производят параллельный доступ к разделяемой памяти.
Такой метод сводится к алгоритму Lockset при использовании только LockCPA, однако при анализе реальных программ обычно применяются несколько различных CPA совместно, что позволяет повысить точность определения состояний гонки по сравнению с классическим алгоритмом Lockset.

\underline{\textbf{Третья глава}} посвящена описанию реализации разработанного метода поиска состояний гонки.
Далее будем называть инструмент, реализующий предложенный метод, CPALockator.
В этой же главе представлены решения, которые используются для эффективного анализа программного обеспечения.

В разделе 3.1 представлена общее устройство инфраструктуры CPAchecker: используемый парсер, набор различных алгоритмов и CPA, а также автоматный способ задания ошибки.

В разделе 3.2 описана конфигурация и типовой набор CPA, которые включаются в инструмент CPALockator.
Обычно используются следующие CPA: ThreadModularCPA, ARGCPA, CompositeCPA, LocationCPA, CallstackCPA, LockCPA, ThreadCPA, PredicateCPA.
Однако, такая конфигурация не является единственно возможной, и, в зависимости от целевой задачи, возможно исключение или добавление других CPA.

Раздел 3.3 представляет основные оптимизации, сделанные при реализации ThreadModularCPA. 
Основная оптимизация заключается в переходе от операций над переходами в окружении к операциям над проекциями, которых значительно меньше.
А уже после операций $merge$ и $stop$ над проекциями вычисляются переходы в окружении.

Раздел 3.4 описывает одну из важных оптимизаций BAMCPA, которая основана на кэшировании результатов проведенного анализа некоторого абстрактного блока.
Такая оптимизация существенно позволяет ускорить анализ программы, однако в данный момент она не позволяет использовать CPA, которые не являются инвариантными к переходам в окружении.
Данное ограничение является техническим, и в дальнейшем возможна реализация данной оптимизации в общем варианте.

Раздел 3.5 описывает служебный ARG CPA, который обеспечивает построение абстрактного графа достижимости (англ. Abstract Reachability Graph, ARG).
Абстрактные переходы данного CPA содержат в себе связи, соответсвующие операторам.
Например, связи parent-child означает, что дочерний переход был получен из родительского с помощью оператора $transfer$.

Раздел 3.6 представляет основные отличия реализации LockCPA от формально описанной модели. 
Так как в реальных программах блокировки часто используются по указателю, то одним из основных отличий реализации от теории является поддержка возможности работы с указателями.
Использование анализа алиасов является неоправданным для сложного программного обеспечения, поэтому делается предположение, что работа с блокировками ведется одинаковым образом, и если одна блокировка была захвачена по одному указателю, то она будет освобождаться с использованием этого же указателя.
Еще одной возможностью стала поддержка расширенныго множества операций над блокировками, в том числе рекурсивный захват.
Также в этом разделе описаны особенности реализации, связанные с оптимизацией BAM.

В разделе 3.7 описаны основные особенности реализации LockationCPA, которые заключаются, в основном, в представлении различных служебных дуг, необходимость в которых появилась только для реализации подхода с раздельным рассмотрением потоков.

Раздел 3.8 представляет описание реализации ThreadCPA.
Основным отличием реализации является отсутствие идентификаторов потока на практике. 
Одним из вариантов решения данной проблемы является использование в качестве идентификаторов имя переменной-описателя потока, однако она может быть переприсвоена даже при активном потоке, что является нетипичной ситуацией, однако не является ошибкой.
Другое важное ограничение заключается в ограниченной поддержке операций типа $thread\_join$, которые не были описаны в теории.

Раздел 3.9 посвящен реализации PredicateCPA.
Анализ предикатов традиционно является самым медленным, то есть именно его операторы тратят большую часть времени анализа программы по сравнению с операторами других CPA.
Поэтому задача повышения его эффективности становится очень актуальной.
В разделе описаны используемые оптимизации, которые используются для повышения скорости анализа различными способами.

Одной из важных оптимизаций является настраиваемое кодирование блоков, которое позволяет сократить количество пересчетов абстракции. 
Однако, подход с раздельным рассмотрением потоков накладывает определенные ограничения на применение этой оптимизации.
Другим важным вопросом является представление эффектов окружения таким образом, чтобы можно было эффективно строить логические формулы, соответствующие примененному эффекту.
Основной проблемой при этом является вычисление корректных SSA-индексов, которые должны соответствовать индексам в состоянии потока.

Уточнение абстракции по контрпримерам также имеет некоторые особенности из-за подхода с раздельным анализом потоков.
В первую очередь возникают проблемы при восстановлении глобального пути, а также при его представлении, так как эффект окружения может не описываться ни одной CFA дугой.
Многие эти проблемы требуют радикального переосмысления механизма уточнение в фреймворке CPAchecker, поэтому не все они были решены в рамках данной работы.

В разделе 3.10 описаны основные особенности реализации CompositeCPA, которые заключаются, в основном, в удобной комбинации CPA, которые реализуют подход с раздельным рассмотрением потоков, и тех, которые являются инвариантными к переходам в окружении, а значит, могут не предоставлять расширенное множество операторов. 
Кроме того, в разделе описаны отличия при реализации оператора $strengthen$.

В разделе 3.11 описан UsageCPA, который предоставляет дополнительные возможности для настройки инструмента на целевой код, а также для поиска высокоуровневых гонок с помощью аннотаций.

Раздел 3.12 посвящен еще одной важной оптимизации -- анализу разделяемых данных, который применяется непосредственно перед основным анализом для поиска разделяемых данных.
Так, если какая-то область памяти не является разделяемой, то для нее не будут выданы предупреждения в дальнейшем.
Анализ является консервативным, то есть если не доказано, что какая-то область памяти является локальной, то считается, что она может быть разделяемой.

В разделе 3.13 представлены основные особенности при вычислении состояний гонки при анализе реальных программ.
Важным отличием является активное использование указателей, однако использование анализа алиасов является очень неэффективным.
Кроме того, анализ алиасов требует явной инициализации каждого указателя. Поэтому для решения данной проблемы была использована специализированная модель памяти BnB, основанная на разделении памяти на непересекающиеся регионы по типам.
Кроме того, каждое поле структуры, от которого не брался адрес выделяется в отдельный регион.
Такая модель памяти позволяет достаточно точно вычислять потенциальные состояния гонки.

Однако, сам процесс вычисления устроен сложным образом, так как необходимо сначала восстановить всю информацию о доступах к данным, потерянную из-за применения оптимизации BAM, а затем обеспечить эффективное хранение, поиск и модификацию этой информации.
Для этого применяются различные оптимизации, связанные с выделением приоритетной информации и упорядочивании ее. 

Раздел 3.14 содержит описание процесса уточнения при поиске состояния гонки.
Так как полученная абстракция программы может является аппроксимацией сверху множества состояний программы, возможны ситуации, при которых некоторые важные детали не будут учтены. 
Это приведет к тому, что будет выдано ложное сообщение об ошибке. 
Процесс уточнения позволяет повысить степень уверенности в том, что найденная ошибка является истинной.
Например, может быть проверена локальная достижимость каждого из путей, участвующих в состоянии гонки.
Этот процесс требует достаточно большого количества ресурсов, однако может быть прерван в любой момент.

Основным отличием от классического варианта является то, состояние гонки определяется парой доступов, соответственно, возникает необходимость уточнения пары путей.
Кроме того, так как процесс уточнения запускается после полного построения абстракции, имеется большой выбор состояний для уточнения, а уточнение всех путей подряд приведет к большому количеству однообразных предикатов.
Для решения этой проблемы используются различные эвристики для определения значимости уточнения каждого конкретного пути.

В разделу 3.15 представлено описание формата визуализации результатов.
Для визуализации используется формат GraphML, который уже применяется в других инструментах статической верификации.
В этом разделе описан механизм построения трассы, приводящей к состоянию гонки, а также показана возможность их визуализации.

%Можно сослаться на свои работы в автореферате. Для этого в файле
%\verb!Synopsis/setup.tex! необходимо присвоить положительное значение
%счётчику \verb!\setcounter{usefootcite}{1}!. В таком случае ссылки на
%работы других авторов будут подстрочными.
%\ifnumgreater{\value{usefootcite}}{0}{
%Изложенные в третьей главе результаты опубликованы в~\cite{vakbib1, vakbib2}.
%}{}

В \underline{\textbf{четвертой главе}} приведены результаты экспериментов.
Сравнение проводилось на множестве задач SV-COMP, множестве задач, подготовленных на основе подсистемы \textit{drivers/net} ОС Linux, а также двух ядрах закрытых операционных систем реального времени.

В разделе 4.1 описана общая схема проведения экспериментов, описаны использованные задачи, конфигурации и машины, на которых производился запуск.

В разделе 4.2 представлены результаты сравнения различных инструментов верификации: CPALockator и участников соревнования SV-COMP.
Основные выводы заключаются в том, что CPALockator продемонстрировал эффективную работу, однако его точность значительно ниже, чем у других инструментов.
Однако, никто из них не решил те несколько примеров, которые основаны на драйверах ОС Linux, кроме CPALockator.
Таким образом, можно заключить, что CPALockator экспериментально подтверждает то, что он не пропускает ошибок, а также то, что он эффективно решает сложные задачи.

В разделе 4.3 представлены результаты сравнения различных вариантов реализации PredicateCPA: различные варианты оператора $merge$, а также использование оптимизаций.
Результаты показывают, что использование оператора $merge_{Join}$ является предпочтительным для большинства задач.
Среди оптимизаций стоит выделить оптимизацию ABE, которая улучшает эффективность анализа любого исходного кода.
Некоторые другие оптимизации имеет смысл применять при анализе системного программного кода, в то время как на искусственных тестах SV-COMP они значительно ухудшают показатели.
Другие оптимизации позволяют значительно ускорить анализ, однако приводят к появлению ложных сообщений об ошибках.

В разделе 4.4 представлены результаты сравнения различных вариантов реализации ThreadCPA.
Эксперименты подтверждают, что расширенный вариант ThreadCPA, инвариантный к переходам в окружении, способен обеспечить сопоставимую точность с вариантом с эффектами окружения.
Тем не менее, простой анализ потоков является достаточным для тех случаев, в которых создание потоков производится искусственным образом в модели окружения.

Различные варианты обработки повторно создаваемого потока также демонстрируют различные результаты в зависимости от целевого исходного кода.
Например, при анализе исходного кода ОС РВ все три способа показывают похожие результаты, в то время как на наборе SV-COMP они проявляют себя совершенно по-разному. 

Общий вывод по результатам заключается в том, что по возможности нужно использовать простой вариант ThreadCPA, который является корректным при некоторых дополнительных предположениях.
Однако, если есть сомнения в выполнении данных предположений для конкретного исходного кода, тогда следует использовать тот вариант, который гарантирует корректность.

В разделе 4.5 приведено сравнение различных вариантов реализации LockCPA.
Сравнение различных вариантов реализации оператора $merge$ не выявило существенных отличий ни на каком наборе задач.
Это демонстрирует невысокую актуальность данного варианта.
Различные варианты оптимизации BAM показывают свою эффективность только на большом объеме исходного кода, однако в этом случае становятся принципиально важными.
Использование уточнения, наоборот, снижает эффективность анализа на любом тестовом наборе, что подтверждает предположение, что затраты на уточнение не перекрывают выгоду от менее точной начальной абстракции.

В разделе 4.6 рассмотрено сравнение вклада в точность и эффективность анализа дополнительных CPA.
Оптимизация BAM в данный момент ограничена только CPA, инвариантными к переходам в окружении.
Результаты демонстрируют ее высокую эффективность, кроме того большие объемы исходного кода могут быть проанализированы только с ее помощью.

Применение анализа разделяемых данных в данной реализации имеет смысл только для решения задачи поиска состояния гонки.
При этом даже для задач на основе драйверов ОС Linux анализ разделяемых данных не показывает существенного улучшения из-за анализа потоков.
Но для ОС РВ анализ разделяемых данных способен существенно снизить количество ложных предупреждений об ошибке.

Анализ предикатов обеспечивает высокую точность, однако для получения легковесного варианта инструмента его можно исключить из конфигурации, это приведет к значительному повышению скорости анализа за счет снижения точности.

В разделе 4.7 представлены результаты поиска известных ошибок в драйверах ОС Linux.
Для этого был проведен анализ исправлений в стабильных версиях ядра Linux, и были найдены 18 коммитов, которые исправляют ошибки, связанные с состоянием гонки.
Две соответствующие ошибки находятся инструментом CPALockator, еще в двух случаях инструмент находят нецелевые ошибки, то есть реальные, но не те, которые исправляются в коммите.
В трех случаях ошибка не является состоянием гонки (хотя проявляется при многопоточном выполнении), а в четырех случаях ошибка является высокоуровневой гонкой, т. е. в этих случаях ошибка не попадает под определение гонки, которое использует инструмент.
В трех случаях ошибка отсутствует в подготовленной задаче из-за неточности модели окружения.
В трех случаях ошибка не была найдена из-за исчерпания лимита времени.
И в одном случае ошибка не была найдена из-за ограничений инструмента, а именно, неявного использования разделяемых данных.

В разделе 4.8 приведены результаты анализа основных причин ложных срабатываний инструмента.
Как и следовало ожидать, основные причины ложных срабатываний приходятся на неточную модель окружения и на неточную модель памяти.
Менее значимые причины включают в себя неточности различных компонентов инструмента: анализ примитивов синхронизации, анализ предикатов, анализ разделяемых данных и~т.~д.

В разделе 4.9 сформулированы основные выводы по результатам всех экспериментов: основные сценарии использования различных конфигураций и оптимизаций в зависимости от целевого исходного кода.
Кроме того, показаны основные проблемные места на основе анализа причин ложных срабатываний и результатов поиска известных ошибок.

%Результаты демонстрируют возможности инструмента CPALockator по его гибкой настройки.
%В качестве набора небольших программ был выбран открытый тестовый набор sv-comp, на нем CPALockator показал достаточно точные результаты. 
%Было получено некоторое количество ложных предупреждений об ошибке, но ни одной реальной ошибки не было пропущено, что подтверждает корректность разработанного метода.
%При этом, конечно, другие методы статической верификации способны выдавать меньшее количество ложных предупреждений на таком наборе небольших рукописных программ.
%Для проверки масштабируемости инструмента были выбраны несколько модулей операционной системы Linux с заранее известными ошибками.
%Более половины из этих ошибок были успешно обнаружены инструментами, при этом в оставшихся случаях верификация завершилась из-за нехватки вычислительных ресурсов.
%Если снизить еще снизить требования к точности, все реальные ошибки успешно обнаруживаются, хотя при этом появляется некоторое количество ложных предупреждений.
%
%Гибкая настройка требований к ресурсам позволяет использовать инструмент для проверки достаточно больших программных систем, содержащих сотни тысяч строк кода и проводить их анализ за минуты.
%При необходимости для каждой конкретной задачи можно выбрать свой баланс между качеством и требуемыми ресурсами, в зависимости от поставленной цели. 

В \underline{\textbf{заключении}} приведены основные результаты работы, которые заключаются в следующем:
%% Согласно ГОСТ Р 7.0.11-2011:
%% 5.3.3 В заключении диссертации излагают итоги выполненного исследования, рекомендации, перспективы дальнейшей разработки темы.
%% 9.2.3 В заключении автореферата диссертации излагают итоги данного исследования, рекомендации и перспективы дальнейшей разработки темы.
\begin{enumerate}
  \item Был разработан общий алгоритм анализа, позволяющий объединить в себе как классический алгоритм CPA, так и алгоритм с раздельным анализом потоков;
  \item Была доказана теорема, позволяющая гарантировать корректность работы алгоритма;
  \item На основе общего алгоритма был разработан алгоритм с раздельным анализом потоков;
  \item Был разработан метод поиска состояний гонки, базирующийся на подходе с раздельным анализом потоков;
  \item Была доказана корректность предложенного метода поиска состояний гонки, который позволяет гибко балансировать точность и скорость анализа;
  \item Были проведены эксперименты, которые показали высокую степень гибкости разработанного метода.
\end{enumerate}



%\newpage
%При использовании пакета \verb!biblatex! список публикаций автора по теме
%диссертации формируется в разделе <<\publications>>\ файла
%\verb!../common/characteristic.tex!  при помощи команды \verb!\nocite! 

\ifdefmacro{\microtypesetup}{\microtypesetup{protrusion=false}}{} % не рекомендуется применять пакет микротипографики к автоматически генерируемому списку литературы
\ifnumequal{\value{bibliosel}}{0}{% Встроенная реализация с загрузкой файла через движок bibtex8
  \renewcommand{\bibname}{\large \authorbibtitle}
  \nocite{*}
  \insertbiblioauthor           % Подключаем Bib-базы
  %\insertbiblioother   % !!! bibtex не умеет работать с несколькими библиографиями !!!
}{% Реализация пакетом biblatex через движок biber
  \ifnumgreater{\value{usefootcite}}{0}{
%  \nocite{*} % Невидимая цитата всех работ, позволит вывести все работы автора
  \insertbiblioauthorcited      % Вывод процитированных в автореферате работ автора
  }{
  \insertbiblioauthor           % Вывод всех работ автора
%  \insertbiblioauthorgrouped    % Вывод всех работ автора, сгруппированных по источникам
%  \insertbiblioauthorimportant  % Вывод наиболее значимых работ автора (определяется в файле characteristic во второй section)
  \insertbiblioother            % Вывод списка литературы, на которую ссылались в тексте автореферата
  }
}
\ifdefmacro{\microtypesetup}{\microtypesetup{protrusion=true}}{}

